% Thank you Josh Davis for this template!
% https://github.com/jdavis/latex-homework-template/blob/master/homework.tex

\documentclass{article}

\newcommand{\hmwkTitle}{HW\ \#7}

% % Packages

\usepackage{fancyhdr}
\usepackage{extramarks}
\usepackage{amsmath}
\usepackage{amssymb}
\usepackage{amsthm}
\usepackage{amsfonts}
\usepackage{tikz}
\usepackage[plain]{algorithm}
\usepackage{algpseudocode}
\usepackage{enumitem}
\usepackage{chngcntr}

% Libraries

\usetikzlibrary{automata, positioning, arrows}

%
% Basic Document Settings
%

\topmargin=-0.45in
\evensidemargin=0in
\oddsidemargin=0in
\textwidth=6.5in
\textheight=9.0in
\headsep=0.25in

\linespread{1.1}

\pagestyle{fancy}
\lhead{\hmwkAuthorName}
\chead{}
\rhead{\hmwkClass\ (\hmwkClassInstructor): \hmwkTitle}
\lfoot{\lastxmark}
\cfoot{\thepage}

\renewcommand\headrulewidth{0.4pt}
\renewcommand\footrulewidth{0.4pt}

\setlength\parindent{0pt}
\setcounter{secnumdepth}{0}

\newcommand{\hmwkClass}{MATH 3380 / Analysis 1}        % Class
\newcommand{\hmwkClassInstructor}{Dr. Welsh}           % Instructor
\newcommand{\hmwkAuthorName}{\textbf{Joshua Mitchell}} % Author

%
% Title Page
%

\title{
    \vspace{2in}
    \textmd{\textbf{\hmwkClass:\ \hmwkTitle}}\\
    \normalsize\vspace{0.1in}\small\vspace{0.1in}\large{\textit{\hmwkClassInstructor}}
    \vspace{3in}
}

\author{\hmwkAuthorName}
\date{}

\renewcommand{\part}[1]{\textbf{\large Part \Alph{partCounter}}\stepcounter{partCounter}\\}

% Integral dx
\newcommand{\dx}{\mathrm{d}x}

%
% Various Helper Commands
%

% For derivatives
\newcommand{\deriv}[1]{\frac{\mathrm{d}}{\mathrm{d}x} (#1)}

% For partial derivatives
\newcommand{\pderiv}[2]{\frac{\partial}{\partial #1} (#2)}


% Alias for the Solution section header
\newcommand{\solution}{\textbf{\large Solution}}

% Probability commands: Expectation, Variance, Covariance, Bias
\newcommand{\E}{\mathrm{E}}
\newcommand{\Var}{\mathrm{Var}}
\newcommand{\Cov}{\mathrm{Cov}}
\newcommand{\Bias}{\mathrm{Bias}}

% Formatting commands:

\newcommand{\mt}[1]{\ensuremath{#1}}
\newcommand{\nm}[1]{\textrm{#1}}

\newcommand\bsc[2][\DefaultOpt]{%
  \def\DefaultOpt{#2}%
  \section[#1]{#2}%
}
\newcommand\ssc[2][\DefaultOpt]{%
  \def\DefaultOpt{#2}%
  \subsection[#1]{#2}%
}
\newcommand{\bgpf}{\begin{proof} $ $\newline}

\newcommand{\bgeq}{\begin{equation*}}
\newcommand{\eeq}{\end{equation*}}	

\newcommand{\balist}{\begin{enumerate}[label=\alph*.]}
\newcommand{\elist}{\end{enumerate}}

\newcommand{\bilist}{\begin{enumerate}[label=\roman*)]}	

\newcommand{\bgsp}{\begin{split}}
% \newcommand{\esp}{\end{split}} % doesn't work for some reason.

\newcommand\prs[1]{~~~\textbf{(#1)}}

\newcommand{\lt}[1]{\textbf{Let: } #1}
     							   %  if you're setting it to be true
\newcommand{\supp}[1]{\textbf{Suppose: } #1}
     							   %  Suppose (if it'll end up false)
\newcommand{\wts}[1]{\textbf{Want to show: } #1}
     							   %  Want to show
\newcommand{\as}[1]{\textbf{Assume: } #1}
     							   %  if you think it follows from truth
\newcommand{\bpth}[1]{\textbf{(#1)}}

\newcommand{\step}[2]{\begin{equation}\tag{#2}#1\end{equation}}
\newcommand{\epf}{\end{proof}}

\newcommand{\sidenote}[1]{-----------------------------------------------------------------Side Notes---------------------------------------------------------------
#1 \

---------------------------------------------------------------------------------------------------------------------------------------------}

% Analysis / Logical commands:

\newcommand{\br}{\mathbb{R}}       % |R
\newcommand{\bq}{\mathbb{Q}}       % |Q
\newcommand{\bn}{\mathbb{N}}       % |N
\newcommand{\bc}{\mathbb{C}}       % |C
\newcommand{\bz}{\mathbb{Z}}       % |Z

\newcommand{\ep}{\epsilon}         % epsilon
\newcommand{\fa}{\forall}          % for all

\newcommand{\es}{\emptyset}        % empty set
\newcommand{\sbs}{\subset}         % subset of

\newcommand{\lra}{\longrightarrow} % implies ----->
\newcommand{\rar}{\Rightarrow}     % implies -->

\newcommand{\lla}{\longleftarrow}  % implies <-----
\newcommand{\lar}{\Leftarrow}      % implies <--

\newcommand{\pr}{^\prime} 		   % prime (i.e. R')

\newcommand{\bnm}[2]{\mt{#1\setminus{#2}}}
\newcommand{\bnt}[2]{\mt{\textrm{#1}\setminus{\textrm{#2}}}}
\newcommand{\bi}{\bnm{\mathbb{R}}{\mathbb{Q}}}

\newcommand{\nbho}[3]{\textrm{N(}#1, #2\textrm{) }\cap \textrm{ #3} \neq \emptyset}
     							   %  N(x, eps) intersect S \= emptyset
\newcommand{\nbhe}[3]{\textrm{N(}#1, #2\textrm{) }\cap \textrm{ #3} = \emptyset}
     							   %  N(x, eps) intersect S  = emptyset
\newcommand{\dnbho}[3]{\textrm{N*(}#1, #2\textrm{) }\cap \textrm{ #3} \neq \emptyset}
     							   %  N*(x, eps) intersect S \= emptyset
\newcommand{\dnbhe}[3]{\textrm{N*(}#1, #2\textrm{) }\cap \textrm{ #3} = \emptyset}
     							   %  N*(x, eps) intersect S = emptyset
     							 

% ----------

% Packages

\usepackage{fancyhdr}
\usepackage{extramarks}
\usepackage{amsmath}
\usepackage{amssymb}
\usepackage{amsthm}
\usepackage{amsfonts}
\usepackage{tikz}
\usepackage[plain]{algorithm}
\usepackage{algpseudocode}
\usepackage{enumitem}
\usepackage{chngcntr}

% Libraries

\usetikzlibrary{automata, positioning, arrows}

%
% Basic Document Settings
%

\topmargin=-0.45in
\evensidemargin=0in
\oddsidemargin=0in
\textwidth=6.5in
\textheight=9.0in
\headsep=0.25in

\linespread{1.1}

\pagestyle{fancy}
\lhead{\hmwkAuthorName}
\chead{}
\rhead{\hmwkClass\ (\hmwkClassInstructor): \hmwkTitle}
\lfoot{\lastxmark}
\cfoot{\thepage}

\renewcommand\headrulewidth{0.4pt}
\renewcommand\footrulewidth{0.4pt}

\setlength\parindent{0pt}
\setcounter{secnumdepth}{0}

\newcommand{\hmwkClass}{MATH 3380 / Analysis 1}        % Class
\newcommand{\hmwkClassInstructor}{Dr. Welsh}           % Instructor
\newcommand{\hmwkAuthorName}{\textbf{Joshua Mitchell}} % Author

%
% Title Page
%

\title{
    \vspace{2in}
    \textmd{\textbf{\hmwkClass:\ \hmwkTitle}}\\
    \normalsize\vspace{0.1in}\small\vspace{0.1in}\large{\textit{\hmwkClassInstructor}}
    \vspace{3in}
}

\author{\hmwkAuthorName}
\date{}

\renewcommand{\part}[1]{\textbf{\large Part \Alph{partCounter}}\stepcounter{partCounter}\\}

% Integral dx
\newcommand{\dx}{\mathrm{d}x}

%
% Various Helper Commands
%

% For derivatives
\newcommand{\deriv}[1]{\frac{\mathrm{d}}{\mathrm{d}x} (#1)}

% For partial derivatives
\newcommand{\pderiv}[2]{\frac{\partial}{\partial #1} (#2)}


% Alias for the Solution section header
\newcommand{\solution}{\textbf{\large Solution}}

% Probability commands: Expectation, Variance, Covariance, Bias
\newcommand{\E}{\mathrm{E}}
\newcommand{\Var}{\mathrm{Var}}
\newcommand{\Cov}{\mathrm{Cov}}
\newcommand{\Bias}{\mathrm{Bias}}

% Formatting commands:

\newcommand{\mt}[1]{\ensuremath{#1}}
\newcommand{\nm}[1]{\textrm{#1}}

\newcommand\bsc[2][\DefaultOpt]{%
  \def\DefaultOpt{#2}%
  \section[#1]{#2}%
}
\newcommand\ssc[2][\DefaultOpt]{%
  \def\DefaultOpt{#2}%
  \subsection[#1]{#2}%
}
\newcommand{\bgpf}{\begin{proof} $ $\newline}

\newcommand{\bgeq}{\begin{equation*}}
\newcommand{\eeq}{\end{equation*}}	

\newcommand{\balist}{\begin{enumerate}[label=\alph*.]}
\newcommand{\elist}{\end{enumerate}}

\newcommand{\bilist}{\begin{enumerate}[label=\roman*)]}	

\newcommand{\bgsp}{\begin{split}}
% \newcommand{\esp}{\end{split}} % doesn't work for some reason.

\newcommand\prs[1]{~~~\textbf{(#1)}}

\newcommand{\lt}[1]{\textbf{Let: } #1}
     							   %  if you're setting it to be true
\newcommand{\supp}[1]{\textbf{Suppose: } #1}
     							   %  Suppose (if it'll end up false)
\newcommand{\wts}[1]{\textbf{Want to show: } #1}
     							   %  Want to show
\newcommand{\as}[1]{\textbf{Assume: } #1}
     							   %  if you think it follows from truth
\newcommand{\bpth}[1]{\textbf{(#1)}}

\newcommand{\step}[2]{\begin{equation}\tag{#2}#1\end{equation}}
\newcommand{\epf}{\end{proof}}

\newcommand{\dbs}[3]{\mt{#1_{#2_#3}}}

\newcommand{\sidenote}[1]{-----------------------------------------------------------------Side Note----------------------------------------------------------------
#1 \

---------------------------------------------------------------------------------------------------------------------------------------------}

% Analysis / Logical commands:

\newcommand{\br}{\mt{\mathbb{R}} }       % |R
\newcommand{\bq}{\mt{\mathbb{Q}} }       % |Q
\newcommand{\bn}{\mt{\mathbb{N}} }       % |N
\newcommand{\bc}{\mt{\mathbb{C}} }       % |C
\newcommand{\bz}{\mt{\mathbb{Z}} }       % |Z

\newcommand{\ep}{\mt{\epsilon} }         % epsilon
\newcommand{\fa}{\mt{\forall} }          % for all
\newcommand{\afa}{\mt{\alpha} }
\newcommand{\bta}{\mt{\beta} }
\newcommand{\mem}{\mt{\in} }
\newcommand{\exs}{\mt{\exists} }

\newcommand{\es}{\mt{\emptyset} }        % empty set
\newcommand{\sbs}{\mt{\subset} }         % subset of
\newcommand{\fs}[2]{\{\uw{#1}{1}, \uw{#1}{2}, ... \uw{#1}{#2}\}}

\newcommand{\lra}{ \mt{\longrightarrow} } % implies ----->
\newcommand{\rar}{ \mt{\Rightarrow} }     % implies -->

\newcommand{\lla}{ \mt{\longleftarrow} }  % implies <-----
\newcommand{\lar}{ \mt{\Leftarrow} }      % implies <--

\newcommand{\av}[1]{\mt{|}#1\mt{|}}  % absolute value

\newcommand{\prn}[1]{(#1)}
\newcommand{\bk}[1]{\{#1\}}

\newcommand{\ps}{\mt{+} }
\newcommand{\ms}{\mt{-} }

\newcommand{\ls}{\mt{<} }
\newcommand{\gr}{\mt{>} }

\newcommand{\lse}{\mt{\leq} }
\newcommand{\gre}{\mt{\geq} }

\newcommand{\eql}{\mt{=} }

\newcommand{\pr}{\mt{^\prime} } 		   % prime (i.e. R')
\newcommand{\uw}[2]{#1\mt{_{#2}}}
\newcommand{\uf}[2]{#1\mt{^{#2}}}
\newcommand{\frc}[2]{\mt{\frac{#1}{#2}}}
\newcommand{\lmti}[1]{\mt{\displaystyle{\lim_{#1 \to \infty}}}}
\newcommand{\limt}[2]{\mt{\displaystyle{\lim_{#1 \to #2}}}}

\newcommand{\bnm}[2]{\mt{#1\setminus{#2}}}
\newcommand{\bnt}[2]{\mt{\textrm{#1}\setminus{\textrm{#2}}}}
\newcommand{\bi}{\bnm{\mathbb{R}}{\mathbb{Q}}}

\newcommand{\urng}[2]{\mt{\bigcup_{#1}^{#2}}}
\newcommand{\nrng}[2]{\mt{\bigcap_{#1}^{#2}}}
\newcommand{\nck}[2]{\mt{{#1 \choose #2}}}

\newcommand{\nbho}[3]{\textrm{N(}#1, #2\textrm{) }\cap \textrm{ #3} \neq \emptyset}
     							   %  N(x, eps) intersect S \= emptyset
\newcommand{\nbhe}[3]{\textrm{N(}#1, #2\textrm{) }\cap \textrm{ #3} = \emptyset}
     							   %  N(x, eps) intersect S  = emptyset
\newcommand{\dnbho}[3]{\textrm{N*(}#1, #2\textrm{) }\cap \textrm{ #3} \neq \emptyset}
     							   %  N*(x, eps) intersect S \= emptyset
\newcommand{\dnbhe}[3]{\textrm{N*(}#1, #2\textrm{) }\cap \textrm{ #3} = \emptyset}
     							   %  N*(x, eps) intersect S = emptyset
     							   
\newcommand{\eqn}[1]{\[#1\]}
\newcommand{\splt}[1]{\begin{split}#1\end{split}}
     							 
% ----------

\begin{document}

Homework 7: pages 184 - 185 numbers 1, 2(a)(b), 3(e), 4, 10, 13, 14 \lla 14 is difficult, but not impossible! (want to show that lim (1 \ps \frc{1}{n})$^n$ exists)

Hint:

\uf{\prn{1 \ps b}}{n} \eql 1 \ps nb \ps \frc{n(n - 1}{2!}\uf{b}{n} \ps ... \ps \frc{n(n - 1)... (n - (r - 1))}{r!}\uf{b}{r} \ps ... \ps \uf{b}{n}

In our problem, b \eql \frc{1}{n}

Look at it as 1 \ps $\sum_{r = 1}^n$ \frc{n(n - 1)...(n - (r - 1))}{r!}\frc{1}{n^r}

\prn{1 \ps \frc{1}{n}}$^n$ goes in there somewhere somehow.

\bsc{Problem 1}{
Mark each statement True or False. Justify each answer.
\balist
\item If a monotone sequence is bounded, then it is convergent.
	
	\textbf{True}
	
	by Theorem 4.3.3
\item If a bounded sequence is monotone, then it is convergent.
	
	\textbf{True}
	
	by Theorem 4.3.3
\item If a convergent sequence is monotone, then it is bounded.
	
	\textbf{True}
	
	by Theorem 4.3.3
\elist
}

\bsc{Problem 2(a)(b)}{
Mark each statement True or False. Justify each answer.
\balist
\item If a convergent sequence is bounded, then it is monotone.
	
	\textbf{False.}
	
	Counterexample: \uw{s}{n} \eql
	\eqn{(-1)^n\frac{1}{n}}
\item If \prn{\uw{s}{n}} is an unbounded increasing sequence, then lim \uw{s}{n} \eql $+\infty$
	
	\textbf{True.}
	
	By Theorem 4.3.8
	
	\as{\prn{\uw{s}{n}} is an unbounded, increasing sequence.}
	
	Then, \fa n \mem \bn, \uw{s}{n} \lse \uw{s}{n + 1}
	
	and
	
	\fa m \mem \br, \exs N \mem \bn st. n \gre N implies \uw{s}{n} \gr m
	
	By Definition 4.2.9:
	
	We say a sequence diverges to $\infty$ if \fa m \mem \br, \exs N \mem \bn st n \gre N implies \uw{s}{n} \gr m
	
	Hence, result.
	
	
\elist

}

\bsc{Problem 3(e)}{
Prove that each sequence is monotone and bounded. Then, find the limit.

\bpth{e} \uw{s}{1} \eql 5 and \uw{s}{n + 1} \eql $\sqrt{4\uw{s}{n} + 1}$ for n \gre 1

\uw{s}{n} is monotone if it's either increasing or decreasing.

\uw{s}{1} \eql 5, \uw{s}{2} \eql $\sqrt{21}$ \eql 4.58257569496, \uw{s}{3} \eql $\sqrt{4\sqrt{21} + 1}$ \eql $\sqrt{\sqrt{336} + 1}$ \eql $\sqrt{19.3303028}$ \eql 4.39662402304

Hmm, limit's probably 4. Let's see. \

\

\textbf{Conjecture}

\bk{\uw{s}{n}} is decreasing and 4 \lse \uw{s}{n} \lse 5, \fa n \mem \bn

P(n) (Proposition as a function of n):

\uw{s}{n} \gre \uw{s}{n + 1}, \fa n \mem \bn 

\uw{s}{1} \eql 5 \gr $\sqrt{21}$ \eql \uw{s}{2}

Suppose that, \fa k \mem \bn,

\eqn{s_{k} = \sqrt{4\uw{s}{k - 1} + 1} \gre \sqrt{4\uw{s}{k} + 1} = s_{k + 1}}

Now,

\eqn{s_{k + 1} = \sqrt{4\uw{s}{k} + 1} \gre \sqrt{4\uw{s}{k + 1} + 1} = s_{k + 2}}

So,

\eqn{s_k \gre s_{k + 1}}

Hence, by induction, P(n): \uw{s}{n} \gre \uw{s}{n + 1} is true \fa n \mem \bn 

Q(n): \uw{s}{n} \gre 4 \fa n \mem \bn

\uw{s}{1} \eql 5 \gr 4

Assume for k \mem \bn that \uw{s}{k} \gr 4

\eqn{s_{k + 1} = \sqrt{4\uw{s}{k} + 1} \gr \sqrt{4(3.75) + 1} = 4}

Hence, by induction, Q(n): \uw{s}{n} \gr 4 is true \fa n \mem \bn 

By the Montone Convergence Theorem,

\exs s \mem \br st

\lmti{n} \uw{s}{n} \eql s

By HW problem 11, page 170.

Thus,

\lmti{n} \uw{s}{n + 1} \eql \lmti{n} \uw{s}{n} \eql s

So, we claim that \lmti{n} \uw{s}{n + 1} \eql s \eql \lmti{n} $\sqrt{4s_n + 1}$ \eql $\sqrt{4s + 1}$

From Example 4.2.6,

\lmti{n} $\sqrt{\uw{t}{n}}$ \eql $\sqrt{t}$ if \lmti{n} \uw{t}{n} \eql t

Also, by Theorem 4.2.1 (b), \lmti{n} $\sqrt{1 + s_n}$ \eql $\sqrt{1 + s}$

(which is like saying \lmti{n} \uw{t}{n} \eql t)

Hence,

\eqn{
	\splt{
		s & = \sqrt{4s + 1} \\
		s^2 & = 4s + 1 \\
		s^2 - 4s - 1 & = 0 \\
		s & = \frac{4 \pm \sqrt{20}}{2}
	}
}

But one of those limits can't be true since limits are unique.

Since \uw{s}{n} \gre 0, \fa n \mem \bn,

then \lmti{n} \uw{s}{n} \eql s \gre 0, \fa n \mem \bn 

(By Corollary 4.2.5)

Hence, 

s \eql $\frac{4 + \sqrt{20}}{2}$ \eql $2 + \sqrt{5}$

\bsc{Problem 4}{
Find an example of a sequence of real numbers satisfying each set of properties.
\balist
\item Cauchy, but not monotone.
	
	\uw{s}{n} \eql \prn{\ms1}$^{n}$\frc{1}{n}
\item Monotone, but not cauchy.

	\uw{s}{n} \eql n
\item Bounded, but not cauchy.
	
	\uw{s}{n} \eql \prn{\ms1}$^{n}$
\elist
}

\bsc{Problem 10}{
\balist
\item Suppose that \av{r} \ls 1. Recall from Exercise 3.1.7 that
\eqn{1 + r + r^2 + ... r^n = \frac{1 - r^{n - 1}}{1 - r}}
	Find \lmti{n} (1 \ps r \ps r$^2$ \ps ... \uf{r}{n}).
	
	In other words, find \lmti{n} \frc{1 - r^{n - 1}}{1 - r}
	
	By Theorem 4.2.1, \lmti{n} \frc{1 - r^{n - 1}}{1 - r} \eql \frc{1 - \lmti{n} r^{n - 1}}{1 - r}
	
	\wts{\lmti{n} r$^{n - 1}$ \eql 0}
	
	Correct me if I'm wrong, but 
	
	\lmti{n} \av{r}$^{n - 1}$ \eql 0 \lra \lmti{n} r$^{n - 1}$ \eql 0
	
	\wts{\lmti{n} \av{r}$^{n - 1}$ \eql 0}
	
	We know that 0 \lse \av{r} \ls 1,
	
	So,
	
	for 1 \ls N \mem \bn, \av{r}$^N$ \ls \av{r} (assuming it is not the trivial case that r \eql 0)
	
	\lt{\ep \gr 0}
	
	\av{\av{r}$^{n-1}$ - 0} \ls \ep
	
	\av{r}$^{n-1}$ - 0 \ls \ep (since that's always positive)
	
	\av{r}$^{n-1}$ \ls \ep
	
	(n - 1) ln \av{r} \ls ln \ep
	
	n ln \av{r} \ls \ep \ps ln \av{r	}
	
	n \gr \frc{\ep + \textrm{ ln \av{r}}}{\textrm{ln \av{r}}}
	
	So, if 
	
	n \gr \frc{\ep + \textrm{ ln \av{r}}}{\textrm{ln \av{r}}}
	
	Then, 
	
	\fa \ep \gr 0, \exs N \mem \bn st n \gre N implies \av{\av{r}$^{n-1}$ \ms 0} \ls \ep
	
	Hence, 
	
	\lmti{n} r$^{n - 1}$ \eql 0
	
	Hence,
	
	\frc{1 - \lmti{n} r^{n - 1}}{1 - r} \eql \frc{1 - 0}{1 - r} \eql \frc{1}{1 - r}
	
\item If we let the infinite repeating decimal 0.9999... stand for the limit:
\eqn{\lmti{n}(\frac{9}{10} + \frac{9}{10^2} + ... + \frac{9}{10^n}),}
	Show that 0.99999.... \eql 1.

	From 10\bpth{a},
	
\eqn{\lmti{n} 1 + r + r^2 + ... r^n \eql \frc{1}{1 - r}}
So,

\eqn{
	\splt{
		\frac{9}{10} + \frac{9}{10^2} + ... + \frac{9}{10^n} & = 9(\frac{1}{10} + \frac{1}{10^2} + ... + \frac{1}{10^n}) \\
		& =  9((\frc{1}{10})^{1} + (\frc{1}{10})^{2} + ... + (\frc{1}{10})^{n})
	}
}

If we let r \eql \frc{1}{10}, then

\eqn{\lmti{n} 1 + (\frc{1}{10})^1 + (\frc{1}{10})^2 + ... (\frc{1}{10})^n = \frc{1}{1 - \frc{1}{10}}}

So,

\eqn{
	\splt{
		\lmti{n} 9((\frc{1}{10})^{1} + (\frc{1}{10})^{2} + ... + (\frc{1}{10})^{n}) & = 9 \lmti{n}((\frc{1}{10})^{1} + (\frc{1}{10})^{2} + ... + (\frc{1}{10})^{n}) \\
		& = 9 (\frac{1}{1 - \frac{1}{10}} - 1) \\
		& = 10 - 9 \\
		& = 1
	}
}

Hence,

0.9999999... \eql 1
	
\elist
}

\bsc{Problem 13}{
Prove Lemma 4.3.11:

Every Cauchy sequence is bounded. (Similar to the proof of Theorem 4.1.13) \

\

\bgpf
\lt{\uw{s}{n} be a Cauchy sequence}

\uw{s}{n} is Cauchy if,

\fa \ep \gr 0, \exs N \mem \bn st for n, m \gre N, \av{\uw{s}{n} \ms \uw{s}{m}} \ls \ep

With \ep \eql 1, we obtain N \mem \bn st

\av{\uw{s}{n} \ms \uw{s}{m}} \ls 1 when n, m \gre N

Thus, n \gre N implies \av{\uw{s}{n}} \ls \av{\uw{s}{m}} \ps 1

If we let

\eqn{M = max \bk{\av{s_1}, \av{s_2}, ... \av{s_N}, \av{\uw{s}{m}} + 1}}

Then we have \av{\uw{s}{n}} \lse M \fa n \mem \bn 

Thus, \prn{\uw{s}{n}} is bounded.

\epf
}

\newpage

\bsc{Problem 14}{
Let \prn{\uw{s}{n}} be the sequence defined by \uw{s}{n} \eql (1 \ps \frc{1}{n})$^n$.

Use the binomial theorem (Exercise 3.1.30) to show that \prn{\uw{s}{n}} is an increasing sequence with \uw{s}{n} \ls 3 \fa n.

Conclude that \prn{\uw{s}{n}} is convergent. The limit of (\uw{s}{n}) is referred to as e and is used as the base for natural logarithms. The approximate value of e is 2.71828. \

\

\lt{\uw{s}{n} \eql (1 \ps \frc{1}{n})$^n$}

\wts{\prn{\uw{s}{n}} is increasing, using the binomial theorem (Exercise 3.1.30)}

\uf{\prn{1 \ps b}}{n} \eql 1 \ps nb \ps \frc{n(n - 1)}{2!}\uf{b}{n} \ps ... \ps \frc{n(n - 1)... (n - (r - 1))}{r!}\uf{b}{r} \ps ... \ps \uf{b}{n}

So,

\uf{\prn{1 \ps (\frc{1}{n})}}{n} \eql 1 \ps n(\frc{1}{n}) \ps \frc{n(n - 1)}{2!}\uf{(\frc{1}{n})}{n} \ps ... \ps \frc{n(n - 1)... (n - (r - 1))}{r!}\uf{(\frc{1}{n})}{r} \ps ... \ps \uf{(\frc{1}{n})}{n}

In other words,

\eqn{\uf{\prn{1 \ps \frc{1}{n}}}{n} \eql \sum_{r = 0}^n \nck{n}{r}(\frac{1}{n})^r}
\eqn{\uf{\prn{1 \ps \frc{1}{n}}}{n} \eql 1 \ps \sum_{r = 1}^n \nck{n}{r}(\frac{1}{n})^r}
\eqn{\uf{\prn{1 \ps \frc{1}{n}}}{n} \eql 1 \ps \sum_{r = 1}^n \frac{n!}{r!(n-r)!} (\frac{1}{n})^r}
\eqn{\uf{\prn{1 \ps \frc{1}{n}}}{n} \eql 1 \ps \sum_{r = 1}^n \frac{n(n - 1)(n - 2)...(2)(1)}{(r)(r - 1)...(2)(1)(n-r)(n-r-1)...(2)(1)} (\frac{1}{n})^r}
\eqn{\uf{\prn{1 \ps \frc{1}{n}}}{n} \eql 1 \ps \sum_{r = 1}^n \frac{n(n - 1)(n - 2)...(2)(1)}{(r)(r - 1)...(2)(1)(n-r)(n-r-1)...(2)(1)} \frac{1}{n^r}}
\eqn{\uf{\prn{1 \ps \frc{1}{n}}}{n} \eql 1 \ps \sum_{r = 1}^n \frac{n(n - 1)(n - 2)...(n - (r - 1))}{(r)(r - 1)...(2)(1)} \frac{1}{n^r}}
\eqn{\uf{\prn{1 \ps \frc{1}{n}}}{n} \eql 1 \ps \sum_{r = 1}^n \frac{n(n - 1)...(n - (r - 1))}{r!} \frac{1}{n^r}}
\eqn{\uf{\prn{1 \ps \frc{1}{n}}}{n} \eql 1 \ps \sum_{r = 1}^n \frac{n}{n}\frac{n - 1}{n}...\frac{n - (r - 1)}{n}\frac{1}{r!}}
\eqn{\uf{\prn{1 \ps \frc{1}{n}}}{n} \eql 1 \ps \sum_{r = 1}^n 1 * ( 1 - \frac{1}{n}) * ( 1 - \frac{2}{n})...(1 - \frac{r - 1}{n})\frac{1}{r!}}
\eqn{\uf{\prn{1 \ps \frc{1}{n + 1}}}{n + 1} \eql 1 \ps \sum_{r = 1}^{n + 1} 1 * ( 1 - \frac{1}{n + 1}) * ( 1 - \frac{2}{n + 1})...(1 - \frac{r - 1}{n + 1})\frac{1}{r!}}

Let i \mem \bk{0, 1, 2, ... r \ms 1}

We know that

\eqn{n + 1 \gre n}
\eqn{\frac{1}{n + 1} \lse \frac{1}{n}}
\eqn{\frac{i}{n + 1} \lse \frac{i}{n}}
\eqn{\fa i}

So,

\eqn{\uf{\prn{1 \ps \frc{1}{n}}}{n} \eql 1 \ps \sum_{r = 1}^n \frac{1}{r!}\prod_i(1 - \frac{i}{n})}
\eqn{\uf{\prn{1 \ps \frc{1}{n + 1}}}{n + 1} \eql 1 \ps \sum_{r = 1}^{n} \frac{1}{r!}\prod_i(1 - \frac{i}{n + 1}) + \frac{1}{r!}\prod_i(1 - \frac{i}{n + 1})}

Since
\eqn{\frac{i}{n + 1} \lse \frac{i}{n}}
\fa i,

\eqn{\sum_{r = 1}^n \frac{1}{r!}\prod_i(1 - \frac{i}{n}) \ls \sum_{r = 1}^{n} \frac{1}{r!}\prod_i(1 - \frac{i}{n + 1}) + \frac{1}{r!}\prod_i(1 - \frac{i}{n + 1})}
In other words:
\eqn{\uf{\prn{1 \ps \frc{1}{n}}}{n} \ls \uf{\prn{1 \ps \frc{1}{n + 1}}}{n + 1}}
\fa n \mem \bn 

Hence, \uw{s}{n} is increasing.



P(n) (Proposition as a function of n):

\uw{s}{n} \lse \uw{s}{n + 1}, \fa n \mem \bn 

\uw{s}{1} \eql 2

\uw{s}{2} \eql 2.25

\as{\uw{s}{k} \lse \uw{s}{k + 1} \fa k \mem \bn}

\eqn{1 \ps \sum_{r = 1}^k \frac{k(k - 1)...(k - (r - 1))}{r!} \frac{1}{k^r} \lse 1 \ps \sum_{r = 1}^{k + 1} \frac{(k + 1)(k)(k - 1)...(k - (r - 1))}{r!} \frac{1}{(k + 1)^r}}
\eqn{\sum_{r = 1}^k \frac{k(k - 1)...(k - (r - 1))}{r!k^r} \lse \sum_{r = 1}^{k + 1} \frac{(k + 1)(k)(k - 1)...(k - (r - 1))}{r!(k + 1)^r}}
\eqn{\sum_{r = 1}^k \frac{k(k - 1)...(k - (r - 1))}{r!k^r} \lse \sum_{r = 1}^{k} \frac{(k + 1)(k)(k - 1)...(k - (r - 1))}{r!(k + 1)^r} + \frac{(k)(k - 1)...(k - ((k + 1) - 1))}{(k + 1)!(k + 1)^{k}}}


asdfasdf

\eqn{\sum_{r = 1}^k \frac{1}{r!k^r} \lse \sum_{r = 1}^{k + 1} \frac{k + 1}{r!(k + 1)^r}}
\eqn{\sum_{r = 1}^k \frac{1}{r!k^r} \lse \sum_{r = 1}^{k + 1} \frac{1}{r!(k + 1)^{r - 1}}}
\eqn{\sum_{r = 1}^{k + 1} \frac{1}{r!k^r} - \frac{1}{(k + 1)!(k + 1)^{k + 1}} \lse \sum_{r = 1}^{k + 1} \frac{1}{r!(k + 1)^{r - 1}}}
\eqn{\sum_{r = 1}^{k + 1} \frac{1}{r!k^r} -  \sum_{r = 1}^{k + 1} \frac{1}{r!(k + 1)^{r - 1}} \lse \frac{1}{(k + 1)!(k + 1)^{k + 1}}}
\eqn{\sum_{r = 1}^{k + 1} \frac{r!(k + 1)^{r - 1}}{(r!(k + 1)^{r - 1})r!k^r} - \frac{r!k^r}{(r!(k + 1)^{r - 1})r!k^r} \lse \frac{1}{(k + 1)!(k + 1)^{k + 1}}}
\eqn{\sum_{r = 1}^{k + 1} \frac{r!(k + 1)^{r - 1} - r!k^r}{(r!(k + 1)^{r - 1})r!k^r} \lse \frac{1}{(k + 1)!(k + 1)^{k + 1}}}
\eqn{\sum_{r = 1}^{k + 1} \frac{(k + 1)^{r - 1} - k^r}{(k + 1)^{r - 1}r!k^r} \lse \frac{1}{(k + 1)!(k + 1)^{k + 1}}}
\eqn{\sum_{r = 1}^{k + 1} \frac{(k + 1)^{r - 1}}{(k + 1)^{r - 1}r!k^r} - \frac{k^r}{{(k + 1)^{r - 1}r!k^r}} \lse \frac{1}{(k + 1)!(k + 1)^{k + 1}}}
\eqn{\sum_{r = 1}^{k + 1} \frac{1}{r!k^r} - \frac{1}{{(k + 1)^{r - 1}r!}} \lse \frac{1}{(k + 1)!(k + 1)^{k + 1}}}

asdfasd

\eqn{\sum_{r = 1}^k \frac{1}{r!k^r} \gre \sum_{r = 1}^{k} \frac{1}{r!(k + 1)^{r - 1}} + \frac{1}{(k + 1)!(k + 1)^{(k + 1) - 1}}}
\eqn{\sum_{r = 1}^k \frac{1}{r!k^r} \gre \sum_{r = 1}^{k} \frac{1}{r!(k + 1)^{r - 1}} + \frac{1}{(k + 1)!(k + 1)^{k}}}
\eqn{\frac{1}{k} + \frac{1}{2!k^2} ... + \frac{1}{k!k^k} \gre \frac{1}{(k + 1)^{0}} + \frac{1}{2!(k + 1)^{1}} + ... + \frac{1}{k!(k + 1)^{k - 1}} + \frac{1}{(k + 1)!(k + 1)^{k}}}
\eqn{\frac{1}{k} + \frac{1}{2!k^2} ... + \frac{1}{k!k^k} \gre \frac{1}{(k + 1)^{0}} + \frac{1}{2!(k + 1)^{1}} + ... + \frac{1}{k!(k + 1)^{k - 1}} + \frac{1}{(k + 1)!(k + 1)^{k}}}


Now,

\eqn{s_{k + 1} = \sqrt{4\uw{s}{k + 1} + 1} \gre \sqrt{4\uw{s}{k + 2} + 1} = s_{k + 2}}

So,

\eqn{s_k \gre s_{k + 1}}

Hence, by induction, P(n): \uw{s}{n} \gre \uw{s}{n + 1} is true \fa n \mem \bn 



\wts{\prn{\uw{s}{n}} is bounded with \uw{s}{n} \ls 3 \fa n, using the binomial theorem (Exercise 3.1.30)}

\wts{\prn{\uw{s}{n}} is convergent}

Look at it as 1 \ps $\sum_{r = 1}^n$ \frc{n(n - 1)...(n - (r - 1))}{r!}\frc{1}{n^r}

\prn{1 \ps \frc{1}{n}}$^n$ goes in there somewhere somehow.

About the last homework (HW 6), problem 9:

If \uw{s}{n} \lse \uw{t}{n} \fa n \mem \bn and \lmti{n} \uw{s}{n} \eql $\infty$,

then \lmti{n} \uw{t}{n} \eql $\infty$

So, \fa M \mem \br, \exs N \mem \bn st

\uw{s}{n} \gr M, \fa n \gre N

Notice that:

\uw{t}{n} \gre \uw{s}{n} \gr M, \fa n \gre N

So by definition, \lmti{n} \uw{t}{n} \eql $\infty$
}
\end{document}