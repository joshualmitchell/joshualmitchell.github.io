% Thank you Josh Davis for this template!
% https://github.com/jdavis/latex-homework-template/blob/master/homework.tex

\documentclass{article}

\newcommand{\hmwkTitle}{Lec\ \#21}

% % Packages

\usepackage{fancyhdr}
\usepackage{extramarks}
\usepackage{amsmath}
\usepackage{amssymb}
\usepackage{amsthm}
\usepackage{amsfonts}
\usepackage{tikz}
\usepackage[plain]{algorithm}
\usepackage{algpseudocode}
\usepackage{enumitem}
\usepackage{chngcntr}

% Libraries

\usetikzlibrary{automata, positioning, arrows}

%
% Basic Document Settings
%

\topmargin=-0.45in
\evensidemargin=0in
\oddsidemargin=0in
\textwidth=6.5in
\textheight=9.0in
\headsep=0.25in

\linespread{1.1}

\pagestyle{fancy}
\lhead{\hmwkAuthorName}
\chead{}
\rhead{\hmwkClass\ (\hmwkClassInstructor): \hmwkTitle}
\lfoot{\lastxmark}
\cfoot{\thepage}

\renewcommand\headrulewidth{0.4pt}
\renewcommand\footrulewidth{0.4pt}

\setlength\parindent{0pt}
\setcounter{secnumdepth}{0}

\newcommand{\hmwkClass}{MATH 3380 / Analysis 1}        % Class
\newcommand{\hmwkClassInstructor}{Dr. Welsh}           % Instructor
\newcommand{\hmwkAuthorName}{\textbf{Joshua Mitchell}} % Author

%
% Title Page
%

\title{
    \vspace{2in}
    \textmd{\textbf{\hmwkClass:\ \hmwkTitle}}\\
    \normalsize\vspace{0.1in}\small\vspace{0.1in}\large{\textit{\hmwkClassInstructor}}
    \vspace{3in}
}

\author{\hmwkAuthorName}
\date{}

\renewcommand{\part}[1]{\textbf{\large Part \Alph{partCounter}}\stepcounter{partCounter}\\}

% Integral dx
\newcommand{\dx}{\mathrm{d}x}

%
% Various Helper Commands
%

% For derivatives
\newcommand{\deriv}[1]{\frac{\mathrm{d}}{\mathrm{d}x} (#1)}

% For partial derivatives
\newcommand{\pderiv}[2]{\frac{\partial}{\partial #1} (#2)}


% Alias for the Solution section header
\newcommand{\solution}{\textbf{\large Solution}}

% Probability commands: Expectation, Variance, Covariance, Bias
\newcommand{\E}{\mathrm{E}}
\newcommand{\Var}{\mathrm{Var}}
\newcommand{\Cov}{\mathrm{Cov}}
\newcommand{\Bias}{\mathrm{Bias}}

% Formatting commands:

\newcommand{\mt}[1]{\ensuremath{#1}}
\newcommand{\nm}[1]{\textrm{#1}}

\newcommand\bsc[2][\DefaultOpt]{%
  \def\DefaultOpt{#2}%
  \section[#1]{#2}%
}
\newcommand\ssc[2][\DefaultOpt]{%
  \def\DefaultOpt{#2}%
  \subsection[#1]{#2}%
}
\newcommand{\bgpf}{\begin{proof} $ $\newline}

\newcommand{\bgeq}{\begin{equation*}}
\newcommand{\eeq}{\end{equation*}}	

\newcommand{\balist}{\begin{enumerate}[label=\alph*.]}
\newcommand{\elist}{\end{enumerate}}

\newcommand{\bilist}{\begin{enumerate}[label=\roman*)]}	

\newcommand{\bgsp}{\begin{split}}
% \newcommand{\esp}{\end{split}} % doesn't work for some reason.

\newcommand\prs[1]{~~~\textbf{(#1)}}

\newcommand{\lt}[1]{\textbf{Let: } #1}
     							   %  if you're setting it to be true
\newcommand{\supp}[1]{\textbf{Suppose: } #1}
     							   %  Suppose (if it'll end up false)
\newcommand{\wts}[1]{\textbf{Want to show: } #1}
     							   %  Want to show
\newcommand{\as}[1]{\textbf{Assume: } #1}
     							   %  if you think it follows from truth
\newcommand{\bpth}[1]{\textbf{(#1)}}

\newcommand{\step}[2]{\begin{equation}\tag{#2}#1\end{equation}}
\newcommand{\epf}{\end{proof}}

\newcommand{\sidenote}[1]{-----------------------------------------------------------------Side Notes---------------------------------------------------------------
#1 \

---------------------------------------------------------------------------------------------------------------------------------------------}

% Analysis / Logical commands:

\newcommand{\br}{\mathbb{R}}       % |R
\newcommand{\bq}{\mathbb{Q}}       % |Q
\newcommand{\bn}{\mathbb{N}}       % |N
\newcommand{\bc}{\mathbb{C}}       % |C
\newcommand{\bz}{\mathbb{Z}}       % |Z

\newcommand{\ep}{\epsilon}         % epsilon
\newcommand{\fa}{\forall}          % for all

\newcommand{\es}{\emptyset}        % empty set
\newcommand{\sbs}{\subset}         % subset of

\newcommand{\lra}{\longrightarrow} % implies ----->
\newcommand{\rar}{\Rightarrow}     % implies -->

\newcommand{\lla}{\longleftarrow}  % implies <-----
\newcommand{\lar}{\Leftarrow}      % implies <--

\newcommand{\pr}{^\prime} 		   % prime (i.e. R')

\newcommand{\bnm}[2]{\mt{#1\setminus{#2}}}
\newcommand{\bnt}[2]{\mt{\textrm{#1}\setminus{\textrm{#2}}}}
\newcommand{\bi}{\bnm{\mathbb{R}}{\mathbb{Q}}}

\newcommand{\nbho}[3]{\textrm{N(}#1, #2\textrm{) }\cap \textrm{ #3} \neq \emptyset}
     							   %  N(x, eps) intersect S \= emptyset
\newcommand{\nbhe}[3]{\textrm{N(}#1, #2\textrm{) }\cap \textrm{ #3} = \emptyset}
     							   %  N(x, eps) intersect S  = emptyset
\newcommand{\dnbho}[3]{\textrm{N*(}#1, #2\textrm{) }\cap \textrm{ #3} \neq \emptyset}
     							   %  N*(x, eps) intersect S \= emptyset
\newcommand{\dnbhe}[3]{\textrm{N*(}#1, #2\textrm{) }\cap \textrm{ #3} = \emptyset}
     							   %  N*(x, eps) intersect S = emptyset
     							 

% ----------

% Packages

\usepackage{fancyhdr}
\usepackage{extramarks}
\usepackage{amsmath}
\usepackage{amssymb}
\usepackage{amsthm}
\usepackage{amsfonts}
\usepackage{tikz}
\usepackage[plain]{algorithm}
\usepackage{algpseudocode}
\usepackage{enumitem}
\usepackage{chngcntr}

% Libraries

\graphicspath{{/Users/jm/iclouddrive/3380pics/}}

\usetikzlibrary{automata, positioning, arrows}

%
% Basic Document Settings
%

\topmargin=-0.45in
\evensidemargin=0in
\oddsidemargin=0in
\textwidth=6.5in
\textheight=9.0in
\headsep=0.25in

\linespread{1.1}

\pagestyle{fancy}
\lhead{\hmwkAuthorName}
\chead{}
\rhead{\hmwkClass\ (\hmwkClassInstructor): \hmwkTitle}
\lfoot{\lastxmark}
\cfoot{\thepage}

\renewcommand\headrulewidth{0.4pt}
\renewcommand\footrulewidth{0.4pt}

\setlength\parindent{0pt}
\setcounter{secnumdepth}{0}

\newcommand{\hmwkClass}{MATH 3380 / Analysis 1}        % Class
\newcommand{\hmwkClassInstructor}{Dr. Welsh}           % Instructor
\newcommand{\hmwkAuthorName}{\textbf{Joshua Mitchell}} % Author

%
% Title Page
%

\title{
    \vspace{2in}
    \textmd{\textbf{\hmwkClass:\ \hmwkTitle}}\\
    \normalsize\vspace{0.1in}\small\vspace{0.1in}\large{\textit{\hmwkClassInstructor}}
    \vspace{3in}
}

\author{\hmwkAuthorName}
\date{}

\renewcommand{\part}[1]{\textbf{\large Part \Alph{partCounter}}\stepcounter{partCounter}\\}

% Integral dx
\newcommand{\dx}{\mathrm{d}x}

%
% Various Helper Commands
%

% For derivatives
\newcommand{\deriv}[1]{\frac{\mathrm{d}}{\mathrm{d}x} (#1)}

% For partial derivatives
\newcommand{\pderiv}[2]{\frac{\partial}{\partial #1} (#2)}


% Alias for the Solution section header
\newcommand{\solution}{\textbf{\large Solution}}

% Probability commands: Expectation, Variance, Covariance, Bias
\newcommand{\E}{\mathrm{E}}
\newcommand{\Var}{\mathrm{Var}}
\newcommand{\Cov}{\mathrm{Cov}}
\newcommand{\Bias}{\mathrm{Bias}}

% Formatting commands:

\newcommand{\mt}[1]{\ensuremath{#1}}
\newcommand{\nm}[1]{\textrm{#1}}

\newcommand\bsc[2][\DefaultOpt]{%
  \def\DefaultOpt{#2}%
  \section[#1]{#2}%
}
\newcommand\ssc[2][\DefaultOpt]{%
  \def\DefaultOpt{#2}%
  \subsection[#1]{#2}%
}
\newcommand{\bgpf}{\begin{proof} $ $\newline}

\newcommand{\bgeq}{\begin{equation*}}
\newcommand{\eeq}{\end{equation*}}	

\newcommand{\balist}{\begin{enumerate}[label=\alph*.]}
\newcommand{\elist}{\end{enumerate}}

\newcommand{\bilist}{\begin{enumerate}[label=\roman*)]}	

\newcommand{\bgsp}{\begin{split}}
% \newcommand{\esp}{\end{split}} % doesn't work for some reason.

\newcommand\prs[1]{~~~\textbf{(#1)}}

\newcommand{\lt}[1]{\textbf{Let: } #1}
     							   %  if you're setting it to be true
\newcommand{\supp}[1]{\textbf{Suppose: } #1}
     							   %  Suppose (if it'll end up false)
\newcommand{\wts}[1]{\textbf{Want to show: } #1}
     							   %  Want to show
\newcommand{\as}[1]{\textbf{Assume: } #1}
     							   %  if you think it follows from truth
\newcommand{\bpth}[1]{\textbf{(#1)}}

\newcommand{\step}[2]{\begin{equation}\tag{#2}#1\end{equation}}
\newcommand{\epf}{\end{proof}}

\newcommand{\dbs}[3]{\mt{#1_{#2_#3}}}

\newcommand{\sidenote}[1]{-----------------------------------------------------------------Side Note----------------------------------------------------------------
#1 \

---------------------------------------------------------------------------------------------------------------------------------------------}

% Analysis / Logical commands:

\newcommand{\br}{\mt{\mathbb{R}} }       % |R
\newcommand{\bq}{\mt{\mathbb{Q}} }       % |Q
\newcommand{\bn}{\mt{\mathbb{N}} }       % |N
\newcommand{\bc}{\mt{\mathbb{C}} }       % |C
\newcommand{\bz}{\mt{\mathbb{Z}} }       % |Z

\newcommand{\ep}{\mt{\epsilon} }         % epsilon
\newcommand{\fa}{\mt{\forall} }          % for all
\newcommand{\afa}{\mt{\alpha} }
\newcommand{\bta}{\mt{\beta} }
\newcommand{\dta}{\mt{\delta} }
\newcommand{\mem}{\mt{\in} }
\newcommand{\exs}{\mt{\exists} }

\newcommand{\es}{\mt{\emptyset} }        % empty set
\newcommand{\sbs}{\mt{\subset} }         % subset of
\newcommand{\fs}[2]{\{\uw{#1}{1}, \uw{#1}{2}, ... \uw{#1}{#2}\}}

\newcommand{\lra}{ \mt{\longrightarrow} } % implies ----->
\newcommand{\rar}{ \mt{\Rightarrow} }     % implies -->

\newcommand{\lla}{ \mt{\longleftarrow} }  % implies <-----
\newcommand{\lar}{ \mt{\Leftarrow} }      % implies <--

\newcommand{\av}[1]{\mt{|}#1\mt{|}}  % absolute value

\newcommand{\prn}[1]{(#1)}
\newcommand{\bk}[1]{\{#1\}}

\newcommand{\ps}{\mt{+} }
\newcommand{\ms}{\mt{-} }

\newcommand{\ls}{\mt{<} }
\newcommand{\gr}{\mt{>} }

\newcommand{\lse}{\mt{\leq} }
\newcommand{\gre}{\mt{\geq} }

\newcommand{\eql}{\mt{=} }

\newcommand{\pr}{\mt{^\prime} } 		   % prime (i.e. R')
\newcommand{\uw}[2]{#1\mt{_{#2}}}
\newcommand{\uf}[2]{#1\mt{^{#2}}}
\newcommand{\frc}[2]{\mt{\frac{#1}{#2}}}
\newcommand{\lmti}[1]{\mt{\displaystyle{\lim_{#1 \to \infty}}}}
\newcommand{\limt}[2]{\mt{\displaystyle{\lim_{#1 \to #2}}}}

\newcommand{\bnm}[2]{\mt{#1\setminus{#2}}}
\newcommand{\bnt}[2]{\mt{\textrm{#1}\setminus{\textrm{#2}}}}
\newcommand{\bi}{\bnm{\mathbb{R}}{\mathbb{Q}}}

\newcommand{\urng}[2]{\mt{\bigcup_{#1}^{#2}}}
\newcommand{\nrng}[2]{\mt{\bigcap_{#1}^{#2}}}
\newcommand{\nck}[2]{\mt{{#1 \choose #2}}}

\newcommand{\nbho}[3]{\textrm{N(}#1, #2\textrm{) }\cap \textrm{ #3} \neq \emptyset}
     							   %  N(x, eps) intersect S \= emptyset
\newcommand{\nbhe}[3]{\textrm{N(}#1, #2\textrm{) }\cap \textrm{ #3} = \emptyset}
     							   %  N(x, eps) intersect S  = emptyset
\newcommand{\dnbho}[3]{\textrm{N*(}#1, #2\textrm{) }\cap \textrm{ #3} \neq \emptyset}
     							   %  N*(x, eps) intersect S \= emptyset
\newcommand{\dnbhe}[3]{\textrm{N*(}#1, #2\textrm{) }\cap \textrm{ #3} = \emptyset}
     							   %  N*(x, eps) intersect S = emptyset
     							   
\newcommand{\eqn}[1]{\[#1\]}
\newcommand{\splt}[1]{\begin{split}#1\end{split}}

\newcommand{\infy}{\mt{\infty} }
\newcommand{\unn}{\mt{\cup} }
\newcommand{\inn}{\mt{\cap} }
\newcommand\tab[1][1cm]{\hspace*{#1}}

     							 
% ----------

\begin{document}

HW 10: page 212 - 214, \#1, 2 (omit d), 3, 5 (prove the result), 10, 11 (just prove the "max" result), 13, 16 (First prove that for any H \sbs \br, \uf{f}{-1}(\bnt{R}{H}) \eql \bnt{\br}{\uf{f}{-1}(H)}, use this in conjunction with Theorem 5.2.14)

\bsc{Lec 21 Continued}{

\ssc{Theorem 5.2.2}{

\lt{f : D \lra \br and c \mem D}

Then the following are equivalent:

\balist
\item f is continuous at c
\item If \bk{\uw{x}{n}} is any sequence in D st \uw{x}{n} \lra c as n \lra \infy (\uw{x}{n} can actually be c),
	
	then \lmti{n} f(\uw{x}{n}) \eql f(c)
\item For every neighborhood V of f(c), \exs a neighborhood U of c st f(U \inn D) \sbs V
	
	Furthermore, if c \mem D\pr, then the above are all equivalent to d
\item f has a limit at c and \limt{x}{c} f(x) \eql f(c)
\elist

\bgpf

Case:
\bilist
\item c \mem \bnt{D}{D\pr} (i.e. c is an isolated point)
	
	Thus, \exs a neighborhood U \sbs \br of c st
	\eqn{U \cap D \eql \bk{c}}
	(i.e. U \eql (c \ms \dta, c \ps \dta) \eql \bk{c})
	
	\bpth{a}
	
	\wts{f is continuous at x \eql c}
	
	For \ep \gr 0, \exs \dta \gr 0 st (c \ms \dta, c \ps \dta) \sbs U.
	
	This follows since a neighborhood is open. Thus,
	\eqn{|f(x) - f(c)| = 0 < \epsilon \textrm{ whenever} |x - c| < \delta \textrm{ and } x \mem D}
	This means by definition that f(x) is continuous at x \eql c.
	
	\bpth{b}
	
	\lt{\bk{\uw{x}{n}} \sbs D st \uw{x}{n} \lra c as n \lra \infy}
	
	and
	
	For \ep \gr 0, \exs \dta \gr 0 st (c \ms \dta, c \ps \dta) \sbs U
		
	\wts{\lmti{n} f(\uw{x}{n}) \eql f(c).}
	
	Since U is open, \exs N \mem \bn st
	\eqn{|x_n - c| < \delta \textrm{ for } n \gre N}
	Thus, \uw{x}{n} \mem U for n \gre N
	
	We see that
	\eqn{|f(x_n) - f(c)| = 0 < \epsilon \textrm{ for } n \gre N}
	Hence,
	\lmti{n} f(\uw{x}{n}) \eql f(c)
	
	\bpth{c}
	
	Now,
	
	\lt{V be a neighborhood of f(c)}
	
	Then, using U as defined prior to (a):
	\eqn{f(U \cap D) \sbs V}
	Hence, a, b, and c, are equivalent if c \mem \bnt{D}{D\pr}

\item c \mem D \inn D\pr (i.e. c is an accumulation point)
	
	\bpth{a} is equivalent to \bpth{d} by Definition 5.2.1
	
	\bpth{b} is equivalent to \bpth{d} by Theorem 5.2.2
	
	\bpth{c} is equivalent to \bpth{d} by Theorem 5.1.8
	
	\textbf{N.B.}: In case \bpth{i}, we proved that a function is always continuous at an isolated point in its domain.
	
	Sometimes in calculus one, we tell a student that a function is continuous in the interval [A, B] if you can trace it on the chalkboard without having to take your hand off. This is a white lie.
	
	It turns out that as long as it's defined at all points on its domain, it's continuous (i.e. sequences are continuous).
	
	\textbf{N.B.}: In definition 5.2.1, we defined continuity of f at a \textbf{point} c in the domain D of f. If S \sbs D and f is continuous at each point of S, then f is continuous on S. If f is continuous at all points of D, then f is a continuous function on D.
\elist

\epf

}

\ssc{Example 5.2.3}{

\lt{p : \br \lra \br be a polynomial}

In Example 5.1.14, we saw that \limt{x}{c} p(x) \eql p(c).

By \bpth{d} iff \bpth{a}, from Theorem 5.2.2, we see that p is a continuous function on \br
}

\ssc{Example 5.2.5}{

Define f(x) \eql x sin(\frc{1}{x}), x $\neq$ 0, but 0 if x \eql 0

Then f: \br \lra \br

Prove that f is continuous at x \eql 0

\

We're thinking that the limit of this function at 0 is 0, so,

\lt{\ep \gr 0}

Now,
\eqn{|f(x) - f(0)| = |x\sin\frac{1}{x} - 0| = |x||\sin\frac{1}{x}| \lse |x| * 1 = |x| \ls \ep}
whenever 0 \ls \av{x} \ls \dta \eql \ep (but since we have 0 in there, it's also true whenever \av{x} \ls \dta \eql \ep) and x \mem D

Hence, by Definition 5.2.1, \limt{x}{0} f(x) \eql f(0), i.e. f is continuous at x \eql 0

}

\ssc{Theorem 5.2.6}{
\lt{f : D \lra \br and c \mem D}

Then,

f is \textbf{discontinuous} at c iff \exs a sequence \bk{\uw{x}{n}} in D st \uw{x}{n} \lra c but \lmti{n} f(\uw{x}{n}) is not f(c).

\bgpf

This is not \bpth{a} iff not \bpth{b} in Theorem 5.2.2

\epf
}

\ssc{Example 5.2.7}{

\lt{f : \br \lra \br, where f(x) \eql \frc{1}{x} if x $\neq$ 0, k if x \eql 0}

Prove that f is discontinuous at x \eql 0.

So,

for \ep \gr 0, \exs \dta \gr 0 st
\eqn{|f(x) - f(0)| = |\frac{1}{x} - k| < \ep}
whenever \av{x \ms 0} \ls \dta and x \mem D

\lt{\uw{x}{n} \eql \frc{1}{n} \fa n \mem \bn}

Then,

\lmti{n} \uw{x}{n} \eql 0 but \lmti{n} f(\uw{x}{n}) \eql \lmti{n} \frc{1}{\frc{1}{n}} \eql \lmti{n} n \eql \infy

So,

\lmti{n} f(x) $\neq$ f(0) \eql k \mem \br

Note: If we define D = ($-\infty$, 0) \unn (0, \infy) and let f : D \lra \br be defined by f(x) \eql \frc{1}{x}, 

then f is continuous on D.

If c \mem D, then \limt{x}{c} f(x) \eql \frc{1}{c} \eql f(c).

Since c \mem D\pr, it follows from Theorem 5.2.2 that f is continuous at c.

}

\ssc{5.2.8}{

The Dirichlet function is f: \br \lra \br defined by:

f(x) \eql 1 if x \mem \bq, 0 if x \mem \bnt{\br}{\bq}

Prove that f is discontinuous everywhere.

For \ep \eql \frc{1}{4}, we must find \dta \gr 0 st \av{f(x) \ms f(c)} \ls \frc{1}{4} whenever 0 \ls \av{x \ms c} \ls \dta and x \mem D

\textbf{Solution:}

\lt{c \mem \br}
Case:
\bilist
\item c \mem \bq
	
	\lt{\uw{x}{n} \eql c \ps \frc{\sqrt{2}}{n}}
	
	Then \uw{x}{n} \mem \bnt{\br}{\bq}, \fa n \mem \bn
	
	and \lmti{n} \uw{x}{n} \eql c
	
	\lmti{n} f(\uw{x}{n}) \eql 0 $\neq$ f(c) \eql 1
	
	By Theorem 5.2.6, f is not continuous at x \eql c
	
\item c \mem \bnt{\br}{\bq}
	
	\lt{\uw{x}{n} \mem \bq \fa n \mem \bn st \uw{x}{n} \lra c as n \lra \infy}
	
	Then \lmti{n} f(\uw{x}{n}) \eql 1 $\neq$ f(c) \eql 0
\elist

}

Take a look at 5.2.9, but it won't be discussed in this class.

\ssc{Theorem 5.2.10}{

\lt{f, g : D \lra \br and c \mem D}

\as{f and g are continuous at c}

\balist
\item f \ps g and fg are continuous at c
\item \frc{f}{g} is continuous at c provided that g(c) $\neq$ 0
\elist

\bgpf

\bpth{a}: Similar to b.

\bpth{b}:

\lt{\bk{\uw{x}{n}} be a sequence in D st \uw{x}{n} \lra c as n \lra \infy}

Then, 

\eqn{\lmti{n} (\frc{f}{g})(\uw{x}{n}) = \lmti{n} \frc{f(\uw{x}{n})}{g(\uw{x}{n})} = \frac{\lmti{n} f(x_n)}{\lmti{n} g(x_n)} = \frc{f(c)}{g(c)} = (\frc{f}{g})(c)}

By Theorem 5.2.2, \bpth{a} iff \bpth{b}, \frc{f}{g} is continuous at c.
\epf
}

\ssc{Example 5.2.11}{
See Exercise \#11, page 214

Prove:

(max \bk{f, g})(x) \eql max \bk{f(x), g(x)} \eql \frc{1}{2}(f(x) \ps g(x)) \ps \frc{1}{2}\av{f(x) \ms g(x)} \fa x \mem D

Use 2 cases.
}

\ssc{Theorem 5.2.12}{

\lt{f : D \lra \br, g : E \lra \br where f(D) \sbs E.}

If f is continuous at c \mem D and g is continuous at f(c) \mem E, then the composition of f and g given by g o f is continuous at x \eql c

(This is essentially saying the composition of two continuous functions is also continuous.)

\bgpf

\lt{W be a neighborhood of g(f(c)).}

Since g is continuous at f(c), \exs a neighborhood V of f(c) st 
g(V \inn E) \sbs W by Theorem 5.2.2. (a) iff (c) \bpth{1}

Since f is continuous at c, there is a neighborhood U of c st f(U \inn D) \sbs V \bpth{2}

Now, f(D) \sbs E, so f(U \inn D) \sbs E.

Thus,

\bpth{2} implies f(U \inn D) \sbs V \inn E

So g(f((U \inn D))) \sbs W (i.e. (g o f)(U \inn D) \sbs W)

By Theorem 5.2.2, g o f is continuous at x \eql c.

\epf
}

\ssc{Example 5.2.13}{

q(x) \eql x sin(\frc{1}{x}) is continuous at any c \mem \br st c $\neq$ 0

\bgpf
q(x) \eql [h(g o f)](x), where f(x) \eql \frc{1}{x}, \tab g(x) \eql sin x, \tab h(x) \eql x

Since f, g, and h are all continuous, so is q(x).
\epf
}

\ssc{Theorem 5.2.14 (Links Topology with Analysis)}{

A function f : D \lra \br is continuous on D iff 

for every open set G \sbs \br, \exs an open set H \sbs \br st \uf{f}{-1}(G) \eql H \inn D

(This is a way of talking about continuity without using distance (i.e. using open sets instead))
}
}
\end{document}