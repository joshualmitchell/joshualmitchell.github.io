% Thank you Josh Davis for this template!
% https://github.com/jdavis/latex-homework-template/blob/master/homework.tex

\documentclass{article}

\newcommand{\hmwkTitle}{Lec\ \#16}

% % Packages

\usepackage{fancyhdr}
\usepackage{extramarks}
\usepackage{amsmath}
\usepackage{amssymb}
\usepackage{amsthm}
\usepackage{amsfonts}
\usepackage{tikz}
\usepackage[plain]{algorithm}
\usepackage{algpseudocode}
\usepackage{enumitem}
\usepackage{chngcntr}

% Libraries

\usetikzlibrary{automata, positioning, arrows}

%
% Basic Document Settings
%

\topmargin=-0.45in
\evensidemargin=0in
\oddsidemargin=0in
\textwidth=6.5in
\textheight=9.0in
\headsep=0.25in

\linespread{1.1}

\pagestyle{fancy}
\lhead{\hmwkAuthorName}
\chead{}
\rhead{\hmwkClass\ (\hmwkClassInstructor): \hmwkTitle}
\lfoot{\lastxmark}
\cfoot{\thepage}

\renewcommand\headrulewidth{0.4pt}
\renewcommand\footrulewidth{0.4pt}

\setlength\parindent{0pt}
\setcounter{secnumdepth}{0}

\newcommand{\hmwkClass}{MATH 3380 / Analysis 1}        % Class
\newcommand{\hmwkClassInstructor}{Dr. Welsh}           % Instructor
\newcommand{\hmwkAuthorName}{\textbf{Joshua Mitchell}} % Author

%
% Title Page
%

\title{
    \vspace{2in}
    \textmd{\textbf{\hmwkClass:\ \hmwkTitle}}\\
    \normalsize\vspace{0.1in}\small\vspace{0.1in}\large{\textit{\hmwkClassInstructor}}
    \vspace{3in}
}

\author{\hmwkAuthorName}
\date{}

\renewcommand{\part}[1]{\textbf{\large Part \Alph{partCounter}}\stepcounter{partCounter}\\}

% Integral dx
\newcommand{\dx}{\mathrm{d}x}

%
% Various Helper Commands
%

% For derivatives
\newcommand{\deriv}[1]{\frac{\mathrm{d}}{\mathrm{d}x} (#1)}

% For partial derivatives
\newcommand{\pderiv}[2]{\frac{\partial}{\partial #1} (#2)}


% Alias for the Solution section header
\newcommand{\solution}{\textbf{\large Solution}}

% Probability commands: Expectation, Variance, Covariance, Bias
\newcommand{\E}{\mathrm{E}}
\newcommand{\Var}{\mathrm{Var}}
\newcommand{\Cov}{\mathrm{Cov}}
\newcommand{\Bias}{\mathrm{Bias}}

% Formatting commands:

\newcommand{\mt}[1]{\ensuremath{#1}}
\newcommand{\nm}[1]{\textrm{#1}}

\newcommand\bsc[2][\DefaultOpt]{%
  \def\DefaultOpt{#2}%
  \section[#1]{#2}%
}
\newcommand\ssc[2][\DefaultOpt]{%
  \def\DefaultOpt{#2}%
  \subsection[#1]{#2}%
}
\newcommand{\bgpf}{\begin{proof} $ $\newline}

\newcommand{\bgeq}{\begin{equation*}}
\newcommand{\eeq}{\end{equation*}}	

\newcommand{\balist}{\begin{enumerate}[label=\alph*.]}
\newcommand{\elist}{\end{enumerate}}

\newcommand{\bilist}{\begin{enumerate}[label=\roman*)]}	

\newcommand{\bgsp}{\begin{split}}
% \newcommand{\esp}{\end{split}} % doesn't work for some reason.

\newcommand\prs[1]{~~~\textbf{(#1)}}

\newcommand{\lt}[1]{\textbf{Let: } #1}
     							   %  if you're setting it to be true
\newcommand{\supp}[1]{\textbf{Suppose: } #1}
     							   %  Suppose (if it'll end up false)
\newcommand{\wts}[1]{\textbf{Want to show: } #1}
     							   %  Want to show
\newcommand{\as}[1]{\textbf{Assume: } #1}
     							   %  if you think it follows from truth
\newcommand{\bpth}[1]{\textbf{(#1)}}

\newcommand{\step}[2]{\begin{equation}\tag{#2}#1\end{equation}}
\newcommand{\epf}{\end{proof}}

\newcommand{\sidenote}[1]{-----------------------------------------------------------------Side Notes---------------------------------------------------------------
#1 \

---------------------------------------------------------------------------------------------------------------------------------------------}

% Analysis / Logical commands:

\newcommand{\br}{\mathbb{R}}       % |R
\newcommand{\bq}{\mathbb{Q}}       % |Q
\newcommand{\bn}{\mathbb{N}}       % |N
\newcommand{\bc}{\mathbb{C}}       % |C
\newcommand{\bz}{\mathbb{Z}}       % |Z

\newcommand{\ep}{\epsilon}         % epsilon
\newcommand{\fa}{\forall}          % for all

\newcommand{\es}{\emptyset}        % empty set
\newcommand{\sbs}{\subset}         % subset of

\newcommand{\lra}{\longrightarrow} % implies ----->
\newcommand{\rar}{\Rightarrow}     % implies -->

\newcommand{\lla}{\longleftarrow}  % implies <-----
\newcommand{\lar}{\Leftarrow}      % implies <--

\newcommand{\pr}{^\prime} 		   % prime (i.e. R')

\newcommand{\bnm}[2]{\mt{#1\setminus{#2}}}
\newcommand{\bnt}[2]{\mt{\textrm{#1}\setminus{\textrm{#2}}}}
\newcommand{\bi}{\bnm{\mathbb{R}}{\mathbb{Q}}}

\newcommand{\nbho}[3]{\textrm{N(}#1, #2\textrm{) }\cap \textrm{ #3} \neq \emptyset}
     							   %  N(x, eps) intersect S \= emptyset
\newcommand{\nbhe}[3]{\textrm{N(}#1, #2\textrm{) }\cap \textrm{ #3} = \emptyset}
     							   %  N(x, eps) intersect S  = emptyset
\newcommand{\dnbho}[3]{\textrm{N*(}#1, #2\textrm{) }\cap \textrm{ #3} \neq \emptyset}
     							   %  N*(x, eps) intersect S \= emptyset
\newcommand{\dnbhe}[3]{\textrm{N*(}#1, #2\textrm{) }\cap \textrm{ #3} = \emptyset}
     							   %  N*(x, eps) intersect S = emptyset
     							 

% ----------

% Packages

\usepackage{fancyhdr}
\usepackage{extramarks}
\usepackage{amsmath}
\usepackage{amssymb}
\usepackage{amsthm}
\usepackage{amsfonts}
\usepackage{tikz}
\usepackage[plain]{algorithm}
\usepackage{algpseudocode}
\usepackage{enumitem}
\usepackage{chngcntr}

% Libraries

\graphicspath{{/Users/jm/iclouddrive/3380pics/}}

\usetikzlibrary{automata, positioning, arrows}

%
% Basic Document Settings
%

\topmargin=-0.45in
\evensidemargin=0in
\oddsidemargin=0in
\textwidth=6.5in
\textheight=9.0in
\headsep=0.25in

\linespread{1.1}

\pagestyle{fancy}
\lhead{\hmwkAuthorName}
\chead{}
\rhead{\hmwkClass\ (\hmwkClassInstructor): \hmwkTitle}
\lfoot{\lastxmark}
\cfoot{\thepage}

\renewcommand\headrulewidth{0.4pt}
\renewcommand\footrulewidth{0.4pt}

\setlength\parindent{0pt}
\setcounter{secnumdepth}{0}

\newcommand{\hmwkClass}{MATH 3380 / Analysis 1}        % Class
\newcommand{\hmwkClassInstructor}{Dr. Welsh}           % Instructor
\newcommand{\hmwkAuthorName}{\textbf{Joshua Mitchell}} % Author

%
% Title Page
%

\title{
    \vspace{2in}
    \textmd{\textbf{\hmwkClass:\ \hmwkTitle}}\\
    \normalsize\vspace{0.1in}\small\vspace{0.1in}\large{\textit{\hmwkClassInstructor}}
    \vspace{3in}
}

\author{\hmwkAuthorName}
\date{}

\renewcommand{\part}[1]{\textbf{\large Part \Alph{partCounter}}\stepcounter{partCounter}\\}

% Integral dx
\newcommand{\dx}{\mathrm{d}x}

%
% Various Helper Commands
%

% For derivatives
\newcommand{\deriv}[1]{\frac{\mathrm{d}}{\mathrm{d}x} (#1)}

% For partial derivatives
\newcommand{\pderiv}[2]{\frac{\partial}{\partial #1} (#2)}


% Alias for the Solution section header
\newcommand{\solution}{\textbf{\large Solution}}

% Probability commands: Expectation, Variance, Covariance, Bias
\newcommand{\E}{\mathrm{E}}
\newcommand{\Var}{\mathrm{Var}}
\newcommand{\Cov}{\mathrm{Cov}}
\newcommand{\Bias}{\mathrm{Bias}}

% Formatting commands:

\newcommand{\mt}[1]{\ensuremath{#1}}
\newcommand{\nm}[1]{\textrm{#1}}

\newcommand\bsc[2][\DefaultOpt]{%
  \def\DefaultOpt{#2}%
  \section[#1]{#2}%
}
\newcommand\ssc[2][\DefaultOpt]{%
  \def\DefaultOpt{#2}%
  \subsection[#1]{#2}%
}
\newcommand{\bgpf}{\begin{proof} $ $\newline}

\newcommand{\bgeq}{\begin{equation*}}
\newcommand{\eeq}{\end{equation*}}	

\newcommand{\balist}{\begin{enumerate}[label=\alph*.]}
\newcommand{\elist}{\end{enumerate}}

\newcommand{\bilist}{\begin{enumerate}[label=\roman*)]}	

\newcommand{\bgsp}{\begin{split}}
% \newcommand{\esp}{\end{split}} % doesn't work for some reason.

\newcommand\prs[1]{~~~\textbf{(#1)}}

\newcommand{\lt}[1]{\textbf{Let: } #1}
     							   %  if you're setting it to be true
\newcommand{\supp}[1]{\textbf{Suppose: } #1}
     							   %  Suppose (if it'll end up false)
\newcommand{\wts}[1]{\textbf{Want to show: } #1}
     							   %  Want to show
\newcommand{\as}[1]{\textbf{Assume: } #1}
     							   %  if you think it follows from truth
\newcommand{\bpth}[1]{\textbf{(#1)}}

\newcommand{\step}[2]{\begin{equation}\tag{#2}#1\end{equation}}
\newcommand{\epf}{\end{proof}}

\newcommand{\dbs}[3]{\mt{#1_{#2_#3}}}

\newcommand{\sidenote}[1]{-----------------------------------------------------------------Side Note----------------------------------------------------------------
#1 \

---------------------------------------------------------------------------------------------------------------------------------------------}

% Analysis / Logical commands:

\newcommand{\br}{\mt{\mathbb{R}} }       % |R
\newcommand{\bq}{\mt{\mathbb{Q}} }       % |Q
\newcommand{\bn}{\mt{\mathbb{N}} }       % |N
\newcommand{\bc}{\mt{\mathbb{C}} }       % |C
\newcommand{\bz}{\mt{\mathbb{Z}} }       % |Z

\newcommand{\ep}{\mt{\epsilon} }         % epsilon
\newcommand{\fa}{\mt{\forall} }          % for all
\newcommand{\afa}{\mt{\alpha} }
\newcommand{\bta}{\mt{\beta} }
\newcommand{\mem}{\mt{\in} }
\newcommand{\exs}{\mt{\exists} }

\newcommand{\es}{\mt{\emptyset} }        % empty set
\newcommand{\sbs}{\mt{\subset} }         % subset of
\newcommand{\fs}[2]{\{\uw{#1}{1}, \uw{#1}{2}, ... \uw{#1}{#2}\}}

\newcommand{\lra}{ \mt{\longrightarrow} } % implies ----->
\newcommand{\rar}{ \mt{\Rightarrow} }     % implies -->

\newcommand{\lla}{ \mt{\longleftarrow} }  % implies <-----
\newcommand{\lar}{ \mt{\Leftarrow} }      % implies <--

\newcommand{\av}[1]{\mt{|}#1\mt{|}}  % absolute value

\newcommand{\prn}[1]{(#1)}
\newcommand{\bk}[1]{\{#1\}}

\newcommand{\ps}{\mt{+} }
\newcommand{\ms}{\mt{-} }

\newcommand{\ls}{\mt{<} }
\newcommand{\gr}{\mt{>} }

\newcommand{\lse}{\mt{\leq} }
\newcommand{\gre}{\mt{\geq} }

\newcommand{\eql}{\mt{=} }

\newcommand{\pr}{\mt{^\prime} } 		   % prime (i.e. R')
\newcommand{\uw}[2]{#1\mt{_{#2}}}
\newcommand{\uf}[2]{#1\mt{^{#2}}}
\newcommand{\frc}[2]{\mt{\frac{#1}{#2}}}
\newcommand{\lmti}[1]{\mt{\displaystyle{\lim_{#1 \to \infty}}}}
\newcommand{\limt}[2]{\mt{\displaystyle{\lim_{#1 \to #2}}}}

\newcommand{\bnm}[2]{\mt{#1\setminus{#2}}}
\newcommand{\bnt}[2]{\mt{\textrm{#1}\setminus{\textrm{#2}}}}
\newcommand{\bi}{\bnm{\mathbb{R}}{\mathbb{Q}}}

\newcommand{\urng}[2]{\mt{\bigcup_{#1}^{#2}}}
\newcommand{\nrng}[2]{\mt{\bigcap_{#1}^{#2}}}
\newcommand{\nck}[2]{\mt{{#1 \choose #2}}}

\newcommand{\nbho}[3]{\textrm{N(}#1, #2\textrm{) }\cap \textrm{ #3} \neq \emptyset}
     							   %  N(x, eps) intersect S \= emptyset
\newcommand{\nbhe}[3]{\textrm{N(}#1, #2\textrm{) }\cap \textrm{ #3} = \emptyset}
     							   %  N(x, eps) intersect S  = emptyset
\newcommand{\dnbho}[3]{\textrm{N*(}#1, #2\textrm{) }\cap \textrm{ #3} \neq \emptyset}
     							   %  N*(x, eps) intersect S \= emptyset
\newcommand{\dnbhe}[3]{\textrm{N*(}#1, #2\textrm{) }\cap \textrm{ #3} = \emptyset}
     							   %  N*(x, eps) intersect S = emptyset
     							   
\newcommand{\eqn}[1]{\[#1\]}
\newcommand{\splt}[1]{\begin{split}#1\end{split}}

\newcommand{\infy}{\mt{\infty} }
     							 
% ----------

\begin{document}

Exam Tuesday, 31st of October (Halloween)

Covers: Section 4.2 (4.2.5 through end of section), 4.3, 4.4

\bsc{Limit Superior \& Limit Inferior}{

\ssc{Definition 4.4.9}{

Let \bk{\uw{s}{n}} be a bounded sequence.

A \textbf{subsequential limit} of \bk{\uw{s}{n}} is a real number s such that s \eql \lmti{k} \dbs{s}{n}{k} for some subsequence \bk{\dbs{s}{n}{k}}.

If S \eql \bk{s \mem \br : \lmti{k} \dbs{s}{n}{k} \eql s for some \bk{\dbs{s}{n}{k}} of \bk{\uw{s}{n}}}, then

\balist
\item the \textbf{limit superior} (or \textbf{upper limit}) of \bk{\uw{s}{n}} is given by lim sup \uw{s}{n} \eql sup S
\item the \textbf{limit inferior} (or \textbf{lower limit}) of \bk{\uw{s}{n}} is given by lim inf \uw{s}{n} \eql inf S
\item Clearly, lim inf \uw{s}{n} \lse lim sup \uw{s}{n}. If it happens that lim inf \uw{s}{n} \ls lim sup \uw{s}{n}, then we say that \bk{\uw{s}{n}} \textbf{oscillates}.
\elist

}

\sidenote{
\av{\uw{s}{n}} \lse M, \fa n \mem \bn

$-$M \ls \uw{s}{n} \ls M

If \lmti{k}\dbs{s}{n}{k} \eql s \mem S, then

$-$M \ls \dbs{s}{n}{k} \ls M, so

$-$M \ls s \ls M

\#18, page 179
}

\ssc{Theorem 1}{A bounded sequence \bk{\uw{s}{n}} converges to s iff lim inf \uw{s}{n} \eql lim sup \uw{s}{n} \eql s}

\bgpf
\lra 
Assume \bk{\uw{s}{n}} converges to s.

By Theorem 4.4.4, S \eql \bk{s} (contains only one element).

Then, 

lim inf \uw{s}{n} \eql inf S \eql s

lim sup \uw{s}{n} \eql sup S \eql s

So,

lim inf \uw{s}{n} \eql lim sup \uw{s}{n} \eql s

\lla 

(see HW 8, Exercise 9, page 194)

\epf
}

\newpage

\ssc{Example 4.4.10}{

\lt{\uw{s}{n} \eql \prn{$-$1}$^n$ \ps \frc{1}{n}}

Show that

lim inf \uw{s}{n} \eql $-$1,

lim sup \uw{s}{n} \eql 1

Notice that if

n is even \rar \uw{s}{n} \eql 1 \ps \frc{1}{n}

n is odd \rar \uw{s}{n} \eql $-$1 \ps \frc{1}{n}

Thus,

\lmti{k} \uw{s}{2k} \eql 1

\lmti{k} \uw{s}{2k + 1} \eql $-$1

Thus,

S \eql \bk{$-$1, 1}

Hence,

lim sup \uw{s}{n} \eql 1

lim inf \uw{s}{n} \eql $-$1

}

\ssc{Theorem 4.4.11}{

Let \bk{\uw{s}{n}} be a bounded sequence and let 

\uf{s}{*} \eql lim sup \uw{s}{n} 

\uw{s}{*} \eql lim inf \uw{s}{n}

\balist
\item \fa \ep \gr 0, \exs N(\ep) \mem \bn st
	\eqn{s_n < s^* + \epsilon \textrm{ for n \gre N}}
\item \fa \ep \gr 0 and i \mem \bn, \exs j \gr i st
	\eqn{s_j > s^* - \epsilon}
	i.e. there are an infinite number of terms of \bk{\uw{s}{n}} that are greater than \uf{s}{*} \ms \ep
	
	i.e. in the interval (\uf{s}{*} \ms \ep, \uf{s}{*} \ps \ep), there are an infinite number of terms of \uw{s}{n}.
	
	Outside of that interval, there are a finite number of terms of \uw{s}{n}.
\item \fa \ep \gr 0, \exs N(\ep) \mem \bn st
	\eqn{s_n > s_* - \epsilon \textrm{ \fa n \gre N}}

\item \fa \ep \gr 0 and i \mem \bn, \exs j \gr i st
	\eqn{s_j < s_* + \epsilon}
\elist

\newpage

\bgpf

We shall prove a and b. c and d are similar.

\bpth{a}

Suppose it's false. i.e.:

\supp{\exs \ep \gr 0 st \fa N \mem \bn, \exs n \gre N st}
\eqn{s_n \gre s^* + \epsilon}

\sidenote{
In other words, suppose: \bk{\dbs{s}{n}{k} \gre \uf{s}{*} \ps \ep}

By Theorem 4.4.4, every bounded sequence has a convergent subsequence.

If we let \bk{\dbs{s}{n}{k}} be a subsequence of itself and label it differently:

\bk{\dbs{s}{n}{l}}$_{i = 1}^\infty$, 

then

\dbs{s}{n}{l} \lra s as l \lra $\infty$
}

So, for N \eql 1, \exs \uw{n}{1} \gre N st
\eqn{\dbs{s}{n}{1} \gre s^* + \epsilon}
Then,

for N \eql \uw{n}{1} \ps 1, \exs \uw{n}{2} \gre \uw{n}{1} \ps 1 \gr \uw{n}{1} st
\eqn{\dbs{s}{n}{2} \gre s^* + \epsilon}
So, inductively, we find a subsequence \bk{\dbs{s}{n}{k}} st
\eqn{\dbs{s}{n}{k} \gre \uf{s}{*} + \epsilon \textrm{ \fa k \mem \bn}}
Since \bk{\dbs{s}{n}{k}} is itself a bounded sequence, there is a subsequence of \bk{\dbs{s}{n}{k}} that we refer to by:

\bk{\dbs{s}{n}{l}}$^\infty_{l = 1}$

st

\lmti{l} \dbs{s}{n}{l} \eql s \mem k (Theorem 4.4.7)

where s \gre \uf{s}{*} \ps \ep

Since s \mem S, we see that lim sup \uw{s}{n} \eql \uf{s}{*} \gre \uf{s}{*} \ps \ep, which is a contradiction.

Hence, \bpth{a} is true.

\bpth{b}

Suppose it's false. i.e.:

\supp{\exs \ep \gr 0 and \exs i \mem \bn st \fa j \gr i,}
\eqn{s_j \lse s^* - \epsilon}
Thus, if \bk{\dbs{s}{n}{k}} is a subsequence st \lmti{k} \dbs{s}{n}{k} \eql s, then
\eqn{s \lse s^* - \epsilon}
which is like saying:

\uf{s}{*} \lse \uf{s}{*} \ms \ep

(a contradiction)

For further clarification, notice that \uf{s}{*} \ms \ep is an upper bound for all s \mem S, which says: \uf{s}{*} \lse \uf{s}{*} \ms \ep

(a contradiction)

\textbf{Summary:}

In \bpth{a}, we said \exs \uw{N}{1} \mem \bn st \uw{s}{n} \ls s \ps \ep \fa n \gre \uw{N}{1}

In \bpth{b}, we said \exs \uw{N}{2} \mem \bn st s \ms \ep \ls \uw{s}{n} \fa n \gre \uw{N}{2}

\newpage

At the bottom of page 190:

Furthermore, 

if \uf{s}{*} \mem \br satisfying \bpth{a} and \bpth{b}, 

then \uf{s}{*} \eql lim sup \uw{s}{n}

Also,

if \uw{s}{*} \mem \br satisfying \bpth{c} and \bpth{d},

then \uw{s}{*} \eql lim inf \uw{s}{n}

We shall complete the proof by proving the result for \uf{s}{*}

\

\lt{\uf{s}{*} \mem \br satisfy \bpth{a} and \bpth{b}}

We claim that \uf{s}{*} \eql lim sup \uw{s}{n}, and will prove it by contradiction.

\supp{\uf{s}{*} $\neq$ lim sup \uw{s}{n}}

Case:

\bilist
\item \uf{s}{*} \gr lim sup \uw{s}{n}
	
	So, \uf{s}{*} \ms \ep is between lim sup \uw{s}{n} and \uf{s}{*}
	
	Let:
	\eqn{\epsilon = \frac{s^* - \textrm{lim sup \uw{s}{n}}}{2}}
	By \bpth{b}, for this \ep \gr 0, and for i \mem \bn, \exs j \mem \bn st
	
	j \gr i and
	\eqn{s_j \gr s^* - \epsilon}
	
	Since there are an infinite number of possible values of j, there is a subsequence \bk{\dbs{s}{n}{k}} st
	\eqn{\dbs{s}{n}{k} \gr s^* - \epsilon}
	\fa k \mem \bn
	
	This contradicts the definition of lim sup \uw{s}{n}.
	
	Thus, there is a further subsequence converging to a limit s st
	\eqn{s \gre s^* - \epsilon \gre s^*}
	
	Which is also a contradiction.
\item 
\elist
\epf

}

\end{document}