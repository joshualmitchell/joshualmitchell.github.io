% Thank you Josh Davis for this template!
% https://github.com/jdavis/latex-homework-template/blob/master/homework.tex

\documentclass{article}

\newcommand{\hmwkTitle}{HW\ \#3}

% Packages

\usepackage{fancyhdr}
\usepackage{extramarks}
\usepackage{amsmath}
\usepackage{amssymb}
\usepackage{amsthm}
\usepackage{amsfonts}
\usepackage{tikz}
\usepackage[plain]{algorithm}
\usepackage{algpseudocode}
\usepackage{enumitem}
\usepackage{chngcntr}

% Libraries

\usetikzlibrary{automata, positioning, arrows}

%
% Basic Document Settings
%

\topmargin=-0.45in
\evensidemargin=0in
\oddsidemargin=0in
\textwidth=6.5in
\textheight=9.0in
\headsep=0.25in

\linespread{1.1}

\pagestyle{fancy}
\lhead{\hmwkAuthorName}
\chead{}
\rhead{\hmwkClass\ (\hmwkClassInstructor): \hmwkTitle}
\lfoot{\lastxmark}
\cfoot{\thepage}

\renewcommand\headrulewidth{0.4pt}
\renewcommand\footrulewidth{0.4pt}

\setlength\parindent{0pt}
\setcounter{secnumdepth}{0}

\newcommand{\hmwkClass}{MATH 3380 / Analysis 1}        % Class
\newcommand{\hmwkClassInstructor}{Dr. Welsh}           % Instructor
\newcommand{\hmwkAuthorName}{\textbf{Joshua Mitchell}} % Author

%
% Title Page
%

\title{
    \vspace{2in}
    \textmd{\textbf{\hmwkClass:\ \hmwkTitle}}\\
    \normalsize\vspace{0.1in}\small\vspace{0.1in}\large{\textit{\hmwkClassInstructor}}
    \vspace{3in}
}

\author{\hmwkAuthorName}
\date{}

\renewcommand{\part}[1]{\textbf{\large Part \Alph{partCounter}}\stepcounter{partCounter}\\}

% Integral dx
\newcommand{\dx}{\mathrm{d}x}

%
% Various Helper Commands
%

% For derivatives
\newcommand{\deriv}[1]{\frac{\mathrm{d}}{\mathrm{d}x} (#1)}

% For partial derivatives
\newcommand{\pderiv}[2]{\frac{\partial}{\partial #1} (#2)}


% Alias for the Solution section header
\newcommand{\solution}{\textbf{\large Solution}}

% Probability commands: Expectation, Variance, Covariance, Bias
\newcommand{\E}{\mathrm{E}}
\newcommand{\Var}{\mathrm{Var}}
\newcommand{\Cov}{\mathrm{Cov}}
\newcommand{\Bias}{\mathrm{Bias}}

% Formatting commands:

\newcommand{\mt}[1]{\ensuremath{#1}}
\newcommand{\nm}[1]{\textrm{#1}}

\newcommand\bsc[2][\DefaultOpt]{%
  \def\DefaultOpt{#2}%
  \section[#1]{#2}%
}
\newcommand\ssc[2][\DefaultOpt]{%
  \def\DefaultOpt{#2}%
  \subsection[#1]{#2}%
}
\newcommand{\bgpf}{\begin{proof} $ $\newline}

\newcommand{\bgeq}{\begin{equation*}}
\newcommand{\eeq}{\end{equation*}}	

\newcommand{\balist}{\begin{enumerate}[label=\alph*.]}
\newcommand{\elist}{\end{enumerate}}

\newcommand{\bilist}{\begin{enumerate}[label=\roman*)]}	

\newcommand{\bgsp}{\begin{split}}
% \newcommand{\esp}{\end{split}} % doesn't work for some reason.

\newcommand\prs[1]{~~~\textbf{(#1)}}

\newcommand{\lt}[1]{\textbf{Let: } #1}
     							   %  if you're setting it to be true
\newcommand{\supp}[1]{\textbf{Suppose: } #1}
     							   %  Suppose (if it'll end up false)
\newcommand{\wts}[1]{\textbf{Want to show: } #1}
     							   %  Want to show
\newcommand{\as}[1]{\textbf{Assume: } #1}
     							   %  if you think it follows from truth
\newcommand{\bpth}[1]{\textbf{(#1)}}

\newcommand{\step}[2]{\begin{equation}\tag{#2}#1\end{equation}}
\newcommand{\epf}{\end{proof}}

\newcommand{\sidenote}[1]{-----------------------------------------------------------------Side Notes---------------------------------------------------------------
#1 \

---------------------------------------------------------------------------------------------------------------------------------------------}

% Analysis / Logical commands:

\newcommand{\br}{\mathbb{R}}       % |R
\newcommand{\bq}{\mathbb{Q}}       % |Q
\newcommand{\bn}{\mathbb{N}}       % |N
\newcommand{\bc}{\mathbb{C}}       % |C
\newcommand{\bz}{\mathbb{Z}}       % |Z

\newcommand{\ep}{\epsilon}         % epsilon
\newcommand{\fa}{\forall}          % for all

\newcommand{\es}{\emptyset}        % empty set
\newcommand{\sbs}{\subset}         % subset of

\newcommand{\lra}{\longrightarrow} % implies ----->
\newcommand{\rar}{\Rightarrow}     % implies -->

\newcommand{\lla}{\longleftarrow}  % implies <-----
\newcommand{\lar}{\Leftarrow}      % implies <--

\newcommand{\pr}{^\prime} 		   % prime (i.e. R')

\newcommand{\bnm}[2]{\mt{#1\setminus{#2}}}
\newcommand{\bnt}[2]{\mt{\textrm{#1}\setminus{\textrm{#2}}}}
\newcommand{\bi}{\bnm{\mathbb{R}}{\mathbb{Q}}}

\newcommand{\nbho}[3]{\textrm{N(}#1, #2\textrm{) }\cap \textrm{ #3} \neq \emptyset}
     							   %  N(x, eps) intersect S \= emptyset
\newcommand{\nbhe}[3]{\textrm{N(}#1, #2\textrm{) }\cap \textrm{ #3} = \emptyset}
     							   %  N(x, eps) intersect S  = emptyset
\newcommand{\dnbho}[3]{\textrm{N*(}#1, #2\textrm{) }\cap \textrm{ #3} \neq \emptyset}
     							   %  N*(x, eps) intersect S \= emptyset
\newcommand{\dnbhe}[3]{\textrm{N*(}#1, #2\textrm{) }\cap \textrm{ #3} = \emptyset}
     							   %  N*(x, eps) intersect S = emptyset
     							 

\begin{document}

HW 2: page 140-141, \#2-5 (Section 3.4)

\bsc{Theorem 3.3.10}{

Each of the following is equivalent to the AP:

\balist
\item \fa z \mem \br, \exs n \mem \bn st n $>$ z
\item \fa x $>$ 0, y \mem \br, \exs n \mem \bn st nx $>$ y
\item \fa x $>$ 0, \exs n \mem \bn st 0 $<$ \frc{1}{n} $<$ x
\elist

\bgpf

We shall prove:

\bilist
\item AP \rar a
\item a \rar b
\item b \rar c
\item c \rar AP
\elist

In other words, they all imply each other.

\ssc{a. AP \rar a}{

\supp{a is false.}

So, \fa z \mem \br, \exs n \mem \bn, P(z, n) (st n $\leq$ z) \textbf{???}

\sidenote{
$\neg$[\exs \uw{x}{1} \fa \uw{x}{2} st p(\uw{x}{1}, \uw{x}{2})] \eql \

\fa \uw{x}{1}, \exs \uw{x}{2} st $\neg$p(\uw{x}{1}, \uw{x}{2})
}

\exs \uw{z}{0} \mem \br st \fa n \mem \bn, n $\leq$ \uw{z}{0}

This indicates that the AP is false.

Thus, AP \rar a.

}
\ssc{b. a \rar b}{

\as{a) is true.}

\lt{z \eql \frc{y}{x} \mem \br}

By \bpth{a}, \exs n \mem \bn st

n $>$ \frc{y}{x}

nx $>$ y

Hence, a \rar b is true.

}
\ssc{c. b \rar c}{

\as{b) is true.}

\fa x $>$ 0, if y \eql 1,

we see from \bpth{b} that \exs n \mem \bn st nx $>$ 1

Then, 

x $>$ \frc{1}{n} $>$ 0.

Hence, b \rar c.

}

\ssc{d. c \rar AP}{

Reminder of c: \fa x where 0 $<$ x \mem \br, \exs n \mem \bn st. 0 $<$ \frc{1}{n} $<$ x

\supp{\bn is bounded above. (In other words, that the AP is false.}

Thus, \exs \uw{z}{0} \mem \br st 0 $<$ n $\leq$ \uw{z}{0}, \fa n \mem \bn

0 $<$ n $\leq$ \uw{z}{0}

\frc{1}{n} $\geq$ \frc{1}{\uw{z}{0}}

This contradicts c with x \eql \frc{1}{\uw{z}{0}} where 0 $<$ \frc{1}{\uw{z}{0}} \mem \br

Hence, result.
}

\epf

}

\bsc{Theorems 3.3.13 and 3.3.15}{

\lt{x, y \mem \br st x $<$ y}

Then:

\balist
\item \exs r \mem \bq st x $<$ r $<$ y
\item \exs z \mem \bnt{\br}{\bq} st x $<$ z $<$ y
\elist

\ssc{a}{
Case:

\bpth{i}: y $>$ 0

y \eql 0.\uw{a}{1}\uw{a}{2}...\uw{a}{n} i.e. 0.141 \eql \frc{141}{1000}

\bpth{ii}: y $\leq$ 0

$-$y $\geq$ 0, $-$y $<$ $-$x, 0 $\leq$ $-$y $<$ $-$x

By case \bpth{i}, \exs r \mem \bq st 

$-$y $<$ r $<$ $-$x

y $>$ $-$r $>$ x

x $<$ $-$r $<$ y

}
\ssc{b}{

\exs z \mem \bnt{\br}{\bq} st x $<$ z $<$ y

Apply \bpth{a} to $\frac{x}{\sqrt{2}}$ $<$ $\frac{y}{\sqrt{2}}$  to find r \mem \bq st

$\frac{x}{\sqrt{2}}$ $<$ r $<$ $\frac{y}{\sqrt{2}}$

x $<$ r$\sqrt{2}$ $<$ y

\lt{r$\sqrt{2}$ \eql z}

x $<$ z $<$ y

}
\

Hence, result.
}

\newpage

\bsc{Section 3.4: Topology of \br}{}

\bsc{Definitions 3.4.1 and 3.4.2}{

Let x \mem \br and \ep $>$ 0. \

\


\bpth{a}

An \ep-neighborhood of x is:

N(x, \ep) \eql \{y \mem \br: $|y - x|$ $<$ \ep\}

\

\bpth{b} \

A deleted \ep-neighborhood of x is:

N*(x, \ep) \eql \{y \mem \br: 0 $<$ $|y - x|$ $<$ \ep\} \
}

\bsc{Open Set Topology: Definition 3.4.3 (interior / boundary point)}{

\lt{S \sbs \br}

A point x \mem \br is an \textbf{interior point} of S if \exs \ep $>$ 0 st N(x, \ep) \sbs S. \

\

If, \fa \ep $>$ 0,

$\nbho{x}{\ep}{S}$

and 

$\nbho{x}{\ep}{\bnt{\br}{S	}}$

Then x is a \textbf{boundary point} of S.

The set of all interior points is denoted by \textbf{int S}.

The set of all boundary points is denoted by \textbf{bd S}.

\textbf{Nota Bene (N.B.)}:

int S \sbs S and bd S \eql bd (\bnt{\br}{S})

\sidenote{
\lt{x \mem int S}

Then \exs \ep $>$ 0 st N(x, \ep) \sbs S

In particular, x \mem S. Thus, int S \sbs S.

\lt{S$^C$ \eql \bnt{\br}{S}, and \bnt{\br}{S$^C$ \eql S}}

Then s \mem bd S$^C$ if \fa \ep $>$ 0, 

$\nbho{x}{\ep}{S$^C$}$

$\nbho{x}{\ep}{\bnt{\br}{S$^C$}}$

Thus, N(x, \ep) $\cap$ (\bnt{\br}{S}) $\neq$ \es, and $\nbho{x}{\ep}{S}$

So, x \mem bd S
}
}

\newpage

\bsc{Theorem 1}{

\lt{x \mem S \sbs \br}

Then either x \mem int S, or x \mem bd S.

\bgpf

\lt{x \mem S \sbs \br}

\bilist
\item \exs \ep $>$ 0 st N(x, \ep) \sbs S. Then, by def, x \mem int S
\item \fa \ep $>$ 0, N(x, \ep) $\cap$ (\bnt{\br}{S}) $\neq$ \es.
	
	However, since x \mem S, then $\nbho{x}{\ep}{S}$.
	
	By definition, x \mem bd S.
\elist

Hence, result.

\epf

}

\bsc{Section 3.4.4 Examples}{

\balist
\item \lt{S \eql (0, 5)}
	
	Here, int S \eql (0, 5) and bd S \eql \{0, 5\} \
	
	To see this*****, \
	
	\lt{x \mem (0, 5), \ep \eql min \{x, 5$-$x\}} \
	
	Then N(x, \ep) \sbs (0, 5)
	
	To see this, let y \mem N(x, \ep).
	
	\wts{0 $<$ y $<$ 5}
	
	Since y \mem N(x, \ep), we have
	\step{x - \ep < y < x + \ep}{1}
	Notice that
	\step{\ep \leq x}{2}
	and
	\step{\ep \leq 5 - x}{3}
	From \bpth{3},
	\step{x + \ep \leq x + (5-x) \eql 5}{4}
	From \bpth{2},
	\step{x - x \leq x - \ep,\textrm{   } 0 \leq x - \ep}{5}
	From \bpth{1}, \bpth{4}, \bpth{5}, \
	
	0 $\leq$ x $-$ \ep $<$ y $\leq$ x $+$ \ep $<$ 5, \
	
	We see that y \mem (0, 5).
	
	Hence, 0 $<$ y $<$ 5.
	
	Since N(x, \ep) \sbs (0, 5), we see that \
	
	int S \eql (0, 5) \eql S
	
\newpage
	
	\wts{0 \mem bd S}
	
	\lt{0 $<$ \ep $<$ 5} 
	
	Notice that:
	
	N(0, \ep) $\cap$ (0, 5) $\neq$ \es and N(0, \ep) $\cap$ (\bnt{\br}{(0, 5)}) $\neq$ \es
	
	Using y \eql ($+$/$-$) \frc{\ep}{2}, notice that:
	
	y \mem N(0, \ep) since $|(+/-) \frac{\epsilon}{2}| < \epsilon$
	
	**2**
	
\item \lt{S \eql [0, 5]}

	Here, int S \eql (0, 5), bd S \eql \{0, 5\}
	
	Notice that bd S \sbs S
	
\item \lt{S \eql [0, 5)}
	
	Here, int S \eql (0, 5), bd S \eql {0, 5}
	
	Notice that some bd points are in S, but some aren't.
	
\item \lt{S \eql [2, $\infty$)}
	
	Here, int S \eql (2, $\infty$), bd S \eql \{2\}

\item \lt{S \eql \br}

	int S \eql \br \eql S, bd S \eql \es
	
	Here, bd S \sbs S
	
\elist

*****


\sidenote{
----(----(---|---|---)--------)----- \

0, x-ep, y, x, xplEp, 5 \

from 0 to x is x, from x to 5 is 5-x
}
**2**

\sidenote{
-----(---|---(---|---)----)----

0-ep, -ep/2, 0, ep/2, 0plEp, 5
}
}
\end{document}