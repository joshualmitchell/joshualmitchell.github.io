% Thank you Josh Davis for this template!
% https://github.com/jdavis/latex-homework-template/blob/master/homework.tex

\documentclass{article}

\newcommand{\hmwkTitle}{Lec\ \#26}

% % Packages

\usepackage{fancyhdr}
\usepackage{extramarks}
\usepackage{amsmath}
\usepackage{amssymb}
\usepackage{amsthm}
\usepackage{amsfonts}
\usepackage{tikz}
\usepackage[plain]{algorithm}
\usepackage{algpseudocode}
\usepackage{enumitem}
\usepackage{chngcntr}

% Libraries

\usetikzlibrary{automata, positioning, arrows}

%
% Basic Document Settings
%

\topmargin=-0.45in
\evensidemargin=0in
\oddsidemargin=0in
\textwidth=6.5in
\textheight=9.0in
\headsep=0.25in

\linespread{1.1}

\pagestyle{fancy}
\lhead{\hmwkAuthorName}
\chead{}
\rhead{\hmwkClass\ (\hmwkClassInstructor): \hmwkTitle}
\lfoot{\lastxmark}
\cfoot{\thepage}

\renewcommand\headrulewidth{0.4pt}
\renewcommand\footrulewidth{0.4pt}

\setlength\parindent{0pt}
\setcounter{secnumdepth}{0}

\newcommand{\hmwkClass}{MATH 3380 / Analysis 1}        % Class
\newcommand{\hmwkClassInstructor}{Dr. Welsh}           % Instructor
\newcommand{\hmwkAuthorName}{\textbf{Joshua Mitchell}} % Author

%
% Title Page
%

\title{
    \vspace{2in}
    \textmd{\textbf{\hmwkClass:\ \hmwkTitle}}\\
    \normalsize\vspace{0.1in}\small\vspace{0.1in}\large{\textit{\hmwkClassInstructor}}
    \vspace{3in}
}

\author{\hmwkAuthorName}
\date{}

\renewcommand{\part}[1]{\textbf{\large Part \Alph{partCounter}}\stepcounter{partCounter}\\}

% Integral dx
\newcommand{\dx}{\mathrm{d}x}

%
% Various Helper Commands
%

% For derivatives
\newcommand{\deriv}[1]{\frac{\mathrm{d}}{\mathrm{d}x} (#1)}

% For partial derivatives
\newcommand{\pderiv}[2]{\frac{\partial}{\partial #1} (#2)}


% Alias for the Solution section header
\newcommand{\solution}{\textbf{\large Solution}}

% Probability commands: Expectation, Variance, Covariance, Bias
\newcommand{\E}{\mathrm{E}}
\newcommand{\Var}{\mathrm{Var}}
\newcommand{\Cov}{\mathrm{Cov}}
\newcommand{\Bias}{\mathrm{Bias}}

% Formatting commands:

\newcommand{\mt}[1]{\ensuremath{#1}}
\newcommand{\nm}[1]{\textrm{#1}}

\newcommand\bsc[2][\DefaultOpt]{%
  \def\DefaultOpt{#2}%
  \section[#1]{#2}%
}
\newcommand\ssc[2][\DefaultOpt]{%
  \def\DefaultOpt{#2}%
  \subsection[#1]{#2}%
}
\newcommand{\bgpf}{\begin{proof} $ $\newline}

\newcommand{\bgeq}{\begin{equation*}}
\newcommand{\eeq}{\end{equation*}}	

\newcommand{\balist}{\begin{enumerate}[label=\alph*.]}
\newcommand{\elist}{\end{enumerate}}

\newcommand{\bilist}{\begin{enumerate}[label=\roman*)]}	

\newcommand{\bgsp}{\begin{split}}
% \newcommand{\esp}{\end{split}} % doesn't work for some reason.

\newcommand\prs[1]{~~~\textbf{(#1)}}

\newcommand{\lt}[1]{\textbf{Let: } #1}
     							   %  if you're setting it to be true
\newcommand{\supp}[1]{\textbf{Suppose: } #1}
     							   %  Suppose (if it'll end up false)
\newcommand{\wts}[1]{\textbf{Want to show: } #1}
     							   %  Want to show
\newcommand{\as}[1]{\textbf{Assume: } #1}
     							   %  if you think it follows from truth
\newcommand{\bpth}[1]{\textbf{(#1)}}

\newcommand{\step}[2]{\begin{equation}\tag{#2}#1\end{equation}}
\newcommand{\epf}{\end{proof}}

\newcommand{\sidenote}[1]{-----------------------------------------------------------------Side Notes---------------------------------------------------------------
#1 \

---------------------------------------------------------------------------------------------------------------------------------------------}

% Analysis / Logical commands:

\newcommand{\br}{\mathbb{R}}       % |R
\newcommand{\bq}{\mathbb{Q}}       % |Q
\newcommand{\bn}{\mathbb{N}}       % |N
\newcommand{\bc}{\mathbb{C}}       % |C
\newcommand{\bz}{\mathbb{Z}}       % |Z

\newcommand{\ep}{\epsilon}         % epsilon
\newcommand{\fa}{\forall}          % for all

\newcommand{\es}{\emptyset}        % empty set
\newcommand{\sbs}{\subset}         % subset of

\newcommand{\lra}{\longrightarrow} % implies ----->
\newcommand{\rar}{\Rightarrow}     % implies -->

\newcommand{\lla}{\longleftarrow}  % implies <-----
\newcommand{\lar}{\Leftarrow}      % implies <--

\newcommand{\pr}{^\prime} 		   % prime (i.e. R')

\newcommand{\bnm}[2]{\mt{#1\setminus{#2}}}
\newcommand{\bnt}[2]{\mt{\textrm{#1}\setminus{\textrm{#2}}}}
\newcommand{\bi}{\bnm{\mathbb{R}}{\mathbb{Q}}}

\newcommand{\nbho}[3]{\textrm{N(}#1, #2\textrm{) }\cap \textrm{ #3} \neq \emptyset}
     							   %  N(x, eps) intersect S \= emptyset
\newcommand{\nbhe}[3]{\textrm{N(}#1, #2\textrm{) }\cap \textrm{ #3} = \emptyset}
     							   %  N(x, eps) intersect S  = emptyset
\newcommand{\dnbho}[3]{\textrm{N*(}#1, #2\textrm{) }\cap \textrm{ #3} \neq \emptyset}
     							   %  N*(x, eps) intersect S \= emptyset
\newcommand{\dnbhe}[3]{\textrm{N*(}#1, #2\textrm{) }\cap \textrm{ #3} = \emptyset}
     							   %  N*(x, eps) intersect S = emptyset
     							 

% ----------

% Packages

\usepackage{fancyhdr}
\usepackage{extramarks}
\usepackage{amsmath}
\usepackage{amssymb}
\usepackage{amsthm}
\usepackage{amsfonts}
\usepackage{tikz}
\usepackage[plain]{algorithm}
\usepackage{algpseudocode}
\usepackage{enumitem}
\usepackage{chngcntr}

% Libraries

\graphicspath{{/Users/jm/iclouddrive/3380pics/}}

\usetikzlibrary{automata, positioning, arrows}

%
% Basic Document Settings
%

\topmargin=-0.45in
\evensidemargin=0in
\oddsidemargin=0in
\textwidth=6.5in
\textheight=9.0in
\headsep=0.25in

\linespread{1.1}

\pagestyle{fancy}
\lhead{\hmwkAuthorName}
\chead{}
\rhead{\hmwkClass\ (\hmwkClassInstructor): \hmwkTitle}
\lfoot{\lastxmark}
\cfoot{\thepage}

\renewcommand\headrulewidth{0.4pt}
\renewcommand\footrulewidth{0.4pt}

\setlength\parindent{0pt}
\setcounter{secnumdepth}{0}

\newcommand{\hmwkClass}{MATH 3380 / Analysis 1}        % Class
\newcommand{\hmwkClassInstructor}{Dr. Welsh}           % Instructor
\newcommand{\hmwkAuthorName}{\textbf{Joshua Mitchell}} % Author

%
% Title Page
%

\title{
    \vspace{2in}
    \textmd{\textbf{\hmwkClass:\ \hmwkTitle}}\\
    \normalsize\vspace{0.1in}\small\vspace{0.1in}\large{\textit{\hmwkClassInstructor}}
    \vspace{3in}
}

\author{\hmwkAuthorName}
\date{}

\renewcommand{\part}[1]{\textbf{\large Part \Alph{partCounter}}\stepcounter{partCounter}\\}

% Integral dx
\newcommand{\dx}{\mathrm{d}x}

%
% Various Helper Commands
%

% For derivatives
\newcommand{\deriv}[1]{\frac{\mathrm{d}}{\mathrm{d}x} (#1)}

% For partial derivatives
\newcommand{\pderiv}[2]{\frac{\partial}{\partial #1} (#2)}


% Alias for the Solution section header
\newcommand{\solution}{\textbf{\large Solution}}

% Probability commands: Expectation, Variance, Covariance, Bias
\newcommand{\E}{\mathrm{E}}
\newcommand{\Var}{\mathrm{Var}}
\newcommand{\Cov}{\mathrm{Cov}}
\newcommand{\Bias}{\mathrm{Bias}}

% Formatting commands:

\newcommand{\mt}[1]{\ensuremath{#1}}
\newcommand{\nm}[1]{\textrm{#1}}

\newcommand\bsc[2][\DefaultOpt]{%
  \def\DefaultOpt{#2}%
  \section[#1]{#2}%
}
\newcommand\ssc[2][\DefaultOpt]{%
  \def\DefaultOpt{#2}%
  \subsection[#1]{#2}%
}
\newcommand{\bgpf}{\begin{proof} $ $\newline}

\newcommand{\bgeq}{\begin{equation*}}
\newcommand{\eeq}{\end{equation*}}	

\newcommand{\balist}{\begin{enumerate}[label=\alph*.]}
\newcommand{\elist}{\end{enumerate}}

\newcommand{\bilist}{\begin{enumerate}[label=\roman*)]}	

\newcommand{\bgsp}{\begin{split}}
% \newcommand{\esp}{\end{split}} % doesn't work for some reason.

\newcommand\prs[1]{~~~\textbf{(#1)}}

\newcommand{\lt}[1]{\textbf{Let: } #1}
     							   %  if you're setting it to be true
\newcommand{\supp}[1]{\textbf{Suppose: } #1}
     							   %  Suppose (if it'll end up false)
\newcommand{\wts}[1]{\textbf{Want to show: } #1}
     							   %  Want to show
\newcommand{\as}[1]{\textbf{Assume: } #1}
     							   %  if you think it follows from truth
\newcommand{\bpth}[1]{\textbf{(#1)}}

\newcommand{\step}[2]{\begin{equation}\tag{#2}#1\end{equation}}
\newcommand{\epf}{\end{proof}}

\newcommand{\dbs}[3]{\mt{#1_{#2_#3}}}

\newcommand{\sidenote}[1]{-----------------------------------------------------------------Side Note----------------------------------------------------------------
#1 \

---------------------------------------------------------------------------------------------------------------------------------------------}

% Analysis / Logical commands:

\newcommand{\br}{\mt{\mathbb{R}} }       % |R
\newcommand{\bq}{\mt{\mathbb{Q}} }       % |Q
\newcommand{\bn}{\mt{\mathbb{N}} }       % |N
\newcommand{\bc}{\mt{\mathbb{C}} }       % |C
\newcommand{\bz}{\mt{\mathbb{Z}} }       % |Z

\newcommand{\ep}{\mt{\epsilon} }         % epsilon
\newcommand{\fa}{\mt{\forall} }          % for all
\newcommand{\afa}{\mt{\alpha} }
\newcommand{\bta}{\mt{\beta} }
\newcommand{\dta}{\mt{\delta} }
\newcommand{\mem}{\mt{\in} }
\newcommand{\exs}{\mt{\exists} }

\newcommand{\es}{\mt{\emptyset} }        % empty set
\newcommand{\sbs}{\mt{\subset} }         % subset of
\newcommand{\fs}[2]{\{\uw{#1}{1}, \uw{#1}{2}, ... \uw{#1}{#2}\}}

\newcommand{\lra}{ \mt{\longrightarrow} } % implies ----->
\newcommand{\rar}{ \mt{\Rightarrow} }     % implies -->

\newcommand{\lla}{ \mt{\longleftarrow} }  % implies <-----
\newcommand{\lar}{ \mt{\Leftarrow} }      % implies <--

\newcommand{\av}[1]{\mt{|}#1\mt{|}}  % absolute value

\newcommand{\prn}[1]{(#1)}
\newcommand{\bk}[1]{\{#1\}}

\newcommand{\ps}{\mt{+} }
\newcommand{\ms}{\mt{-} }

\newcommand{\ls}{\mt{<} }
\newcommand{\gr}{\mt{>} }

\newcommand{\lse}{\mt{\leq} }
\newcommand{\gre}{\mt{\geq} }

\newcommand{\eql}{\mt{=} }

\newcommand{\pr}{\mt{^\prime} } 		   % prime (i.e. R')
\newcommand{\uw}[2]{#1\mt{_{#2}}}
\newcommand{\uf}[2]{#1\mt{^{#2}}}
\newcommand{\frc}[2]{\mt{\frac{#1}{#2}}}
\newcommand{\lmti}[1]{\mt{\displaystyle{\lim_{#1 \to \infty}}}}
\newcommand{\limt}[2]{\mt{\displaystyle{\lim_{#1 \to #2}}}}

\newcommand{\bnm}[2]{\mt{#1\setminus{#2}}}
\newcommand{\bnt}[2]{\mt{\textrm{#1}\setminus{\textrm{#2}}}}
\newcommand{\bi}{\bnm{\mathbb{R}}{\mathbb{Q}}}

\newcommand{\urng}[2]{\mt{\bigcup_{#1}^{#2}}}
\newcommand{\nrng}[2]{\mt{\bigcap_{#1}^{#2}}}
\newcommand{\nck}[2]{\mt{{#1 \choose #2}}}

\newcommand{\nbho}[3]{\textrm{N(}#1, #2\textrm{) }\cap \textrm{ #3} \neq \emptyset}
     							   %  N(x, eps) intersect S \= emptyset
\newcommand{\nbhe}[3]{\textrm{N(}#1, #2\textrm{) }\cap \textrm{ #3} = \emptyset}
     							   %  N(x, eps) intersect S  = emptyset
\newcommand{\dnbho}[3]{\textrm{N*(}#1, #2\textrm{) }\cap \textrm{ #3} \neq \emptyset}
     							   %  N*(x, eps) intersect S \= emptyset
\newcommand{\dnbhe}[3]{\textrm{N*(}#1, #2\textrm{) }\cap \textrm{ #3} = \emptyset}
     							   %  N*(x, eps) intersect S = emptyset
     							   
\newcommand{\eqn}[1]{\[#1\]}
\newcommand{\splt}[1]{\begin{split}#1\end{split}}

\newcommand{\infy}{\mt{\infty} }
\newcommand{\unn}{\mt{\cup} }
\newcommand{\inn}{\mt{\cap} }
\newcommand\tab[1][1cm]{\hspace*{#1}}

\newcommand{\wit}[1]{\mt{\widetilde{#1}}}
     							 
% ----------

\begin{document}

\bsc{6.1 Continued}{

\ssc{Theorem 6.1.3}{

\lt{I be an interval containing the point c}

\as{f : I \lra \br}

f is differentiable at c 

iff 

for every sequence \bk{\uw{x}{n}} in I st \uw{x}{n} \lra c as n \lra \infy with \uw{x}{n} $\neq$ c \fa n \mem \bn, 

the sequence 
\eqn{\frac{f(x_n) - f(c)}{x_n - c}}
converges.

Furthermore, if f is differentiable at c, then the sequence of quotients converges to f(c).

(this looks weird, haven't we already proved it?

they're just saying it to be specific, emphasizing the p iff q and not p iff not q part)

\bgpf

\lra

\as{f\pr(c) exists}

Then \limt{x}{c} (\frc{f(x) \ms f(c)}{x - c}) \eql f\pr(c).

By Theorem 5.1.8, if \bk{\uw{x}{n}} lies in I, \uw{x}{n} $\neq$ c \fa n \mem \bn, and \uw{x}{n} \lra c as n \lra \infy, then
\eqn{\lmti{n} \frac{f(x_n) - f(c)}{x_n - c} = f\pr(c)}

\lla

Conversely,

\as{\bk{\uw{x}{n}} lies in I, \uw{x}{n} $\neq$ c \fa n \mem \bn, and \uw{x}{n} \lra c as n \lra \infy}

Then,
\eqn{\lmti{n} \frac{f(x_n) - f(c)}{x_n - c}}
exists.

Then, by the negation of Theorem 5.1.10,
\eqn{\frac{f(x) - f(c)}{x - c}}
has a limit at x \eql c.

Hence, result.

\epf 
}

\newpage 

\ssc{Example 6.1.4}{

\balist
\item \lt{f(x) \eql \av{x} \eql x if x \gre 0, $-$x if x \ls 0}. Prove that f is not differentiable at x \eql 0.

Solution:

\lt{\uw{x}{n} \eql \frc{(-1)^n}{n} \fa n \mem \bn}

Notice that if
\eqn{\limt{x}{0} \frac{f(x_n) - f(0)}{x_n - 0}}
exists, then f is differentiable and
\eqn{\lmti{n} \frac{f(x_n) - f(0)}{x_n - 0}}
exists.

Also, notice that \uw{x}{n} $\neq$ 0 \fa n \mem \bn, \uw{x}{n} \lra 0 as n \lra \infy and 
\eqn{s_n = \frac{f(x_n) - f(0)}{x_n - 0} = \frac{|\frac{(-1)^n}{n}| - |0|}{\frac{(-1)^n}{n}} = \frac{\frac{1}{n}}{\frac{(-1)^n}{n}} = \frac{1}{(-1)^n}}
Since \uw{s}{n} \eql $-1$, 1, $-1$, 1, $-1$, 1, ..., \bk{\uw{s}{n}} does not converge.

So by Theorem 6.1.3, f is not differentiable at x \eql 0.

\item \lt{f(x) \eql 3\uf{x}{2} \ps 1 if x \ls 1, 2\uf{x}{3} \ps 2 if x \gre 1.} Prove that f is differentiable at x \eql 1.

Solution: We must prove that
\eqn{\limt{x}{1} \frac{f(x) - f(1)}{x - 1}}
exists.

We know that:

\limt{x}{c} g(x) exists iff \limt{x}{c^-} g(x) \eql \limt{x}{c^+} g(x) \eql L

Left hand side limit:

\eqn{\limt{x}{1-} \frac{f(x) - f(1)}{x - 1} = \limt{x}{1-} \frac{3x^2 + 1 - 4}{x - 1} = \limt{x}{1-} \frac{3(x - 1)(x + 1)	}{x - 1} = \limt{x}{1-} 3(x + 1) = 6}

Right hand side limit:

\eqn{\limt{x}{1+} \frac{f(x) - f(1)}{x - 1} = \limt{x}{1+} \frac{2x^3 + 2 - 4}{x - 1} = \limt{x}{1+} \frac{2(x^3 - 1)}{x - 1} = \limt{x}{1+} \frac{2(x - 1)(x^2 + x + 1)}{x - 1} = \limt{x}{1+} 2(x^2 + x + 1) = 6}
Hence,
\eqn{\limt{x}{1} \frac{f(x) - f(1)}{x - 1} = 6}

\elist

}

\newpage

\ssc{Practice 6.1.5}{

\lt{f : \br \lra \br be defined by f(x) \eql x sin(\frc{1}{x}) if x $\neq$ 0 and 0 if x \eql 0}

Solution:
\eqn{\limt{x}{0} \frac{f(x) - f(0)}{x - 0} = \limt{x}{0} \frac{x \sin{\frac{1}{x}} - 0)}{x - 0}}
\lt{\uw{x}{n} \eql \frc{2}{n\pi} \fa n \mem \bn}

Then \uw{x}{n} $\neq$ 0 \fa n \mem \bn and \uw{x}{n} \lra 0 as n \lra \infy.

However,
\eqn{\frac{f(x_n) - f(0)}{x_n - 0} = \sin{\frac{1}{x_n}} = \sin{\frac{n\pi}{2}}}
Since sin(\frc{n\pi}{2}) \eql 1, 0, $-1$, 0, 1, 0, $-1$, ...,

by Theorem 6.1.3, f is not differentiable at x \eql 0.
}

\ssc{Theorem 6.1.6}{

If f : I \lra \br is differentiable at a point c \mem I, then f is continuous.

\bgpf

Recall:

\limt{x}{c} f(x) \eql f(c) is saying 3 things:

The limit exists, the function is defined at c, and that they're equal to each other.

(you can't say undefined \eql undefined)

\lt{x \mem I with x $\neq$ c}

Then
\eqn{f(x) = (\frac{f(x) - f(c)}{x - c})(x - c) + f(c) \lra f(c)}
as x \lra c.

by Theorem 5.1.13.

Thus, by Theorem 5.2.21(d)(a), f is continuous.

\epf

}

\newpage

\ssc{Theorem 6.1.7}{

\as{f : I \lra \br and g : I \lra \br are differentiable at c \mem I.}

Then,

\balist
\item If k \mem \br, then (kf)\pr(c) \eql k * f\pr(c)
\item (f \ps g)\pr(c) \eql f\pr(c) \ps g\pr(c)
\item (fg)\pr(c) \eql f(c)g\pr(c) \ps f\pr(c)g(c)
\item If g(c) $\neq$ 0, then (\frc{f}{g})\pr(c) \eql \frc{g(c)f\pr(c) - f(c)g\pr(c)}{[g(c)]^2}
\elist

There's no homework this week, but make sure you know how to prove a, b, and c on your own.

\bgpf

\bpth{d}: 

\lt{x \mem I, x $\neq$ c}

Then,

\eqn{\frac{(\frac{f}{g})(x) - (\frac{f}{g})(c)}{x - c} = \frac{\frac{f(x)}{g(x)} - \frac{f(c)}{g(c)}}{x - c} = \frac{f(x)g(c) - f(c)g(x)}{[g(x)g(c)](x - c)}}

We have a problem. What if g(x) \ms g(c) \eql 0?

Since g(c) $\neq$ 0 and g is differentiable at c, we know that g is continuous at c by 5.2 HW problem 11 on 213.

Recall:

\eqn{|a| - |b| \lse ||a| - |b|| \lse |a - b|}
\eqn{|b| \gre |a| - |a - b|}
so,
\eqn{|g(x)| \gre g(c) \ms |g(c)| \ms g(x)| \gr \frac{|g(c)|}{2} \gr 0}

Hence,

\exs \dta \gr 0 st
\eqn{-|g(c) - g(x)| > \frac{-|g(c)|}{2}}
where \av{x \ms c} \ls \dta

So we know that:

\exs an interval J \sbs I st c \mem J and if x \mem J, then g(x) $\neq$ 0.

Thus,
\eqn{(\frac{f}{g})(x) - (\frac{f}{g})(c) = \frac{f(x)g(c) - f(c)g(c) \times f(c)g(c) - f(c)g(x)}{g(x)g(c)(x - c)} = \frac{g(c)[f(x) - f(c)]}{g(x)g(c)(x - c)} - \frac{f(c)[g(x) - g(c)]}{g(x)g(c)(x - c)}}
\eqn{= \frac{g(c)f\pr(c) - f(c)(g\pr(c)}{[g(c)]^2}}
\epf

}

}

\end{document}