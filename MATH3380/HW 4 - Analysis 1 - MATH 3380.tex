% Thank you Josh Davis for this template!
% https://github.com/jdavis/latex-homework-template/blob/master/homework.tex

\documentclass{article}

\newcommand{\hmwkTitle}{HW\ \#4}

% Packages

\usepackage{fancyhdr}
\usepackage{extramarks}
\usepackage{amsmath}
\usepackage{amssymb}
\usepackage{amsthm}
\usepackage{amsfonts}
\usepackage{tikz}
\usepackage[plain]{algorithm}
\usepackage{algpseudocode}
\usepackage{enumitem}
\usepackage{chngcntr}

% Libraries

\usetikzlibrary{automata, positioning, arrows}

%
% Basic Document Settings
%

\topmargin=-0.45in
\evensidemargin=0in
\oddsidemargin=0in
\textwidth=6.5in
\textheight=9.0in
\headsep=0.25in

\linespread{1.1}

\pagestyle{fancy}
\lhead{\hmwkAuthorName}
\chead{}
\rhead{\hmwkClass\ (\hmwkClassInstructor): \hmwkTitle}
\lfoot{\lastxmark}
\cfoot{\thepage}

\renewcommand\headrulewidth{0.4pt}
\renewcommand\footrulewidth{0.4pt}

\setlength\parindent{0pt}
\setcounter{secnumdepth}{0}

\newcommand{\hmwkClass}{MATH 3380 / Analysis 1}        % Class
\newcommand{\hmwkClassInstructor}{Dr. Welsh}           % Instructor
\newcommand{\hmwkAuthorName}{\textbf{Joshua Mitchell}} % Author

%
% Title Page
%

\title{
    \vspace{2in}
    \textmd{\textbf{\hmwkClass:\ \hmwkTitle}}\\
    \normalsize\vspace{0.1in}\small\vspace{0.1in}\large{\textit{\hmwkClassInstructor}}
    \vspace{3in}
}

\author{\hmwkAuthorName}
\date{}

\renewcommand{\part}[1]{\textbf{\large Part \Alph{partCounter}}\stepcounter{partCounter}\\}

% Integral dx
\newcommand{\dx}{\mathrm{d}x}

%
% Various Helper Commands
%

% For derivatives
\newcommand{\deriv}[1]{\frac{\mathrm{d}}{\mathrm{d}x} (#1)}

% For partial derivatives
\newcommand{\pderiv}[2]{\frac{\partial}{\partial #1} (#2)}


% Alias for the Solution section header
\newcommand{\solution}{\textbf{\large Solution}}

% Probability commands: Expectation, Variance, Covariance, Bias
\newcommand{\E}{\mathrm{E}}
\newcommand{\Var}{\mathrm{Var}}
\newcommand{\Cov}{\mathrm{Cov}}
\newcommand{\Bias}{\mathrm{Bias}}

% Formatting commands:

\newcommand{\mt}[1]{\ensuremath{#1}}
\newcommand{\nm}[1]{\textrm{#1}}

\newcommand\bsc[2][\DefaultOpt]{%
  \def\DefaultOpt{#2}%
  \section[#1]{#2}%
}
\newcommand\ssc[2][\DefaultOpt]{%
  \def\DefaultOpt{#2}%
  \subsection[#1]{#2}%
}
\newcommand{\bgpf}{\begin{proof} $ $\newline}

\newcommand{\bgeq}{\begin{equation*}}
\newcommand{\eeq}{\end{equation*}}	

\newcommand{\balist}{\begin{enumerate}[label=\alph*.]}
\newcommand{\elist}{\end{enumerate}}

\newcommand{\bilist}{\begin{enumerate}[label=\roman*)]}	

\newcommand{\bgsp}{\begin{split}}
% \newcommand{\esp}{\end{split}} % doesn't work for some reason.

\newcommand\prs[1]{~~~\textbf{(#1)}}

\newcommand{\lt}[1]{\textbf{Let: } #1}
     							   %  if you're setting it to be true
\newcommand{\supp}[1]{\textbf{Suppose: } #1}
     							   %  Suppose (if it'll end up false)
\newcommand{\wts}[1]{\textbf{Want to show: } #1}
     							   %  Want to show
\newcommand{\as}[1]{\textbf{Assume: } #1}
     							   %  if you think it follows from truth
\newcommand{\bpth}[1]{\textbf{(#1)}}

\newcommand{\step}[2]{\begin{equation}\tag{#2}#1\end{equation}}
\newcommand{\epf}{\end{proof}}

\newcommand{\sidenote}[1]{-----------------------------------------------------------------Side Notes---------------------------------------------------------------
#1 \

---------------------------------------------------------------------------------------------------------------------------------------------}

% Analysis / Logical commands:

\newcommand{\br}{\mathbb{R}}       % |R
\newcommand{\bq}{\mathbb{Q}}       % |Q
\newcommand{\bn}{\mathbb{N}}       % |N
\newcommand{\bc}{\mathbb{C}}       % |C
\newcommand{\bz}{\mathbb{Z}}       % |Z

\newcommand{\ep}{\epsilon}         % epsilon
\newcommand{\fa}{\forall}          % for all

\newcommand{\es}{\emptyset}        % empty set
\newcommand{\sbs}{\subset}         % subset of

\newcommand{\lra}{\longrightarrow} % implies ----->
\newcommand{\rar}{\Rightarrow}     % implies -->

\newcommand{\lla}{\longleftarrow}  % implies <-----
\newcommand{\lar}{\Leftarrow}      % implies <--

\newcommand{\pr}{^\prime} 		   % prime (i.e. R')

\newcommand{\bnm}[2]{\mt{#1\setminus{#2}}}
\newcommand{\bnt}[2]{\mt{\textrm{#1}\setminus{\textrm{#2}}}}
\newcommand{\bi}{\bnm{\mathbb{R}}{\mathbb{Q}}}

\newcommand{\nbho}[3]{\textrm{N(}#1, #2\textrm{) }\cap \textrm{ #3} \neq \emptyset}
     							   %  N(x, eps) intersect S \= emptyset
\newcommand{\nbhe}[3]{\textrm{N(}#1, #2\textrm{) }\cap \textrm{ #3} = \emptyset}
     							   %  N(x, eps) intersect S  = emptyset
\newcommand{\dnbho}[3]{\textrm{N*(}#1, #2\textrm{) }\cap \textrm{ #3} \neq \emptyset}
     							   %  N*(x, eps) intersect S \= emptyset
\newcommand{\dnbhe}[3]{\textrm{N*(}#1, #2\textrm{) }\cap \textrm{ #3} = \emptyset}
     							   %  N*(x, eps) intersect S = emptyset
     							 

\begin{document}

Assignment Set: {1, 2, 3, 4, 6, 8} from pages 148 - 149

\bsc{1)}{
Mark each statement as true or false. Justify each answer.

\balist
\item A set S is compact iff every open cover of S contains a finite subcover.
	
	True.
	
	The original definition is:
	
	A set S \sbs \br is said to be compact if every open cover has a finite subcover.
	
	\lra (every open cover \rar compact)

	\lt{S \sbs \br be a set st every open cover has a finite subcover.}
	
	Then, by definition, S is compact.
	
	\lla (every open cover \mt{\Leftarrow} compact)
	
	\lt{S \sbs \br be compact}
	
	Then, by definition, S is a set st every open cover has a finite subcover.
	
	Hence, result.

	
\item Every finite set is compact.

	True.
	
	\lt{S be a finite set, 0 $<$ $|$S$|$ $< \infty$, S \eql \{\uw{s}{1}, \uw{s}{2}, ... \uw{s}{n}\}}
	
	\lt{l \eql min \{\uw{s}{1}, \uw{s}{2}, ... \uw{s}{n}\}, u \eql max \{\uw{s}{1}, \uw{s}{2}, ... \uw{s}{n}\}}
	
	Since S has lower bound l and upper bound u, S is bounded.
	
	Since l, u \mem bd S by definition of maximum and minimum, S is closed.
	
	Hence, by Heine-Borel, S is compact. 

\item No infinite set is compact.
	
	Not true.
	
	If S \eql [0, 1], there are infinite values between 0 and 1.
	
	However, [0, 1] is compact.
\item If a set is compact, then it has a maximum and a minimum.
	
	True.
	
	\lt{S $\neq$ \es be a compact set}
	
	Since S is compact, S is also closed and bounded.
	
	Since S $\neq$ \es, there is at least one element.
	
	Since S is closed, bd S \sbs S.
	
	Since there is at least one element, and all boundary points of S are in S, there is a maximum and minimum element.
	
\item If a set has a maximum and a minimum, then it is compact.
	
	\lt{S $\neq$ \es, l \eql min S, u \eql max S}
	
	Since l, u \mem bd S by definition of minimum and maximum, S is bounded.
	
	Notice that l, u $\not\in$ int S.
	
	Since \exs s \mem S st s $\not\in$ int S, S is not open.
	
	Therefore, S is closed.
	
	By the Heine-Borel theorem, since S is closed and bounded, S is compact.
	
	
\elist

}

\bsc{2)}{
Mark each statement as true or false. Justify each answer.

\balist
\item Some unbounded sets are compact.

	False.
	
	By Heine-Borel, S is compact only if closed and bounded.
	
\item If S \sbs \br is compact, then \exs x \mem \br st s \mem S\pr
	
	False.
	
	The empty set is compact and contains no elements.
	
\item If S is compact and s \mem S\pr, then s \mem S.
	
	True.
	
	By Heine-Borel, if S is compact, then S is closed and bounded.
	
	If S is closed, then S\pr \sbs S by Theorem 3.4.17
	
	So, if s \mem S\pr, then s \mem S
	
\item If S is unbounded, then S has at least one accumulation point.
	
	False.
	
	\bn is a counter example.
\item \lt{F \eql \{\uw{A}{i}, i \mem \bn\}}. Suppose that the intersection of any finite subfamily of F is nonempty. If $\cap$ F \eql \es, then, for some k \mem \bn, \uw{A}{k} is not compact.
	
	True.
	
	\supp{\fa \uw{A}{i} \mem F, \uw{A}{i} is compact}
	
	Then, by Heine-Borel, \uw{A}{i} is closed and bounded.
	
	If any finite subfamily of F is nonempty, but $\cap$ F \eql \es, then \lmti{i} F \eql \es, so F gets smaller as i grows.
	
	However, as F gets smaller, since \uw{A}{i} is closed, bd \uw{A}{i} \sbs \uw{A}{i}, which means eventually F should converge to a set with a single element in it.
	
	But this is not true, so \exs \uw{A}{i} \mem F st \uw{A}{i} is not compact.
	
\elist


}

\bsc{3)}{
Show that each subset of \br is not compact by describing an open cover for it that has no finite subcover.

\balist
\item `[1, 3) 
	
	\{n \mem \bn : \urng{i = 1}{n} (0, 2 + $\sum_{k = 1}^i$ \frc{1}{2^k} )\}
	
	Simpler solution: \urng{n = 1}{\infty} (0, 3 $-$ \frc{1}{n})
\item `[1,2)
	
	\{n \mem \bn : \urng{i = 1}{n} ((2 $-$ $\sum_{i = 1}^n$ \frc{1}{2^i}), (3 $+$ $\sum_{i = 1}^n$ \frc{1}{2^i}))\}
	
	Simpler solution: \urng{n = 1}{\infty} (0, 2 $-$ \frc{1}{n})
\item \bn
	
	\{\urng{n = 1}{\infty} (0, n)\}
	
	Other solution: \urng{n = 1}{\infty} (n $-$ \frc{1}{2}, n $+$ \frc{1}{2})
\item \{\frc{1}{n} : n \mem \bn\} (Closed in \br, but not closed in \bq)
	
	\{ n \mem \bn: \urng{i = 1}{n} (0, $\sum_{k = 1}^i$ \frc{1}{2^k} )\}
	
	Simpler solution: \urng{n = 1}{\infty} (0, 1 $-$ \frc{1}{n})
\item \{x \mem \bq : 0 $\leq$ x $\leq$ 2\} 
	
	but wait, isn't this a closed and bounded set? It's obviously bounded, so why is that not closed?
	
	Because the set only contains rational numbers and there are some irrational numbers that are accumulation points. A set has to contain all of its accumulation points to be closed.
	
	In any case, here's a cover:
	
	\urng{i = 0}{\infty} ($-$1, $\sum_{k = 0}^i$ \nck{2k}{k} $\frac{1}{8^k}$) $\cup$ ($\sqrt{2}$ $-$ ($\sqrt{2}$ $-$ $\sum_{k = 0}^i$ \nck{2k}{k} $\frac{1}{8^k}$), 3)
	
	Other solution: \urng{n = 1}{\infty} [($-$1, $\sqrt{2}$), $\cup$ ($\sqrt{2}$ $+$ \frc{1}{n}, 3)]
	
\elist

}

\bsc{4)}{
Prove that the intersection of any collection of compact sets is compact.

\lt{S be \nrng{\afa \mem I}{} \uw{G}{\afa} where I is an index set and all \uw{G}{\afa}'s are compact}

\lt{S be nonempty (since if it's empty, then it's compact anyhow)}


By Heine-Borel, \uw{G}{\afa} is both closed and bounded \fa \afa.

\lt{U be the set of all least upper bounds from each \uw{G}{\afa}, and L be the set of all greatest lower bounds from each \uw{G}{\afa}'s}

Since \uw{G}{\afa} is closed \fa \uw{G}{\afa}, each element in U is a max, and each element in L is a min.

Since S is an intersection, its minimum will be max L, and its maximum will be min U, which we know exists because S $\neq$ \es

Since S has a min and a max, the min and max are members of its boundary set.

Since the min and max are members of the boundary set, they can't be interior points.

Since \exs s \mem S that isn't an interior point, S is not open and therefore closed.

Since S has a minimum and a maximum, it is also bounded.

Since S is closed and bounded, it is compact by Heine-Borel.
}

\bsc{6)}{
Show that compactedness is necessary in Corollary 3.5.8. That is, find a family of intervals \{\uw{A}{n} : n \mem \bn\} with \uw{A}{n + 1} \sbs \uw{A}{n} \fa n, \urng{n = 1}{\infty} \uw{A}{n} \eql \es, and such that:

\balist
\item The sets \uw{A}{n} are all closed. 
	
	\{n \mem \bn : \es\}
\item The sets \uw{A}{n} are all bounded.
	
	\{n \mem \bn : (0, 0)\}
\elist

}

\bsc{8)}{
If S \sbs \br is compact and T \sbs S is closed, then T is compact.

\balist
\item Prove this using the definition of compactness.
	
	If S is compact, then \fa \urng{\afa \mem I}{} \uw{G}{\afa} that cover S (\uw{G}{\afa} is open), \exs \urng{i = 1}{n} \dbs{G}{\afa}{i} - a finite subcover of S.
	
	\lt{CV be an arbitrary open cover for S, SCV be CV's finite subcover for S}
	
	By definition, S \sbs CV and S \sbs SCV
	
	Since T \sbs S, T \sbs CV and T \sbs SCV
	
	So, \fa CV of S and \fa SCV of S, CV and SCV are covers and finite subcovers for T.
	
	 So, T is compact.
	
	
	
	
\item Prove using the Heine-Borel theorem: If S \sbs \br is compact and T \sbs S is closed, then T is compact.

	Since S is compact, by Heine-Borel, S is closed and bounded.
	
	\lt{T $\neq$ \es} (since if T \eql \es then it is compact anyhow)
	
	Since T is closed, bd T \sbs T.
	
	Since T $\neq$ \es and bd T \sbs T, T contains a maximum and a minimum.
	
	Therefore, T is bounded.
	
	Since T is both closed and bounded, T is compact.
	
	
\elist

}
\end{document}