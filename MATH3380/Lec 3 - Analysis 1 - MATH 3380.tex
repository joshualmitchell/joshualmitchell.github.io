% Thank you Josh Davis for this template!
% https://github.com/jdavis/latex-homework-template/blob/master/homework.tex

\documentclass{article}

\newcommand{\hmwkTitle}{Lec\ \#3}

% Packages

\usepackage{fancyhdr}
\usepackage{extramarks}
\usepackage{amsmath}
\usepackage{amssymb}
\usepackage{amsthm}
\usepackage{amsfonts}
\usepackage{tikz}
\usepackage[plain]{algorithm}
\usepackage{algpseudocode}
\usepackage{enumitem}
\usepackage{chngcntr}

% Libraries

\usetikzlibrary{automata, positioning, arrows}

%
% Basic Document Settings
%

\topmargin=-0.45in
\evensidemargin=0in
\oddsidemargin=0in
\textwidth=6.5in
\textheight=9.0in
\headsep=0.25in

\linespread{1.1}

\pagestyle{fancy}
\lhead{\hmwkAuthorName}
\chead{}
\rhead{\hmwkClass\ (\hmwkClassInstructor): \hmwkTitle}
\lfoot{\lastxmark}
\cfoot{\thepage}

\renewcommand\headrulewidth{0.4pt}
\renewcommand\footrulewidth{0.4pt}

\setlength\parindent{0pt}
\setcounter{secnumdepth}{0}

\newcommand{\hmwkClass}{MATH 3380 / Analysis 1}        % Class
\newcommand{\hmwkClassInstructor}{Dr. Welsh}           % Instructor
\newcommand{\hmwkAuthorName}{\textbf{Joshua Mitchell}} % Author

%
% Title Page
%

\title{
    \vspace{2in}
    \textmd{\textbf{\hmwkClass:\ \hmwkTitle}}\\
    \normalsize\vspace{0.1in}\small\vspace{0.1in}\large{\textit{\hmwkClassInstructor}}
    \vspace{3in}
}

\author{\hmwkAuthorName}
\date{}

\renewcommand{\part}[1]{\textbf{\large Part \Alph{partCounter}}\stepcounter{partCounter}\\}

% Integral dx
\newcommand{\dx}{\mathrm{d}x}

%
% Various Helper Commands
%

% For derivatives
\newcommand{\deriv}[1]{\frac{\mathrm{d}}{\mathrm{d}x} (#1)}

% For partial derivatives
\newcommand{\pderiv}[2]{\frac{\partial}{\partial #1} (#2)}


% Alias for the Solution section header
\newcommand{\solution}{\textbf{\large Solution}}

% Probability commands: Expectation, Variance, Covariance, Bias
\newcommand{\E}{\mathrm{E}}
\newcommand{\Var}{\mathrm{Var}}
\newcommand{\Cov}{\mathrm{Cov}}
\newcommand{\Bias}{\mathrm{Bias}}

% Formatting commands:

\newcommand{\mt}[1]{\ensuremath{#1}}
\newcommand{\nm}[1]{\textrm{#1}}

\newcommand\bsc[2][\DefaultOpt]{%
  \def\DefaultOpt{#2}%
  \section[#1]{#2}%
}
\newcommand\ssc[2][\DefaultOpt]{%
  \def\DefaultOpt{#2}%
  \subsection[#1]{#2}%
}
\newcommand{\bgpf}{\begin{proof} $ $\newline}

\newcommand{\bgeq}{\begin{equation*}}
\newcommand{\eeq}{\end{equation*}}	

\newcommand{\balist}{\begin{enumerate}[label=\alph*.]}
\newcommand{\elist}{\end{enumerate}}

\newcommand{\bilist}{\begin{enumerate}[label=\roman*)]}	

\newcommand{\bgsp}{\begin{split}}
% \newcommand{\esp}{\end{split}} % doesn't work for some reason.

\newcommand\prs[1]{~~~\textbf{(#1)}}

\newcommand{\lt}[1]{\textbf{Let: } #1}
     							   %  if you're setting it to be true
\newcommand{\supp}[1]{\textbf{Suppose: } #1}
     							   %  Suppose (if it'll end up false)
\newcommand{\wts}[1]{\textbf{Want to show: } #1}
     							   %  Want to show
\newcommand{\as}[1]{\textbf{Assume: } #1}
     							   %  if you think it follows from truth
\newcommand{\bpth}[1]{\textbf{(#1)}}

\newcommand{\step}[2]{\begin{equation}\tag{#2}#1\end{equation}}
\newcommand{\epf}{\end{proof}}

\newcommand{\sidenote}[1]{-----------------------------------------------------------------Side Notes---------------------------------------------------------------
#1 \

---------------------------------------------------------------------------------------------------------------------------------------------}

% Analysis / Logical commands:

\newcommand{\br}{\mathbb{R}}       % |R
\newcommand{\bq}{\mathbb{Q}}       % |Q
\newcommand{\bn}{\mathbb{N}}       % |N
\newcommand{\bc}{\mathbb{C}}       % |C
\newcommand{\bz}{\mathbb{Z}}       % |Z

\newcommand{\ep}{\epsilon}         % epsilon
\newcommand{\fa}{\forall}          % for all

\newcommand{\es}{\emptyset}        % empty set
\newcommand{\sbs}{\subset}         % subset of

\newcommand{\lra}{\longrightarrow} % implies ----->
\newcommand{\rar}{\Rightarrow}     % implies -->

\newcommand{\lla}{\longleftarrow}  % implies <-----
\newcommand{\lar}{\Leftarrow}      % implies <--

\newcommand{\pr}{^\prime} 		   % prime (i.e. R')

\newcommand{\bnm}[2]{\mt{#1\setminus{#2}}}
\newcommand{\bnt}[2]{\mt{\textrm{#1}\setminus{\textrm{#2}}}}
\newcommand{\bi}{\bnm{\mathbb{R}}{\mathbb{Q}}}

\newcommand{\nbho}[3]{\textrm{N(}#1, #2\textrm{) }\cap \textrm{ #3} \neq \emptyset}
     							   %  N(x, eps) intersect S \= emptyset
\newcommand{\nbhe}[3]{\textrm{N(}#1, #2\textrm{) }\cap \textrm{ #3} = \emptyset}
     							   %  N(x, eps) intersect S  = emptyset
\newcommand{\dnbho}[3]{\textrm{N*(}#1, #2\textrm{) }\cap \textrm{ #3} \neq \emptyset}
     							   %  N*(x, eps) intersect S \= emptyset
\newcommand{\dnbhe}[3]{\textrm{N*(}#1, #2\textrm{) }\cap \textrm{ #3} = \emptyset}
     							   %  N*(x, eps) intersect S = emptyset
     							 

\begin{document}

\bsc{Theorem 1 (infinum definition)}{

\lt{\es $\neq$ S \sbs \br, S is bounded below.}

Then S possesses a greatest lower bound denoted by \textbf{inf S} (the \textbf{infinum} of S), where inf S \mem \br, satisfying:

\bilist
\item inf S $\leq$ s \fa s \mem S
\item \fa \ep $>$ 0, \exs \uw{s}{1}  st  inf S $+$ \ep $>$ \uw{s}{1}
\elist

\bgpf

\lt{S be bounded below}

Then 

\step{\exs m \mem \br \textrm{  st  }m \leq s \textrm{  }\fa s \mem S}{1}

Define the set -S to be \{$-s$ : s \mem S\}

So, \es $\neq$ -S \sbs \br

From \bpth{1}, -S $\leq$ $-$m \fa s \mem S.

Thus, $-$m is an upper bound for -S.

By the Axiom of Completeness of \br (AoC), sup(-S) \mem \br exists.

By definition,

\step{-s \leq sup(-S), \textrm{  } \fa s \textrm{ } \mem S}{2}

and \fa \ep $>$ 0, \exs \uw{-s}{1} \mem S  st

\step{sup(-S) - \ep < \uw{-s}{1} \textrm{  where  } \uw{s}{1} \mem S}{3}

From \bpth{2}, 

\step{-sup(-S) \leq S \textrm{  } \fa s \mem S}{4}

\wts{$-$sup(-S) \eql inf S}

From \bpth{3},

\step{-sup(-S) + \ep > \uw{s}{1} \textrm{  where  } \uw{s}{1} \mem S }{5}

We see that from \bpth{4} and \bpth{5},

inf S \eql $-$sup(-S).

Hence, result.
\epf

}

\newpage

\bsc{Theorem 3.3.7}{

Given nonempty subsets of A, B (A, B \sbs \br),

\lt{C \eql \{x $+$ y: x \mem A, y \mem B\}}

If A and B have suprema, then C has a supremum: sup C \eql sup A $+$ sup B

\bgpf

\lt{c \mem C}

Then c \eql x $+$ y for some x \mem A, y \mem B.

It follows that:

x $\leq$ sup A, y $\leq$ sup B

x $+$ y $\leq$ sup A $+$ sup B

c $\leq$ sup A $+$ sup B

\step{c \leq sup\textrm{ }A + sup\textrm{ }B \textrm{ } \fa\textrm{ }c \mem C}{1}

By the AoC, sup C \mem \br exists.

For \ep $>$ 0, \exs \uw{x}{0} \mem A, \uw{y}{0} \mem B st

\step{sup \textrm{  } A - \frac{\ep}{2} < \uw{x}{0}}{2}
\step{sup \textrm{  } B - \frac{\ep}{2} < \uw{y}{0}}{3}

From \bpth{2} and \bpth{3}, 

sup A - \frc{\ep}{2} + sup B - \frc{\ep}{2} $<$ \uw{x}{0} $+$ \uw{y}{0} = \uw{c}{0} \mem C

So, 

\step{sup\textrm{ }A + sup\textrm{ }B - \ep < \uw{c}{0}}{4}

From \bpth{1} and \bpth{4}, sup C \eql sup A $+$ sup B

Hence, result.

\epf

}

\newpage

\bsc{Theorem 3.3.8}{

Suppose \es $\neq$ D \sbs \br and 

f : D \lra \br

g : D \lra \br

f(D) \eql \{f(x) : x \mem D\}

If \fa x, y \mem D, f(x) $\leq$ g(y), then
 
f(D) is bounded above and g(D) is bounded below.

Furthermore, sup(f(D)) $\leq$ inf(g(D))

\bgpf

\lt{\uw{y}{0} \mem D}

Then f(x) $\leq$ g(\uw{y}{0}) \fa x \mem D

So, f(D) is bounded above by g(\uw{y}{0}).

By the AoC, sup(f(D)) exists and sup(f(D)) $\leq$ g(\uw{y}{0}) \

\

Since \uw{y}{0} \mem D was arbitrary, we see that

sup(f(D)) $\leq$ g(y) \fa y \mem D

Thus, sup(f(D)) is a lower bound for g(D)

g(D) \eql \{g(y) : y \mem D\}

Hence, inf(g(D)) \mem \br exists and

sup(f(D)) $\leq$ inf(g(D))

Hence, result.

\epf
}

\bsc{Theorem 3.3.9: Archimedian Property / Principle of \br (AP)}{

The set \bn \eql \{1, 2, 3...\} is unbounded above in \br

\bgpf

\supp{\bn is bounded above.}

By the AoC, sup \bn \mem \br exists.

So,

\bilist
\item n $\leq$ sup \bn  \fa n \mem \bn \bpth{1}
\item \fa \ep $>$ 0, \exs n \mem \bn st sup \bn $-$ \ep $<$ \uw{n}{0} \bpth{2}
\elist

Using \bpth{2} with \ep \eql 1, \exs \uw{n}{0}(1) \mem \bn st sup \bn $-$ \ep $<$ \uw{n}{0}

Then, sup \bn $<$ 1 $+$ \uw{n}{0} \bpth{3}

See that \bpth{3} contradicts \bpth{1} with n \eql 1 + \uw{n}{0} \mem \bn

By contradiction, \bn is unbounded.

\epf

}

\end{document}