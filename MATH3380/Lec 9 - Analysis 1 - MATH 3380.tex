% Thank you Josh Davis for this template!
% https://github.com/jdavis/latex-homework-template/blob/master/homework.tex

\documentclass{article}

\newcommand{\hmwkTitle}{Lec\ \#9}

% % Packages

\usepackage{fancyhdr}
\usepackage{extramarks}
\usepackage{amsmath}
\usepackage{amssymb}
\usepackage{amsthm}
\usepackage{amsfonts}
\usepackage{tikz}
\usepackage[plain]{algorithm}
\usepackage{algpseudocode}
\usepackage{enumitem}
\usepackage{chngcntr}

% Libraries

\usetikzlibrary{automata, positioning, arrows}

%
% Basic Document Settings
%

\topmargin=-0.45in
\evensidemargin=0in
\oddsidemargin=0in
\textwidth=6.5in
\textheight=9.0in
\headsep=0.25in

\linespread{1.1}

\pagestyle{fancy}
\lhead{\hmwkAuthorName}
\chead{}
\rhead{\hmwkClass\ (\hmwkClassInstructor): \hmwkTitle}
\lfoot{\lastxmark}
\cfoot{\thepage}

\renewcommand\headrulewidth{0.4pt}
\renewcommand\footrulewidth{0.4pt}

\setlength\parindent{0pt}
\setcounter{secnumdepth}{0}

\newcommand{\hmwkClass}{MATH 3380 / Analysis 1}        % Class
\newcommand{\hmwkClassInstructor}{Dr. Welsh}           % Instructor
\newcommand{\hmwkAuthorName}{\textbf{Joshua Mitchell}} % Author

%
% Title Page
%

\title{
    \vspace{2in}
    \textmd{\textbf{\hmwkClass:\ \hmwkTitle}}\\
    \normalsize\vspace{0.1in}\small\vspace{0.1in}\large{\textit{\hmwkClassInstructor}}
    \vspace{3in}
}

\author{\hmwkAuthorName}
\date{}

\renewcommand{\part}[1]{\textbf{\large Part \Alph{partCounter}}\stepcounter{partCounter}\\}

% Integral dx
\newcommand{\dx}{\mathrm{d}x}

%
% Various Helper Commands
%

% For derivatives
\newcommand{\deriv}[1]{\frac{\mathrm{d}}{\mathrm{d}x} (#1)}

% For partial derivatives
\newcommand{\pderiv}[2]{\frac{\partial}{\partial #1} (#2)}


% Alias for the Solution section header
\newcommand{\solution}{\textbf{\large Solution}}

% Probability commands: Expectation, Variance, Covariance, Bias
\newcommand{\E}{\mathrm{E}}
\newcommand{\Var}{\mathrm{Var}}
\newcommand{\Cov}{\mathrm{Cov}}
\newcommand{\Bias}{\mathrm{Bias}}

% Formatting commands:

\newcommand{\mt}[1]{\ensuremath{#1}}
\newcommand{\nm}[1]{\textrm{#1}}

\newcommand\bsc[2][\DefaultOpt]{%
  \def\DefaultOpt{#2}%
  \section[#1]{#2}%
}
\newcommand\ssc[2][\DefaultOpt]{%
  \def\DefaultOpt{#2}%
  \subsection[#1]{#2}%
}
\newcommand{\bgpf}{\begin{proof} $ $\newline}

\newcommand{\bgeq}{\begin{equation*}}
\newcommand{\eeq}{\end{equation*}}	

\newcommand{\balist}{\begin{enumerate}[label=\alph*.]}
\newcommand{\elist}{\end{enumerate}}

\newcommand{\bilist}{\begin{enumerate}[label=\roman*)]}	

\newcommand{\bgsp}{\begin{split}}
% \newcommand{\esp}{\end{split}} % doesn't work for some reason.

\newcommand\prs[1]{~~~\textbf{(#1)}}

\newcommand{\lt}[1]{\textbf{Let: } #1}
     							   %  if you're setting it to be true
\newcommand{\supp}[1]{\textbf{Suppose: } #1}
     							   %  Suppose (if it'll end up false)
\newcommand{\wts}[1]{\textbf{Want to show: } #1}
     							   %  Want to show
\newcommand{\as}[1]{\textbf{Assume: } #1}
     							   %  if you think it follows from truth
\newcommand{\bpth}[1]{\textbf{(#1)}}

\newcommand{\step}[2]{\begin{equation}\tag{#2}#1\end{equation}}
\newcommand{\epf}{\end{proof}}

\newcommand{\sidenote}[1]{-----------------------------------------------------------------Side Notes---------------------------------------------------------------
#1 \

---------------------------------------------------------------------------------------------------------------------------------------------}

% Analysis / Logical commands:

\newcommand{\br}{\mathbb{R}}       % |R
\newcommand{\bq}{\mathbb{Q}}       % |Q
\newcommand{\bn}{\mathbb{N}}       % |N
\newcommand{\bc}{\mathbb{C}}       % |C
\newcommand{\bz}{\mathbb{Z}}       % |Z

\newcommand{\ep}{\epsilon}         % epsilon
\newcommand{\fa}{\forall}          % for all

\newcommand{\es}{\emptyset}        % empty set
\newcommand{\sbs}{\subset}         % subset of

\newcommand{\lra}{\longrightarrow} % implies ----->
\newcommand{\rar}{\Rightarrow}     % implies -->

\newcommand{\lla}{\longleftarrow}  % implies <-----
\newcommand{\lar}{\Leftarrow}      % implies <--

\newcommand{\pr}{^\prime} 		   % prime (i.e. R')

\newcommand{\bnm}[2]{\mt{#1\setminus{#2}}}
\newcommand{\bnt}[2]{\mt{\textrm{#1}\setminus{\textrm{#2}}}}
\newcommand{\bi}{\bnm{\mathbb{R}}{\mathbb{Q}}}

\newcommand{\nbho}[3]{\textrm{N(}#1, #2\textrm{) }\cap \textrm{ #3} \neq \emptyset}
     							   %  N(x, eps) intersect S \= emptyset
\newcommand{\nbhe}[3]{\textrm{N(}#1, #2\textrm{) }\cap \textrm{ #3} = \emptyset}
     							   %  N(x, eps) intersect S  = emptyset
\newcommand{\dnbho}[3]{\textrm{N*(}#1, #2\textrm{) }\cap \textrm{ #3} \neq \emptyset}
     							   %  N*(x, eps) intersect S \= emptyset
\newcommand{\dnbhe}[3]{\textrm{N*(}#1, #2\textrm{) }\cap \textrm{ #3} = \emptyset}
     							   %  N*(x, eps) intersect S = emptyset
     							 

% ----------

% Packages

\usepackage{fancyhdr}
\usepackage{extramarks}
\usepackage{amsmath}
\usepackage{amssymb}
\usepackage{amsthm}
\usepackage{amsfonts}
\usepackage{tikz}
\usepackage[plain]{algorithm}
\usepackage{algpseudocode}
\usepackage{enumitem}
\usepackage{chngcntr}

% Libraries

\usetikzlibrary{automata, positioning, arrows}

%
% Basic Document Settings
%

\topmargin=-0.45in
\evensidemargin=0in
\oddsidemargin=0in
\textwidth=6.5in
\textheight=9.0in
\headsep=0.25in

\linespread{1.1}

\pagestyle{fancy}
\lhead{\hmwkAuthorName}
\chead{}
\rhead{\hmwkClass\ (\hmwkClassInstructor): \hmwkTitle}
\lfoot{\lastxmark}
\cfoot{\thepage}

\renewcommand\headrulewidth{0.4pt}
\renewcommand\footrulewidth{0.4pt}

\setlength\parindent{0pt}
\setcounter{secnumdepth}{0}

\newcommand{\hmwkClass}{MATH 3380 / Analysis 1}        % Class
\newcommand{\hmwkClassInstructor}{Dr. Welsh}           % Instructor
\newcommand{\hmwkAuthorName}{\textbf{Joshua Mitchell}} % Author

%
% Title Page
%

\title{
    \vspace{2in}
    \textmd{\textbf{\hmwkClass:\ \hmwkTitle}}\\
    \normalsize\vspace{0.1in}\small\vspace{0.1in}\large{\textit{\hmwkClassInstructor}}
    \vspace{3in}
}

\author{\hmwkAuthorName}
\date{}

\renewcommand{\part}[1]{\textbf{\large Part \Alph{partCounter}}\stepcounter{partCounter}\\}

% Integral dx
\newcommand{\dx}{\mathrm{d}x}

%
% Various Helper Commands
%

% For derivatives
\newcommand{\deriv}[1]{\frac{\mathrm{d}}{\mathrm{d}x} (#1)}

% For partial derivatives
\newcommand{\pderiv}[2]{\frac{\partial}{\partial #1} (#2)}


% Alias for the Solution section header
\newcommand{\solution}{\textbf{\large Solution}}

% Probability commands: Expectation, Variance, Covariance, Bias
\newcommand{\E}{\mathrm{E}}
\newcommand{\Var}{\mathrm{Var}}
\newcommand{\Cov}{\mathrm{Cov}}
\newcommand{\Bias}{\mathrm{Bias}}

% Formatting commands:

\newcommand{\mt}[1]{\ensuremath{#1}}
\newcommand{\nm}[1]{\textrm{#1}}

\newcommand\bsc[2][\DefaultOpt]{%
  \def\DefaultOpt{#2}%
  \section[#1]{#2}%
}
\newcommand\ssc[2][\DefaultOpt]{%
  \def\DefaultOpt{#2}%
  \subsection[#1]{#2}%
}
\newcommand{\bgpf}{\begin{proof} $ $\newline}

\newcommand{\bgeq}{\begin{equation*}}
\newcommand{\eeq}{\end{equation*}}	

\newcommand{\balist}{\begin{enumerate}[label=\alph*.]}
\newcommand{\elist}{\end{enumerate}}

\newcommand{\bilist}{\begin{enumerate}[label=\roman*)]}	

\newcommand{\bgsp}{\begin{split}}
% \newcommand{\esp}{\end{split}} % doesn't work for some reason.

\newcommand\prs[1]{~~~\textbf{(#1)}}

\newcommand{\lt}[1]{\textbf{Let: } #1}
     							   %  if you're setting it to be true
\newcommand{\supp}[1]{\textbf{Suppose: } #1}
     							   %  Suppose (if it'll end up false)
\newcommand{\wts}[1]{\textbf{Want to show: } #1}
     							   %  Want to show
\newcommand{\as}[1]{\textbf{Assume: } #1}
     							   %  if you think it follows from truth
\newcommand{\bpth}[1]{\textbf{(#1)}}

\newcommand{\step}[2]{\begin{equation}\tag{#2}#1\end{equation}}
\newcommand{\epf}{\end{proof}}

\newcommand{\dbs}[3]{\mt{#1_{#2_#3}}}

\newcommand{\sidenote}[1]{-----------------------------------------------------------------Side Note----------------------------------------------------------------
#1 \

---------------------------------------------------------------------------------------------------------------------------------------------}

% Analysis / Logical commands:

\newcommand{\br}{\mt{\mathbb{R}} }       % |R
\newcommand{\bq}{\mt{\mathbb{Q}} }       % |Q
\newcommand{\bn}{\mt{\mathbb{N}} }       % |N
\newcommand{\bc}{\mt{\mathbb{C}} }       % |C
\newcommand{\bz}{\mt{\mathbb{Z}} }       % |Z

\newcommand{\ep}{\mt{\epsilon} }         % epsilon
\newcommand{\fa}{\mt{\forall} }          % for all
\newcommand{\afa}{\mt{\alpha} }
\newcommand{\bta}{\mt{\beta} }
\newcommand{\mem}{\mt{\in} }
\newcommand{\exs}{\mt{\exists} }

\newcommand{\es}{\mt{\emptyset} }        % empty set
\newcommand{\sbs}{\mt{\subset} }         % subset of
\newcommand{\fs}[2]{\{\uw{#1}{1}, \uw{#1}{2}, ... \uw{#1}{#2}\}}

\newcommand{\lra}{ \mt{\longrightarrow} } % implies ----->
\newcommand{\rar}{ \mt{\Rightarrow} }     % implies -->

\newcommand{\lla}{ \mt{\longleftarrow} }  % implies <-----
\newcommand{\lar}{ \mt{\Leftarrow} }      % implies <--

\newcommand{\eql}{\mt{=} }
\newcommand{\pr}{\mt{^\prime} } 		   % prime (i.e. R')
\newcommand{\uw}[2]{#1\mt{_{#2}}}
\newcommand{\frc}[2]{\mt{\frac{#1}{#2}}}

\newcommand{\bnm}[2]{\mt{#1\setminus{#2}}}
\newcommand{\bnt}[2]{\mt{\textrm{#1}\setminus{\textrm{#2}}}}
\newcommand{\bi}{\bnm{\mathbb{R}}{\mathbb{Q}}}

\newcommand{\urng}[2]{\mt{\bigcup_{#1}^{#2}}}
\newcommand{\nrng}[2]{\mt{\bigcap_{#1}^{#2}}}

\newcommand{\nbho}[3]{\textrm{N(}#1, #2\textrm{) }\cap \textrm{ #3} \neq \emptyset}
     							   %  N(x, eps) intersect S \= emptyset
\newcommand{\nbhe}[3]{\textrm{N(}#1, #2\textrm{) }\cap \textrm{ #3} = \emptyset}
     							   %  N(x, eps) intersect S  = emptyset
\newcommand{\dnbho}[3]{\textrm{N*(}#1, #2\textrm{) }\cap \textrm{ #3} \neq \emptyset}
     							   %  N*(x, eps) intersect S \= emptyset
\newcommand{\dnbhe}[3]{\textrm{N*(}#1, #2\textrm{) }\cap \textrm{ #3} = \emptyset}
     							   %  N*(x, eps) intersect S = emptyset
     							 


% ----------

\begin{document}

\bsc{Ch 4: Sequences}{}

\bsc{4.1: Convergence}{

\ssc{Definition 1: Sequence}{

A \textbf{sequence} is a function S: \bn \lra \br

We write S(n) \eql \uw{S}{n} \fa n \mem \bn and refer to \{\uw{S}{n}\} (the book uses (\uw{S}{n})) as the \textbf{sequence}.

We refer to the set \{ \uw{S}{n} : n \mem \bn\} as the range of the sequence.

\sidenote{
\uw{S}{n} \eql ($-$1)$^n$ \fa n \mem \bn

\{($-$1)$^n$\}

range\{\uw{S}{n}\} \eql \{$-$1, 1\}

Here \{\uw{S}{n}\} \eql \{1, $-$1, 1, $-$1...\}
}
An alternative to writing \{\uw{S}{n}\} for a sequence is to list the elements: \uw{S}{1}, \uw{S}{2}, ... \uw{S}{n}

Sometimes the domain of the sequence is \bn $\cup$ \{0\} or \{n \mem \bn : n $\geq$ m\} for some m \mem \bn.

In this case, we write \{\uw{S}{n}\}$^\infty_{n = 0}$ or \{\uw{S}{n}\}$^\infty_{n = m}$

\textbf{Note 1}: A denumerable set (or a countably infinite set) S is a set for which there is a bijection S: \bn \lra \br

This bijection may be thought of as a sequence \{\uw{S}{n}\}, where \uw{S}{n} \eql S(n) \fa n \mem \bn of distinct terms.

\ssc{Definition 4.1.2}{

A sequence \{\uw{S}{n}\} is said to \textbf{converge} to s \mem \br provided that \fa \ep $>$ 0, \exs N(\ep) \mem \bn st

$|\uw{S}{n} - S|$ $<$ \ep \textrm{   } \fa n $\geq$ N

\sidenote{
-|-|-(-|-)-|-|-|-|--

s6, s5, sminusep, S / Sn, splusep, s4, s3, s2, s1
}

We call s the \textbf{limit} of the sequence and write:

lim$_{n\lra\infty}$ \uw{S}{n} \eql s  or  lim \uw{S}{n}   or \uw{S}{n} \lra s as n \lra $\infty$.

If a sequence does not converge, then it is said to diverge.

\ssc{Example 4.1.3}{

Show that the sequence \{\uw{S}{n}\}, where \uw{S}{n} \eql \frc{1}{n} \fa n \mem \bn, (\{\uw{S}{n}\}) converges to 0.

\bgpf
\wts{$|$\frc{1}{n} $-$ 0$|$ $<$ \ep for sufficiently large values of n}

Now: 

\step{|\frc{1}{n} - 0| = \frc{1}{n}}{1}

Since \frc{1}{n} $<$ \ep implies n $>$ \frc{1}{\ep},

By the AP (Theorem 3.3.10),

\exs N \mem \bn st N $>$ \frc{1}{\ep}

Thus, 

\frc{1}{N} $<$ \ep and \frc{1}{n} $\leq$ \frc{1}{N} $\leq$ \ep, \fa n $\geq$ N.

From \bpth{1}, $|$\frc{1}{n} $-$ 0$|$ $<$ \ep,   \fa n $\geq$ N

\newpage

[Let N \mem \bn satisfy N $>$ \frc{1}{\ep}.

Then \fa n $\geq$ N, $|$\frc{1}{n} $-$ 0$|$ \eql \frc{1}{n} $<$ \ep]

\epf

}

\ssc{Example 4.1.4}{

Prove that for \{$\frac{1}{\sqrt{n}}$\}, the limit is 0.

\bgpf

\lt{\ep $>$ 0}

Then:

$|$$\frac{1}{\sqrt{n}}$ $-$ 0$|$ \eql $\frac{1}{\sqrt{n}}$ \fa n \mem \bn \bpth{1} \

\

$\frac{1}{\sqrt{n}}$ $<$ \ep

\frc{1}{n} $<$ \ep$^2$

n $>$ $\frac{1}{\epsilon^2}$

By Theorem 3.3.10 a), 

% \exs N \mem \bn st N $>$ \frc{1}{\ep$^2$}

From \bpth{1},

% $|$\frc{1}{$\sqrt{n}$} $-$ 0$|$ \eql $\frac{1}{\sqrt{n}}$ $>$ \ep, \fa n $\geq$ N

\epf

}

\ssc{Example 4.1.5}{

Show that if \uw{S}{n} \eql 1 $+$ $\frac{1}{2^n}$, then \uw{S}{n} \lra 1 as n \lra $\infty$.

\bgpf

\lt{\ep $>$ 0}

Then

\uw{S}{n} $-$ S

$|$1 $+$ $\frac{1}{2^n}$ $-$ 1$|$ \eql $\frac{1}{2^n}$ $\leq$ \frc{1}{n} \eql \frc{1}{N} \fa n \mem \bn

Then if N \mem \bn st \frc{1}{N} $<$ \ep

Then $|$1 $+$ $\frac{1}{2^n}$ $-$1$|$ $<$ \ep  \fa n $\geq$ N

\epf

}

\ssc{Theorem 4.1.8}{

\lt{\{\uw{S}{n}\} and \{\uw{a}{n}\} be sequences, s \mem \br}

If some k $>$ 0 and some m \mem \bn, we have:

$|$\uw{S}{n} $-$ s$|$ $\leq$ k$|$\uw{a}{n}$|$, \fa n $\geq$ m \bpth{1}

and if lim$_{n\lra\infty}$ \uw{a}{n} \eql 0, then lim$_{n\lra\infty}$ \uw{S}{n} \eql s.

\bgpf

For \ep $>$ 0, \exs N \mem \bn st

$|$\uw{a}{n}$|$ \eql $|$\uw{a}{n} $-$ 0$|$ $<$ $\frac{\epsilon}{k}$, \fa n $\geq$ N \bpth{2}

From \bpth{1}, 

$|$\uw{S}{n} $-$ s$|$ $\leq$ k$|$\uw{a}{n}$|$ $<$ k(\frc{\ep}{k}) \eql \ep, \fa n $\geq$ N

Hence, \uw{S}{n} \lra as n \lra $\infty$.

\epf

}

\ssc{Example 4.1.11}{

Prove that if \uw{S}{n} \eql n$^\frac{1}{n}$, \fa n \mem \bn,

then,

\uw{S}{n} \lra 1 as n \lra $\infty$

\bgpf

Recall that

n$^\frac{1}{n}$ \eql  e$^{\frac{1}{n}\textrm{ ln n}}$

a$^x$, 0 $<$ a \mem \br \eql e$^{x ln a}$, x \mem \br

Notice that n$^\frac{1}{n}$ $\geq$ 1, \fa n \mem \bn

We write that:

n$^\frac{1}{n}$ \eql 1 $+$ \uw{b}{n}, where \uw{b}{n} $\geq$ 0

Thus:

(n$^\frac{1}{n}$)$^n$ \eql (1 $+$ \uw{b}{n})$^n$

n \eql (1 $+$ \uw{b}{n})$^n$

\textbf{Recall}:

[(a $+$ b)$^n$ \eql (n choose 0) a$^n$ $+$ (n choose 1) + ... + (n choose r) a$^{n - r}$b$^r$ ... $+$ (n choose n$-$1) ab$^{n - 1}$ $+$ (n choose n) a$^0$b$^{n}$]

where

(n choose r) \eql $\frac{n!}{r!(n - r)!}$ for r \eql 0, 1, ... n

(n choose 0) \eql 1, (n choose 1) \eql n, (n choose 2) \eql \frc{1}{2}n(n $-$ 1)

Thus, 

n \eql (1 $+$ \uw{b}{n})$^n$

\eql 1 $+$ n\uw{b}{n} $+$ \frc{1}{2}n(n $+$ 1)\uw{b}{n}$^2$ $+$ ... $+$ bn$^2$ \bpth{1}

\wts{lim$_{n\lra\infty}$ \uw{b}{n} \eql 0}

From \bpth{1},

n $\geq$ \frc{1}{2}n(n $-$ 1)bn$^2$, \fa n $\geq$ 2

1 $\geq$ \frc{1}{2}(n $-$ 1)bn$^2$, \fa n $\geq$ 2

Then bn$^2$ $\leq$ \frc{2}{n - 1} $<$ \ep, \fa n $\geq$ N, 

where N \mem \bn is chosen st N $>$ 2\ep$^2$ $+$ 1 (FIX) \

\

FIX:

bn$^2$ $\leq$ \frc{2}{n-1} $\leq$ \ep$^2$

\frc{n - 1}{2} $>$ \ep$^2$

n $-$ 1 $>$ 2\ep$^2$

n $>$ 2\ep$^2$ $+$ 1

Hence, \uw{b}{n} $<$ \ep, \fa n $\geq$ N.

This proves that lim$_{n\lra\infty}$ \uw{b}{n} \eql 0, implying that lim$_{n\lra\infty}$ n$^\frac{1}{n}$ \eql 1

\epf 
}

\ssc{Example 4.1.12}{

Prove that the sequence \{\uw{S}{n}\}, where \uw{S}{n} \eql 1 $+$ ($-$1)$^n$ is divergent.

\bgpf
Here \{\uw{S}{n}\} \eql 0, 2, 0, 2...

We use contradiction.

\supp{the sequence converges to s \mem \br}

For \ep \eql 1, \exs N \mem \bn st

\step{|1 + (-1)^n - s| < 1}{1}
\fa n $\geq$ N

Notice that from \bpth{1},

\step{|s| < 1}{2}
\fa odd n $\geq$ N

Also from \bpth{1},

\step{|2 - s| < 1}{3}
\fa even n $\geq$ N

From \bpth{2}, $-$1 $<$ s $<$ 1

From \bpth{3},

$-$1 $<$ s $<$ 1

\epf

}

}



}

}

\end{document}