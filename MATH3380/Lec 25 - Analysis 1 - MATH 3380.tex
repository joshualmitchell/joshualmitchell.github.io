% Thank you Josh Davis for this template!
% https://github.com/jdavis/latex-homework-template/blob/master/homework.tex

\documentclass{article}

\newcommand{\hmwkTitle}{Lec\ \#25}

% % Packages

\usepackage{fancyhdr}
\usepackage{extramarks}
\usepackage{amsmath}
\usepackage{amssymb}
\usepackage{amsthm}
\usepackage{amsfonts}
\usepackage{tikz}
\usepackage[plain]{algorithm}
\usepackage{algpseudocode}
\usepackage{enumitem}
\usepackage{chngcntr}

% Libraries

\usetikzlibrary{automata, positioning, arrows}

%
% Basic Document Settings
%

\topmargin=-0.45in
\evensidemargin=0in
\oddsidemargin=0in
\textwidth=6.5in
\textheight=9.0in
\headsep=0.25in

\linespread{1.1}

\pagestyle{fancy}
\lhead{\hmwkAuthorName}
\chead{}
\rhead{\hmwkClass\ (\hmwkClassInstructor): \hmwkTitle}
\lfoot{\lastxmark}
\cfoot{\thepage}

\renewcommand\headrulewidth{0.4pt}
\renewcommand\footrulewidth{0.4pt}

\setlength\parindent{0pt}
\setcounter{secnumdepth}{0}

\newcommand{\hmwkClass}{MATH 3380 / Analysis 1}        % Class
\newcommand{\hmwkClassInstructor}{Dr. Welsh}           % Instructor
\newcommand{\hmwkAuthorName}{\textbf{Joshua Mitchell}} % Author

%
% Title Page
%

\title{
    \vspace{2in}
    \textmd{\textbf{\hmwkClass:\ \hmwkTitle}}\\
    \normalsize\vspace{0.1in}\small\vspace{0.1in}\large{\textit{\hmwkClassInstructor}}
    \vspace{3in}
}

\author{\hmwkAuthorName}
\date{}

\renewcommand{\part}[1]{\textbf{\large Part \Alph{partCounter}}\stepcounter{partCounter}\\}

% Integral dx
\newcommand{\dx}{\mathrm{d}x}

%
% Various Helper Commands
%

% For derivatives
\newcommand{\deriv}[1]{\frac{\mathrm{d}}{\mathrm{d}x} (#1)}

% For partial derivatives
\newcommand{\pderiv}[2]{\frac{\partial}{\partial #1} (#2)}


% Alias for the Solution section header
\newcommand{\solution}{\textbf{\large Solution}}

% Probability commands: Expectation, Variance, Covariance, Bias
\newcommand{\E}{\mathrm{E}}
\newcommand{\Var}{\mathrm{Var}}
\newcommand{\Cov}{\mathrm{Cov}}
\newcommand{\Bias}{\mathrm{Bias}}

% Formatting commands:

\newcommand{\mt}[1]{\ensuremath{#1}}
\newcommand{\nm}[1]{\textrm{#1}}

\newcommand\bsc[2][\DefaultOpt]{%
  \def\DefaultOpt{#2}%
  \section[#1]{#2}%
}
\newcommand\ssc[2][\DefaultOpt]{%
  \def\DefaultOpt{#2}%
  \subsection[#1]{#2}%
}
\newcommand{\bgpf}{\begin{proof} $ $\newline}

\newcommand{\bgeq}{\begin{equation*}}
\newcommand{\eeq}{\end{equation*}}	

\newcommand{\balist}{\begin{enumerate}[label=\alph*.]}
\newcommand{\elist}{\end{enumerate}}

\newcommand{\bilist}{\begin{enumerate}[label=\roman*)]}	

\newcommand{\bgsp}{\begin{split}}
% \newcommand{\esp}{\end{split}} % doesn't work for some reason.

\newcommand\prs[1]{~~~\textbf{(#1)}}

\newcommand{\lt}[1]{\textbf{Let: } #1}
     							   %  if you're setting it to be true
\newcommand{\supp}[1]{\textbf{Suppose: } #1}
     							   %  Suppose (if it'll end up false)
\newcommand{\wts}[1]{\textbf{Want to show: } #1}
     							   %  Want to show
\newcommand{\as}[1]{\textbf{Assume: } #1}
     							   %  if you think it follows from truth
\newcommand{\bpth}[1]{\textbf{(#1)}}

\newcommand{\step}[2]{\begin{equation}\tag{#2}#1\end{equation}}
\newcommand{\epf}{\end{proof}}

\newcommand{\sidenote}[1]{-----------------------------------------------------------------Side Notes---------------------------------------------------------------
#1 \

---------------------------------------------------------------------------------------------------------------------------------------------}

% Analysis / Logical commands:

\newcommand{\br}{\mathbb{R}}       % |R
\newcommand{\bq}{\mathbb{Q}}       % |Q
\newcommand{\bn}{\mathbb{N}}       % |N
\newcommand{\bc}{\mathbb{C}}       % |C
\newcommand{\bz}{\mathbb{Z}}       % |Z

\newcommand{\ep}{\epsilon}         % epsilon
\newcommand{\fa}{\forall}          % for all

\newcommand{\es}{\emptyset}        % empty set
\newcommand{\sbs}{\subset}         % subset of

\newcommand{\lra}{\longrightarrow} % implies ----->
\newcommand{\rar}{\Rightarrow}     % implies -->

\newcommand{\lla}{\longleftarrow}  % implies <-----
\newcommand{\lar}{\Leftarrow}      % implies <--

\newcommand{\pr}{^\prime} 		   % prime (i.e. R')

\newcommand{\bnm}[2]{\mt{#1\setminus{#2}}}
\newcommand{\bnt}[2]{\mt{\textrm{#1}\setminus{\textrm{#2}}}}
\newcommand{\bi}{\bnm{\mathbb{R}}{\mathbb{Q}}}

\newcommand{\nbho}[3]{\textrm{N(}#1, #2\textrm{) }\cap \textrm{ #3} \neq \emptyset}
     							   %  N(x, eps) intersect S \= emptyset
\newcommand{\nbhe}[3]{\textrm{N(}#1, #2\textrm{) }\cap \textrm{ #3} = \emptyset}
     							   %  N(x, eps) intersect S  = emptyset
\newcommand{\dnbho}[3]{\textrm{N*(}#1, #2\textrm{) }\cap \textrm{ #3} \neq \emptyset}
     							   %  N*(x, eps) intersect S \= emptyset
\newcommand{\dnbhe}[3]{\textrm{N*(}#1, #2\textrm{) }\cap \textrm{ #3} = \emptyset}
     							   %  N*(x, eps) intersect S = emptyset
     							 

% ----------

% Packages

\usepackage{fancyhdr}
\usepackage{extramarks}
\usepackage{amsmath}
\usepackage{amssymb}
\usepackage{amsthm}
\usepackage{amsfonts}
\usepackage{tikz}
\usepackage[plain]{algorithm}
\usepackage{algpseudocode}
\usepackage{enumitem}
\usepackage{chngcntr}

% Libraries

\graphicspath{{/Users/jm/iclouddrive/3380pics/}}

\usetikzlibrary{automata, positioning, arrows}

%
% Basic Document Settings
%

\topmargin=-0.45in
\evensidemargin=0in
\oddsidemargin=0in
\textwidth=6.5in
\textheight=9.0in
\headsep=0.25in

\linespread{1.1}

\pagestyle{fancy}
\lhead{\hmwkAuthorName}
\chead{}
\rhead{\hmwkClass\ (\hmwkClassInstructor): \hmwkTitle}
\lfoot{\lastxmark}
\cfoot{\thepage}

\renewcommand\headrulewidth{0.4pt}
\renewcommand\footrulewidth{0.4pt}

\setlength\parindent{0pt}
\setcounter{secnumdepth}{0}

\newcommand{\hmwkClass}{MATH 3380 / Analysis 1}        % Class
\newcommand{\hmwkClassInstructor}{Dr. Welsh}           % Instructor
\newcommand{\hmwkAuthorName}{\textbf{Joshua Mitchell}} % Author

%
% Title Page
%

\title{
    \vspace{2in}
    \textmd{\textbf{\hmwkClass:\ \hmwkTitle}}\\
    \normalsize\vspace{0.1in}\small\vspace{0.1in}\large{\textit{\hmwkClassInstructor}}
    \vspace{3in}
}

\author{\hmwkAuthorName}
\date{}

\renewcommand{\part}[1]{\textbf{\large Part \Alph{partCounter}}\stepcounter{partCounter}\\}

% Integral dx
\newcommand{\dx}{\mathrm{d}x}

%
% Various Helper Commands
%

% For derivatives
\newcommand{\deriv}[1]{\frac{\mathrm{d}}{\mathrm{d}x} (#1)}

% For partial derivatives
\newcommand{\pderiv}[2]{\frac{\partial}{\partial #1} (#2)}


% Alias for the Solution section header
\newcommand{\solution}{\textbf{\large Solution}}

% Probability commands: Expectation, Variance, Covariance, Bias
\newcommand{\E}{\mathrm{E}}
\newcommand{\Var}{\mathrm{Var}}
\newcommand{\Cov}{\mathrm{Cov}}
\newcommand{\Bias}{\mathrm{Bias}}

% Formatting commands:

\newcommand{\mt}[1]{\ensuremath{#1}}
\newcommand{\nm}[1]{\textrm{#1}}

\newcommand\bsc[2][\DefaultOpt]{%
  \def\DefaultOpt{#2}%
  \section[#1]{#2}%
}
\newcommand\ssc[2][\DefaultOpt]{%
  \def\DefaultOpt{#2}%
  \subsection[#1]{#2}%
}
\newcommand{\bgpf}{\begin{proof} $ $\newline}

\newcommand{\bgeq}{\begin{equation*}}
\newcommand{\eeq}{\end{equation*}}	

\newcommand{\balist}{\begin{enumerate}[label=\alph*.]}
\newcommand{\elist}{\end{enumerate}}

\newcommand{\bilist}{\begin{enumerate}[label=\roman*)]}	

\newcommand{\bgsp}{\begin{split}}
% \newcommand{\esp}{\end{split}} % doesn't work for some reason.

\newcommand\prs[1]{~~~\textbf{(#1)}}

\newcommand{\lt}[1]{\textbf{Let: } #1}
     							   %  if you're setting it to be true
\newcommand{\supp}[1]{\textbf{Suppose: } #1}
     							   %  Suppose (if it'll end up false)
\newcommand{\wts}[1]{\textbf{Want to show: } #1}
     							   %  Want to show
\newcommand{\as}[1]{\textbf{Assume: } #1}
     							   %  if you think it follows from truth
\newcommand{\bpth}[1]{\textbf{(#1)}}

\newcommand{\step}[2]{\begin{equation}\tag{#2}#1\end{equation}}
\newcommand{\epf}{\end{proof}}

\newcommand{\dbs}[3]{\mt{#1_{#2_#3}}}

\newcommand{\sidenote}[1]{-----------------------------------------------------------------Side Note----------------------------------------------------------------
#1 \

---------------------------------------------------------------------------------------------------------------------------------------------}

% Analysis / Logical commands:

\newcommand{\br}{\mt{\mathbb{R}} }       % |R
\newcommand{\bq}{\mt{\mathbb{Q}} }       % |Q
\newcommand{\bn}{\mt{\mathbb{N}} }       % |N
\newcommand{\bc}{\mt{\mathbb{C}} }       % |C
\newcommand{\bz}{\mt{\mathbb{Z}} }       % |Z

\newcommand{\ep}{\mt{\epsilon} }         % epsilon
\newcommand{\fa}{\mt{\forall} }          % for all
\newcommand{\afa}{\mt{\alpha} }
\newcommand{\bta}{\mt{\beta} }
\newcommand{\dta}{\mt{\delta} }
\newcommand{\mem}{\mt{\in} }
\newcommand{\exs}{\mt{\exists} }

\newcommand{\es}{\mt{\emptyset} }        % empty set
\newcommand{\sbs}{\mt{\subset} }         % subset of
\newcommand{\fs}[2]{\{\uw{#1}{1}, \uw{#1}{2}, ... \uw{#1}{#2}\}}

\newcommand{\lra}{ \mt{\longrightarrow} } % implies ----->
\newcommand{\rar}{ \mt{\Rightarrow} }     % implies -->

\newcommand{\lla}{ \mt{\longleftarrow} }  % implies <-----
\newcommand{\lar}{ \mt{\Leftarrow} }      % implies <--

\newcommand{\av}[1]{\mt{|}#1\mt{|}}  % absolute value

\newcommand{\prn}[1]{(#1)}
\newcommand{\bk}[1]{\{#1\}}

\newcommand{\ps}{\mt{+} }
\newcommand{\ms}{\mt{-} }

\newcommand{\ls}{\mt{<} }
\newcommand{\gr}{\mt{>} }

\newcommand{\lse}{\mt{\leq} }
\newcommand{\gre}{\mt{\geq} }

\newcommand{\eql}{\mt{=} }

\newcommand{\pr}{\mt{^\prime} } 		   % prime (i.e. R')
\newcommand{\uw}[2]{#1\mt{_{#2}}}
\newcommand{\uf}[2]{#1\mt{^{#2}}}
\newcommand{\frc}[2]{\mt{\frac{#1}{#2}}}
\newcommand{\lmti}[1]{\mt{\displaystyle{\lim_{#1 \to \infty}}}}
\newcommand{\limt}[2]{\mt{\displaystyle{\lim_{#1 \to #2}}}}

\newcommand{\bnm}[2]{\mt{#1\setminus{#2}}}
\newcommand{\bnt}[2]{\mt{\textrm{#1}\setminus{\textrm{#2}}}}
\newcommand{\bi}{\bnm{\mathbb{R}}{\mathbb{Q}}}

\newcommand{\urng}[2]{\mt{\bigcup_{#1}^{#2}}}
\newcommand{\nrng}[2]{\mt{\bigcap_{#1}^{#2}}}
\newcommand{\nck}[2]{\mt{{#1 \choose #2}}}

\newcommand{\nbho}[3]{\textrm{N(}#1, #2\textrm{) }\cap \textrm{ #3} \neq \emptyset}
     							   %  N(x, eps) intersect S \= emptyset
\newcommand{\nbhe}[3]{\textrm{N(}#1, #2\textrm{) }\cap \textrm{ #3} = \emptyset}
     							   %  N(x, eps) intersect S  = emptyset
\newcommand{\dnbho}[3]{\textrm{N*(}#1, #2\textrm{) }\cap \textrm{ #3} \neq \emptyset}
     							   %  N*(x, eps) intersect S \= emptyset
\newcommand{\dnbhe}[3]{\textrm{N*(}#1, #2\textrm{) }\cap \textrm{ #3} = \emptyset}
     							   %  N*(x, eps) intersect S = emptyset
     							   
\newcommand{\eqn}[1]{\[#1\]}
\newcommand{\splt}[1]{\begin{split}#1\end{split}}

\newcommand{\infy}{\mt{\infty} }
\newcommand{\unn}{\mt{\cup} }
\newcommand{\inn}{\mt{\cap} }
\newcommand\tab[1][1cm]{\hspace*{#1}}

\newcommand{\wit}[1]{\mt{\widetilde{#1}}}
     							 
% ----------

\begin{document}

HW 11: page 220 - 221, \#1, 2, 5 and page 226-227, \# 1 - 3, 4(a)(b), 5, 11

\bsc{5.4 Continued}{

\ssc{Theorem 5.4.6}{

f : D \lra \br is continuous and D is compact.

Then f is uniformly continuous on D.

\bgpf

\lt{c \mem D and let \ep \gr 0}

Since if is continuous on D, \exs \dta(c) \gr 0 st
\step{|f(x) - f(c)| < \frac{\epsilon}{2}}{1}
whenever \av{x \ms c} \ls \dta(c) and x \mem D

Notice that:
\eqn{D \sbs \urng{c \mem D}{} N (c, \frac{\delta(c)}{2})}
Since D is compact,
\step{D \sbs \urng{i = 1}{n} N(c_i, \frac{\dta(c_i)}{2})}{2}
\lt{\dta \eql min\bk{\frc{\dta(\uw{c}{1})}{2}, \frc{\dta(\uw{c}{2})}{2}, ... \frc{\dta(\uw{c}{n})}{2}} and x, y \mem D st \av{x \ms y} \ls \dta}

Then, from \bpth{2},

\exs k \mem \bk{1, 2, ... n} st x \mem N(\uw{c}{k}, \frc{\dta(\uw{c}{k})}{2})

Thus,
\step{|x - c_k| < \frac{\delta(c_k)}{2} < \delta(c_k)}{3}
and
\step{|y - c_k| < |y - x| + |x - c_k| < \delta + \frac{\delta(c_k)}{2} \lse \frac{\delta(c_k)}{2} + \frac{\delta(c_k)}{2} = \delta(c_k)}{4}
Now:
\eqn{|f(x) - f(y)| \lse |f(x) - f(c_k)| + |f(c_k) - f(y)|}
So, from \bpth{1}, \bpth{3}, and \bpth{4},
\eqn{|f(x) - f(y)| < \frac{\epsilon}{2} + \frac{\epsilon}{2} = \epsilon}
Hence, result.
\epf

}

\newpage

\ssc{Practice 5.4.7}{

Find a continuous function f : D \lra \br and a Cauchy sequence \bk{\uw{x}{n}} in D st \bk{f(\uw{x}{n})} is divergent.

This is a good practice problem because it will show us why the next theorem is so useful.

\bgpf

f : (0, 1) \lra \br where f(x) \eql \frc{1}{x} (we could have also used \uf{x}{2})

\lt{\uw{x}{n} \eql \frc{1}{n} \fa n \mem \bn}

Then \lmti{n} \uw{x}{n} \eql 0

So \bk{\uw{x}{n}} is a Cauchy sequence.

However,

f(\uw{x}{n}) \eql \frc{1}{\frc{1}{n}} \eql n \lra \infy as n \lra \infy

Hence,

\bk{f(\uw{x}{n})} diverges.

\epf

}

\ssc{Theorem 5.4.8}{

\lt{f : D \lra \br be uniformly continuous on D}

\as{\bk{\uw{x}{n}} is a Cauchy sequence in D}

Then,

\bk{f(\uw{x}{n})} is a Cauchy sequence.

\bgpf

For \ep \gr 0, \exs \dta \gr 0 st
\eqn{\av{x - y} < \dta \textrm{ and } x, y \in D \rar |f(x) - f(y)| < \ep}

Since \av{\uw{x}{n}} is Cauchy,

\exs N \mem \bn st \av{\uw{x}{n} \ms \uw{x}{m}} \ls \dta whenever n, m \gre N

Hence,

\av{f(\uw{x}{n}) \ms f(\uw{x}{m})} \ls \ep whenever n, m \gre N

which shows that \bk{f(\uw{x}{n})} is Cauchy.
\epf

}

\newpage

\ssc{Theorem 5.4.9}{

A function f : (a, b) \lra \br is uniformly continuous on (a, b)

iff f can be extended to a function \wit{f} : [a, b] \lra \br

where

\wit{f} is continuous on [a, b].

\sidenote{
We say that a function \wit{f} : E \lra \br is an extension of a function f : D \lra \br

if D \sbs E and \wit{f}(x) \eql f(x), \fa x \mem D
}

\bgpf

\lra 

\as{f : (a, b) \lra is uniformly continuous on (a, b)}

\lt{\bk{\uw{x}{n}} and\bk{\uw{y}{n}} be sequences in (a, b) st \uw{x}{n} \lra a and \uw{y}{n} \lra b as n \lra \infy}

Then \bk{\uw{x}{n}} and\bk{\uw{y}{n}} are Cauchy sequences in D.

By Theorem 5.4.8, \bk{f(\uw{x}{n})} and \bk{f(\uw{y}{n})} are Cauchy sequences which, therefore, converge. 

\lt{\lmti{n} f(\uw{x}{n}) \eql p and \lmti{n} f(\uw{y}{n}) \eql q}

Define \wit{f} : [a, b] \lra \br by

\wit{f}(x) \eql f(x) if x \mem (a, b), \tab p if x \eql a, and \tab q if x \eql b

Then \wit{f} is an extension of f, which is continuous.

Notice that \uw{x}{n} \lra a as n \lra \infy and \lmti{n} \wit{f}(\uw{x}{n}) \eql \lmti{n} f(\uw{x}{n}) \eql p \eql \wit{f}(a)

Hence, 

\wit{f}(x) is continuous at x \eql a by Theorem 5.2.2(b).

Similarly, 

\lmti{n} \wit{f}(\uw{y}{n}) \eql \lmti{n} f(\uw{y}{n}) \eql q \eql \wit{f}(b)

Hence, 

\wit{f}(x) is continuous at x \eql b by Theorem 5.2.2(b).

Since \wit{f}(x) \eql f(x) \fa x \mem (a, b), then \wit{f} is continuous on (a, b).

Hence,

\wit{f} is continuous on [a, b].

\lla

Conversely,

\as{f can be extended to a function \wit{f} : [a, b] \lra \br where \wit{f} is continuous on [a, b]}

By Theorem 5.4.6, \wit{f} is uniformly continuous on [a, b], since by the Heine-Borel Theorem, [a, b] is compact.

Hence,

\wit{f} is uniformly continuous on (a, b).

Since \wit{f}(x) \eql f(x) \fa x \mem (a, b), 

f is uniformly continuous on (a, b).

Hence, result.

\epf

}

\newpage 

\ssc{Practice: 5.4.10}{

Use Thm 5.4.9 to determine whether or not the function f(x) \eql sin(\frc{1}{x}) is uniformly continuous on (0, \frc{1}{\pi}).

\bgpf

\lt{\uw{x}{n} \eql \frc{2}{n\pi}, \fa n \mem \bn}

Then f(\uw{x}{n}) \eql sin(n\frc{\pi}{2}) \fa n \mem \bn

Here, \lmti{n} \uw{x}{n} \eql 0

Notice that \lmti{k} f(\uw{x}{2k}) \eql 0

However, \lmti{k} f(\uw{x}{4k - 3}) \eql 1

Hence,

\bk{f(\uw{x}{n})} does not converge (since, if it did, then all its subsequences would have to converge to the same limit, which they do not).

\epf

}

}

\bsc{Chapter 6: Section 6.1}{

\ssc{Definition 6.1.1}{

\lt{f be real-valued and defined on an interval containing the point c (possibly an end-point)}

We say that f is \textbf{differentiable at c} (i.e. has a derivative at c) if
\eqn{\limt{x}{c} \frac{f(x) - f(c)}{x - c}}
exists (and is, therefore, finite).


In this case,
\eqn{f\pr(c) = \limt{x}{c} \frac{f(x) - f(c)}{x - c}}
Alternatively,
\eqn{f\pr(x) = \limt{t}{x} \frac{f(t) - f(x)}{t - x}}

}

\ssc{Example 6.1.2}{

\lt{f(x) \eql \uf{x}{2}, \fa x \mem \br}

For any c \mem \br, we have:

\eqn{f\pr(c) = \limt{x}{c} \frac{f(x) - f(c)}{x - c} = \limt{x}{c} \frac{x^2 - c^2}{x - c} = \limt{x}{c} \frac{(x - c)(x + c)}{x - c} = \limt{x}{c} (x \ps c) = 2c}

}

}

\end{document}