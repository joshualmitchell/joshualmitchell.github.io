% Thank you Josh Davis for this template!
% https://github.com/jdavis/latex-homework-template/blob/master/homework.tex

\documentclass{article}

\newcommand{\hmwkTitle}{Lecture\ \#6}

% % Packages

\usepackage{fancyhdr}
\usepackage{extramarks}
\usepackage{amsmath}
\usepackage{amssymb}
\usepackage{amsthm}
\usepackage{amsfonts}
\usepackage{tikz}
\usepackage[plain]{algorithm}
\usepackage{algpseudocode}
\usepackage{enumitem}
\usepackage{chngcntr}

% Libraries

\usetikzlibrary{automata, positioning, arrows}

%
% Basic Document Settings
%

\topmargin=-0.45in
\evensidemargin=0in
\oddsidemargin=0in
\textwidth=6.5in
\textheight=9.0in
\headsep=0.25in

\linespread{1.1}

\pagestyle{fancy}
\lhead{\hmwkAuthorName}
\chead{}
\rhead{\hmwkClass\ (\hmwkClassInstructor): \hmwkTitle}
\lfoot{\lastxmark}
\cfoot{\thepage}

\renewcommand\headrulewidth{0.4pt}
\renewcommand\footrulewidth{0.4pt}

\setlength\parindent{0pt}
\setcounter{secnumdepth}{0}

\newcommand{\hmwkClass}{MATH 3380 / Analysis 1}        % Class
\newcommand{\hmwkClassInstructor}{Dr. Welsh}           % Instructor
\newcommand{\hmwkAuthorName}{\textbf{Joshua Mitchell}} % Author

%
% Title Page
%

\title{
    \vspace{2in}
    \textmd{\textbf{\hmwkClass:\ \hmwkTitle}}\\
    \normalsize\vspace{0.1in}\small\vspace{0.1in}\large{\textit{\hmwkClassInstructor}}
    \vspace{3in}
}

\author{\hmwkAuthorName}
\date{}

\renewcommand{\part}[1]{\textbf{\large Part \Alph{partCounter}}\stepcounter{partCounter}\\}

% Integral dx
\newcommand{\dx}{\mathrm{d}x}

%
% Various Helper Commands
%

% For derivatives
\newcommand{\deriv}[1]{\frac{\mathrm{d}}{\mathrm{d}x} (#1)}

% For partial derivatives
\newcommand{\pderiv}[2]{\frac{\partial}{\partial #1} (#2)}


% Alias for the Solution section header
\newcommand{\solution}{\textbf{\large Solution}}

% Probability commands: Expectation, Variance, Covariance, Bias
\newcommand{\E}{\mathrm{E}}
\newcommand{\Var}{\mathrm{Var}}
\newcommand{\Cov}{\mathrm{Cov}}
\newcommand{\Bias}{\mathrm{Bias}}

% Formatting commands:

\newcommand{\mt}[1]{\ensuremath{#1}}
\newcommand{\nm}[1]{\textrm{#1}}

\newcommand\bsc[2][\DefaultOpt]{%
  \def\DefaultOpt{#2}%
  \section[#1]{#2}%
}
\newcommand\ssc[2][\DefaultOpt]{%
  \def\DefaultOpt{#2}%
  \subsection[#1]{#2}%
}
\newcommand{\bgpf}{\begin{proof} $ $\newline}

\newcommand{\bgeq}{\begin{equation*}}
\newcommand{\eeq}{\end{equation*}}	

\newcommand{\balist}{\begin{enumerate}[label=\alph*.]}
\newcommand{\elist}{\end{enumerate}}

\newcommand{\bilist}{\begin{enumerate}[label=\roman*)]}	

\newcommand{\bgsp}{\begin{split}}
% \newcommand{\esp}{\end{split}} % doesn't work for some reason.

\newcommand\prs[1]{~~~\textbf{(#1)}}

\newcommand{\lt}[1]{\textbf{Let: } #1}
     							   %  if you're setting it to be true
\newcommand{\supp}[1]{\textbf{Suppose: } #1}
     							   %  Suppose (if it'll end up false)
\newcommand{\wts}[1]{\textbf{Want to show: } #1}
     							   %  Want to show
\newcommand{\as}[1]{\textbf{Assume: } #1}
     							   %  if you think it follows from truth
\newcommand{\bpth}[1]{\textbf{(#1)}}

\newcommand{\step}[2]{\begin{equation}\tag{#2}#1\end{equation}}
\newcommand{\epf}{\end{proof}}

\newcommand{\sidenote}[1]{-----------------------------------------------------------------Side Notes---------------------------------------------------------------
#1 \

---------------------------------------------------------------------------------------------------------------------------------------------}

% Analysis / Logical commands:

\newcommand{\br}{\mathbb{R}}       % |R
\newcommand{\bq}{\mathbb{Q}}       % |Q
\newcommand{\bn}{\mathbb{N}}       % |N
\newcommand{\bc}{\mathbb{C}}       % |C
\newcommand{\bz}{\mathbb{Z}}       % |Z

\newcommand{\ep}{\epsilon}         % epsilon
\newcommand{\fa}{\forall}          % for all

\newcommand{\es}{\emptyset}        % empty set
\newcommand{\sbs}{\subset}         % subset of

\newcommand{\lra}{\longrightarrow} % implies ----->
\newcommand{\rar}{\Rightarrow}     % implies -->

\newcommand{\lla}{\longleftarrow}  % implies <-----
\newcommand{\lar}{\Leftarrow}      % implies <--

\newcommand{\pr}{^\prime} 		   % prime (i.e. R')

\newcommand{\bnm}[2]{\mt{#1\setminus{#2}}}
\newcommand{\bnt}[2]{\mt{\textrm{#1}\setminus{\textrm{#2}}}}
\newcommand{\bi}{\bnm{\mathbb{R}}{\mathbb{Q}}}

\newcommand{\nbho}[3]{\textrm{N(}#1, #2\textrm{) }\cap \textrm{ #3} \neq \emptyset}
     							   %  N(x, eps) intersect S \= emptyset
\newcommand{\nbhe}[3]{\textrm{N(}#1, #2\textrm{) }\cap \textrm{ #3} = \emptyset}
     							   %  N(x, eps) intersect S  = emptyset
\newcommand{\dnbho}[3]{\textrm{N*(}#1, #2\textrm{) }\cap \textrm{ #3} \neq \emptyset}
     							   %  N*(x, eps) intersect S \= emptyset
\newcommand{\dnbhe}[3]{\textrm{N*(}#1, #2\textrm{) }\cap \textrm{ #3} = \emptyset}
     							   %  N*(x, eps) intersect S = emptyset
     							 

% ----------


% Packages

\usepackage{fancyhdr}
\usepackage{extramarks}
\usepackage{amsmath}
\usepackage{amssymb}
\usepackage{amsthm}
\usepackage{amsfonts}
\usepackage{tikz}
\usepackage[plain]{algorithm}
\usepackage{algpseudocode}
\usepackage{enumitem}
\usepackage{chngcntr}

% Libraries

\usetikzlibrary{automata, positioning, arrows}

%
% Basic Document Settings
%

\topmargin=-0.45in
\evensidemargin=0in
\oddsidemargin=0in
\textwidth=6.5in
\textheight=9.0in
\headsep=0.25in

\linespread{1.1}

\pagestyle{fancy}
\lhead{\hmwkAuthorName}
\chead{}
\rhead{\hmwkClass\ (\hmwkClassInstructor): \hmwkTitle}
\lfoot{\lastxmark}
\cfoot{\thepage}

\renewcommand\headrulewidth{0.4pt}
\renewcommand\footrulewidth{0.4pt}

\setlength\parindent{0pt}
\setcounter{secnumdepth}{0}

\newcommand{\hmwkClass}{MATH 3380 / Analysis 1}        % Class
\newcommand{\hmwkClassInstructor}{Dr. Welsh}           % Instructor
\newcommand{\hmwkAuthorName}{\textbf{Joshua Mitchell}} % Author

%
% Title Page
%

\title{
    \vspace{2in}
    \textmd{\textbf{\hmwkClass:\ \hmwkTitle}}\\
    \normalsize\vspace{0.1in}\small\vspace{0.1in}\large{\textit{\hmwkClassInstructor}}
    \vspace{3in}
}

\author{\hmwkAuthorName}
\date{}

\renewcommand{\part}[1]{\textbf{\large Part \Alph{partCounter}}\stepcounter{partCounter}\\}

% Integral dx
\newcommand{\dx}{\mathrm{d}x}

%
% Various Helper Commands
%

% For derivatives
\newcommand{\deriv}[1]{\frac{\mathrm{d}}{\mathrm{d}x} (#1)}

% For partial derivatives
\newcommand{\pderiv}[2]{\frac{\partial}{\partial #1} (#2)}


% Alias for the Solution section header
\newcommand{\solution}{\textbf{\large Solution}}

% Probability commands: Expectation, Variance, Covariance, Bias
\newcommand{\E}{\mathrm{E}}
\newcommand{\Var}{\mathrm{Var}}
\newcommand{\Cov}{\mathrm{Cov}}
\newcommand{\Bias}{\mathrm{Bias}}

% Formatting commands:

\newcommand{\mt}[1]{\ensuremath{#1}}
\newcommand{\nm}[1]{\textrm{#1}}

\newcommand\bsc[2][\DefaultOpt]{%
  \def\DefaultOpt{#2}%
  \section[#1]{#2}%
}
\newcommand\ssc[2][\DefaultOpt]{%
  \def\DefaultOpt{#2}%
  \subsection[#1]{#2}%
}
\newcommand{\bgpf}{\begin{proof} $ $\newline}

\newcommand{\bgeq}{\begin{equation*}}
\newcommand{\eeq}{\end{equation*}}	

\newcommand{\balist}{\begin{enumerate}[label=\alph*.]}
\newcommand{\elist}{\end{enumerate}}

\newcommand{\bilist}{\begin{enumerate}[label=\roman*)]}	

\newcommand{\bgsp}{\begin{split}}
% \newcommand{\esp}{\end{split}} % doesn't work for some reason.

\newcommand\prs[1]{~~~\textbf{(#1)}}

\newcommand{\lt}[1]{\textbf{Let: } #1}
     							   %  if you're setting it to be true
\newcommand{\supp}[1]{\textbf{Suppose: } #1}
     							   %  Suppose (if it'll end up false)
\newcommand{\wts}[1]{\textbf{Want to show: } #1}
     							   %  Want to show
\newcommand{\as}[1]{\textbf{Assume: } #1}
     							   %  if you think it follows from truth
\newcommand{\bpth}[1]{\textbf{(#1)}}

\newcommand{\step}[2]{\begin{equation}\tag{#2}#1\end{equation}}
\newcommand{\epf}{\end{proof}}

\newcommand{\dbs}[3]{\mt{#1_{#2_#3}}}

\newcommand{\sidenote}[1]{-----------------------------------------------------------------Side Notes---------------------------------------------------------------
#1 \

---------------------------------------------------------------------------------------------------------------------------------------------}

% Analysis / Logical commands:

\newcommand{\br}{\mt{\mathbb{R}} }       % |R
\newcommand{\bq}{\mt{\mathbb{Q}} }       % |Q
\newcommand{\bn}{\mt{\mathbb{N}} }       % |N
\newcommand{\bc}{\mt{\mathbb{C}} }       % |C
\newcommand{\bz}{\mt{\mathbb{Z}} }       % |Z

\newcommand{\ep}{\mt{\epsilon} }         % epsilon
\newcommand{\fa}{\mt{\forall} }          % for all
\newcommand{\afa}{\mt{\alpha} }
\newcommand{\mem}{\mt{\in} }
\newcommand{\exs}{\mt{\exists} }

\newcommand{\es}{\mt{\emptyset}}        % empty set
\newcommand{\sbs}{\mt{\subset} }         % subset of
\newcommand{\fs}[2]{\{\uw{#1}{1}, \uw{#1}{2}, ... \uw{#1}{#2}\}}

\newcommand{\lra}{\mt{\longrightarrow}} % implies ----->
\newcommand{\rar}{\mt{\Rightarrow}}     % implies -->

\newcommand{\lla}{\mt{\longleftarrow}}  % implies <-----
\newcommand{\lar}{\mt{\Leftarrow}}      % implies <--

\newcommand{\eql}{\mt{=} }
\newcommand{\pr}{\mt{^\prime}} 		   % prime (i.e. R')
\newcommand{\uw}[2]{#1\mt{_#2}}
\newcommand{\frc}[2]{\mt{\frac{#1}{#2}}}

\newcommand{\bnm}[2]{\mt{#1\setminus{#2}}}
\newcommand{\bnt}[2]{\mt{\textrm{#1}\setminus{\textrm{#2}}}}
\newcommand{\bi}{\bnm{\mathbb{R}}{\mathbb{Q}}}

\newcommand{\urng}[2]{\mt{\bigcup_{#1}^{#2}}}

\newcommand{\nbho}[3]{\textrm{N(}#1, #2\textrm{) }\cap \textrm{ #3} \neq \emptyset}
     							   %  N(x, eps) intersect S \= emptyset
\newcommand{\nbhe}[3]{\textrm{N(}#1, #2\textrm{) }\cap \textrm{ #3} = \emptyset}
     							   %  N(x, eps) intersect S  = emptyset
\newcommand{\dnbho}[3]{\textrm{N*(}#1, #2\textrm{) }\cap \textrm{ #3} \neq \emptyset}
     							   %  N*(x, eps) intersect S \= emptyset
\newcommand{\dnbhe}[3]{\textrm{N*(}#1, #2\textrm{) }\cap \textrm{ #3} = \emptyset}
     							   %  N*(x, eps) intersect S = emptyset
     							 


% ---------

\begin{document}

\bsc{Section 3.5: Compact Sets}{
Three big areas of analysis: compactedness, continuity, and connectedness.

\ssc{Definition 3.5.1}{A set s \sbs \br  is said to be compact if every \textbf{open cover} has a finite \textbf{subcover} \

(i.e. if S \sbs \urng{\alpha \in I}{} \uw{G}{\afa}) , \

where \uw{G}{\afa} is open \fa \afa \mem I; then \exs n \mem \bn and \exs \fs{n}{k} \sbs I \

st S \sbs \urng{i=1}{n} \dbs{G}{\alpha}{i}
}

\ssc{Example 3.5.2}{
\balist
\item Show that S \eql (0, 2) is not compact.
\item Show that S \eql \{\uw{x}{1}, \uw{x}{2}, ... \uw{x}{n}\} \sbs \br is compact. 
\elist

\textbf{(a)} \

Notice that:
\step{(0, 2) \sbs \urng{n=1}{\infty} (\frac{1}{n}, 3)}{1}
}

If (0,2) were compact, then from \bpth{1} there would exist a \textbf{finite} subcover.

\as{(0, 2) is compact.} \

So \exs k \mem \bn and \{\uw{n}{1}, \uw{n}{2}, ... \uw{n}{k}\} \sbs \uw{\bn}{k} st

\step{(0, 2) \sbs \urng{i=1}{k} (\frac{1}{n_i}, 3)}{2}

Choose m \eql max \fs{n}{k} \

Then, notice that (\frc{1}{n_i} , 3) \sbs (\frc{1}{m}, 3) \fa i \eql 1, 2, ... k \

From \bpth{1}, (0, 2) $\sbs$ ($\frac{1}{m}$, 3).

Notice that 0 $< \frac{1}{m + 1} < \frac{1}{m}$ \

and $\frac{1}{m + 1} \in$ (0, 2). \

However, $\frac{1}{m + 1} \not\in$ ($\frac{1}{m}$, 3).

\textbf{(b)} \

Suppose that S $\sbs$ $\bigcup_\alpha G_\alpha$ ($\alpha \in I$)

where I is an index set and G$_\alpha$ is open $\fa \alpha \in$ I.

$\fa i = 1, 2,...$ n, $\exists \alpha_i \in$ I st x$_i \in G_\alpha$ ($\alpha_i$). \

Then, S $\sbs \bigcup^n_i G_\alpha$ ($\alpha_i$).

We see that a \textbf{finite} subset of $\br$ is compact.
}
\ssc{Lemma 3.5.4}{
If $\es \neq S \sbs \br$ and S is \textbf{closed} and \textbf{bounded}, then S has a maximum and a minimum. In fact, in this, max S $=$ sup S, and min S $=$ inf S.

\bgpf

Since S is bounded, inf S, sup S $\in \br$ both exist.
\wts{max S $=$ sup S}

For $\epsilon > 0$, $\exists s_1 (\ep) \in$ S st \

sup S $- \ep < S_1 \leq$ sup S $<$ sup S $+ \ep$.

So,

$-\ep < s_1 - sup S \leq \ep$

Thus, $s_1 \in$ N(sup S, $\ep$). \

So, 

\step{\nbho{\textrm{sup S}}{\ep}{S}}{1}

Also, sup S $+ \frac{\epsilon}{2} \in$ N(sup S, $\ep$) and sup S $+ \frac{\epsilon}{2} \in \bnm{\br}{S}$.

(s $\leq$ sup S $\fa$ s $\in$ S, and sup S $\in$ S)

---

% \step{\nbho{\textrm{sup S}}{\epsilon}{\butnot{\br}{S}}}

From \bpth{1} and \bpth{2}, sup S $\in$ bd S $\sbs$ S, since S is closed. Hence, sup S $=$ max S.

\epf
}

\ssc{Theorem 3.5.5 (Heine-Borel)}{

A subset $\es \neq$ S $\sbs \br$ is compact iff S is closed and bounded.

\bgpf
$\lra$ \

\supp{S is compact}

\wts{S$_\infty$ is bounded}

Notice that S $\sbs$ From n=1 to $\infty$, $\bigcup$ ($-$n, n) $= \br$,

where ($-$n, n) $=$ N(0, n) is open $\fa n \in \bn$.

G$_n \sbs$ G sub n + 1 $\fa n \in \bn$.

Since S is compact, $\exists k \in \bn$ and \{n$_1$, n$_2$, ... n$_k$\} $\sbs \bn$ st \

S $\sbs$ from i$=$1 to k $\bigcup$ ($-$n$_i$, n$_i$) \

\lt{m $=$ max \{n$_1$, n$_2$, ... n$_k$\}. \

Then, ($-$n$_i$, n$_i$) $\sbs$ ($-$m, m) $\fa i = 1, 2, ... k$.

Thus, S $\sbs$ ($-$m, m).

So, $|S| <$ m, $\fa s \in$ S. Or, equivalently, \

$-$m $<$ s $<$ m, $\fa$ s $\in$ S.

Hence, S is bounded. \

\wts{S is closed} \

\supp{S is not closed} \

Thus, $\exists$ p $\in$ $\bnt{cl S}{S}$, i.e. p $\in$ s$\pr$.

---------- \

S is closed iff cl S $=$ S $\cup$ S$\pr =$ S \

S $\sbs$ S $\cup$ S$\pr$ \

If cl S $\neq$ S, then S $\sbs$ S $\cup$ S$\pr$

---------- \

Notice that: \

From n $=$ 1 to $\infty$, $\bigcap$ [p $- \frac{1}{n}$, p $+ \frac{1}{n}$] $=$ \{p\}

So, $\br$ but not From n $=$ 1 to $\infty$, $\bigcap$ [p $- \frac{1}{n}$, p $+ \frac{1}{n}$] is equal to $\br$ but not \{p\}.

S $\sbs$ $\br$ but not From n $=$ 1 to $\infty$, $\bigcap$ [p $- \frac{1}{n}$, p $+ \frac{1}{n}$] \

S $\sbs$ From n $=$ 1 to $\infty$, $\bigcup$ $\br$ but not [p $- \frac{1}{n}$, p $+ \frac{1}{n}$]

S $\sbs$ From n $=$ 1 to $\infty$, $\bigcup$ [($-\infty$, p $- \frac{1}{n}$) $\bigcup$ (p $+ \frac{1}{n}$, $\infty$)]

Since S is compact, $\exists$ k $\in \bn$ and \{n$_1$, n$_2$, ... n$_k$\} $\sbs$ $\bn$ st \

S $\sbs$ From i $=$ 1 to k, $\bigcup$ [($-\infty$, p $- \frac{1}{n_i}$) $\bigcup$ (p $+ \frac{1}{n_i}$, $\infty$)]

\lt{m $=$ max \{n$_1$, n$_2$, ... n$_k$\}}

Then ($-\infty$, p $- \frac{1}{n_i}$) $\bigcup$ (p $+ \frac{1}{n_i}$, $\infty$) $\sbs$ ($-\infty$, p $- \frac{1}{m}$) $\bigcup$ (p $+ \frac{1}{m}$, $\infty$).

Thus, S $\sbs$ [($-\infty$, p $- \frac{1}{m}$) $\bigcup$ (p $+ \frac{1}{m}$, $\infty$)]

$\lla$ \

Conversely, \
 
\supp{S is closed and bounded} \

\wts{S is compact} \

Let us suppose that S $\sbs$ when $\alpha \in I, \bigcup$ G$_\alpha$,

where G$_\alpha$ is open $\fa \alpha \in I$, where I is some index.

$\fa x \in \br$, define:

S$_x$ $=$ S $\cap$ ($-\infty$, x] \

Also define the set:

Beta $=$ \{x $\in \br$: S$_x$ is covered by a finite collection of the G$_\alpha$'s\}

Notice that S is bounded, so inf S $\in \br$.

and S$_inf S$ $=$ S $\cap$ ($-\infty$, inf S] $=$ \{inf S\}\

Since by Lemma 3.5.4, inf S $=$ min S.

Now, since min S $=$ inf S $\in$ S, then $\exists$ $\alpha_0$ $\in$ I st inf S $\in$ G$_\alpha$ ($\alpha_0$). \

This proves that \

S$_inf S$ $=$ \{inf S\} $\sbs$ G$_\alpha$ ($\alpha_0$)

Hence, inf S \mt{\in} Beta $\neq$ $\es$. \

\epf

}
\end{document}