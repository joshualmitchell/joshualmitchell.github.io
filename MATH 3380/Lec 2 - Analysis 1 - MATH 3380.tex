% Thank you Josh Davis for this template!
% https://github.com/jdavis/latex-homework-template/blob/master/homework.tex

\documentclass{article}

% Packages

\usepackage{fancyhdr}
\usepackage{extramarks}
\usepackage{amsmath}
\usepackage{amssymb}
\usepackage{amsthm}
\usepackage{amsfonts}
\usepackage{tikz}
\usepackage[plain]{algorithm}
\usepackage{algpseudocode}
\usepackage{enumitem}
\usepackage{chngcntr}

% Libraries

\usetikzlibrary{automata, positioning, arrows}

%
% Basic Document Settings
%

\topmargin=-0.45in
\evensidemargin=0in
\oddsidemargin=0in
\textwidth=6.5in
\textheight=9.0in
\headsep=0.25in

\linespread{1.1}

\pagestyle{fancy}
\lhead{\hmwkAuthorName}
\chead{}
\rhead{\hmwkClass\ (\hmwkClassInstructor): \hmwkTitle}
\lfoot{\lastxmark}
\cfoot{\thepage}

\renewcommand\headrulewidth{0.4pt}
\renewcommand\footrulewidth{0.4pt}

\setlength\parindent{0pt}
\setcounter{secnumdepth}{0}

\newcommand{\hmwkClass}{MATH 3380 / Analysis 1}        % Class
\newcommand{\hmwkClassInstructor}{Dr. Welsh}           % Instructor
\newcommand{\hmwkAuthorName}{\textbf{Joshua Mitchell}} % Author

%
% Title Page
%

\title{
    \vspace{2in}
    \textmd{\textbf{\hmwkClass:\ \hmwkTitle}}\\
    \normalsize\vspace{0.1in}\small\vspace{0.1in}\large{\textit{\hmwkClassInstructor}}
    \vspace{3in}
}

\author{\hmwkAuthorName}
\date{}

\renewcommand{\part}[1]{\textbf{\large Part \Alph{partCounter}}\stepcounter{partCounter}\\}

% Integral dx
\newcommand{\dx}{\mathrm{d}x}

%
% Various Helper Commands
%

% For derivatives
\newcommand{\deriv}[1]{\frac{\mathrm{d}}{\mathrm{d}x} (#1)}

% For partial derivatives
\newcommand{\pderiv}[2]{\frac{\partial}{\partial #1} (#2)}


% Alias for the Solution section header
\newcommand{\solution}{\textbf{\large Solution}}

% Probability commands: Expectation, Variance, Covariance, Bias
\newcommand{\E}{\mathrm{E}}
\newcommand{\Var}{\mathrm{Var}}
\newcommand{\Cov}{\mathrm{Cov}}
\newcommand{\Bias}{\mathrm{Bias}}

% Formatting commands:

\newcommand{\mt}[1]{\ensuremath{#1}}
\newcommand{\nm}[1]{\textrm{#1}}

\newcommand\bsc[2][\DefaultOpt]{%
  \def\DefaultOpt{#2}%
  \section[#1]{#2}%
}
\newcommand\ssc[2][\DefaultOpt]{%
  \def\DefaultOpt{#2}%
  \subsection[#1]{#2}%
}
\newcommand{\bgpf}{\begin{proof} $ $\newline}

\newcommand{\bgeq}{\begin{equation*}}
\newcommand{\eeq}{\end{equation*}}	

\newcommand{\balist}{\begin{enumerate}[label=\alph*.]}
\newcommand{\elist}{\end{enumerate}}

\newcommand{\bilist}{\begin{enumerate}[label=\roman*)]}	

\newcommand{\bgsp}{\begin{split}}
% \newcommand{\esp}{\end{split}} % doesn't work for some reason.

\newcommand\prs[1]{~~~\textbf{(#1)}}

\newcommand{\lt}[1]{\textbf{Let: } #1}
     							   %  if you're setting it to be true
\newcommand{\supp}[1]{\textbf{Suppose: } #1}
     							   %  Suppose (if it'll end up false)
\newcommand{\wts}[1]{\textbf{Want to show: } #1}
     							   %  Want to show
\newcommand{\as}[1]{\textbf{Assume: } #1}
     							   %  if you think it follows from truth
\newcommand{\bpth}[1]{\textbf{(#1)}}

\newcommand{\step}[2]{\begin{equation}\tag{#2}#1\end{equation}}
\newcommand{\epf}{\end{proof}}

\newcommand{\sidenote}[1]{-----------------------------------------------------------------Side Notes---------------------------------------------------------------
#1 \

---------------------------------------------------------------------------------------------------------------------------------------------}

% Analysis / Logical commands:

\newcommand{\br}{\mathbb{R}}       % |R
\newcommand{\bq}{\mathbb{Q}}       % |Q
\newcommand{\bn}{\mathbb{N}}       % |N
\newcommand{\bc}{\mathbb{C}}       % |C
\newcommand{\bz}{\mathbb{Z}}       % |Z

\newcommand{\ep}{\epsilon}         % epsilon
\newcommand{\fa}{\forall}          % for all

\newcommand{\es}{\emptyset}        % empty set
\newcommand{\sbs}{\subset}         % subset of

\newcommand{\lra}{\longrightarrow} % implies ----->
\newcommand{\rar}{\Rightarrow}     % implies -->

\newcommand{\lla}{\longleftarrow}  % implies <-----
\newcommand{\lar}{\Leftarrow}      % implies <--

\newcommand{\pr}{^\prime} 		   % prime (i.e. R')

\newcommand{\bnm}[2]{\mt{#1\setminus{#2}}}
\newcommand{\bnt}[2]{\mt{\textrm{#1}\setminus{\textrm{#2}}}}
\newcommand{\bi}{\bnm{\mathbb{R}}{\mathbb{Q}}}

\newcommand{\nbho}[3]{\textrm{N(}#1, #2\textrm{) }\cap \textrm{ #3} \neq \emptyset}
     							   %  N(x, eps) intersect S \= emptyset
\newcommand{\nbhe}[3]{\textrm{N(}#1, #2\textrm{) }\cap \textrm{ #3} = \emptyset}
     							   %  N(x, eps) intersect S  = emptyset
\newcommand{\dnbho}[3]{\textrm{N*(}#1, #2\textrm{) }\cap \textrm{ #3} \neq \emptyset}
     							   %  N*(x, eps) intersect S \= emptyset
\newcommand{\dnbhe}[3]{\textrm{N*(}#1, #2\textrm{) }\cap \textrm{ #3} = \emptyset}
     							   %  N*(x, eps) intersect S = emptyset
     							 

\newcommand{\hmwkTitle}{Lecture\ \#2}

\begin{document}

\bsc{Theorem 3.3: The Completeness Axiom}{
Recall the Fundamental Theorem of Arithmetic: \

if n $\in \bn$ with n $\geq$ 2, then n may be expressed as the product of prime numbers (the prime factorization (PF)). \

The PF is unique with respect to (WRT) order. \

Ex: $12 = 2 * 2 * 2 * 3$
}
\bsc{Theorem 3.3.1}{
\lt{p be a prime number} \

Then $\sqrt{p} \in $ $\butnot{$\br$}{$\bq$}$ \
\bgpf

\as{$\sqrt{p} \in \bq$} \

Then $\sqrt{p}$ = $\frac{a}{b}$, where a, b $\in \bn$ and gcd(a, b) $=$ 1 \

So, \

p $= \frac{a^2}{b^2}$ \

$a^2 = pb^2$ \

therefore,

\step{p | a^2}{1}

$p | a^2$ $\rar$ $\exists k \in \bz$ st $a^2 = pk$ \

Since the PF of $a^2$ and a contain exactly the same distinct primes, \

(i.e. a $=$ p$_1 \times $ p$_2 \times$ ... p$_n$ $\rar$ a$^2 =$ p$^2_1 \times$ p$^2_2 \times$ ... $p^2_n$) \

and since p is prime (i.e. p is a component of a$^2$ but can't be, say, p$^2_2$ because that would mean it has an integer square root and therefore isn't prime), it has to be one of the p$_n$'s,

p $|$ a.

Thus, $\exists$ k $\in \bz$ st. a $=$ pk. \

Then a$^2$ = p$^2$k$^2$ = pb$^2$ from \bpth{1}.

Thus, b$^2$ = pk$^2$, and we see that p $|$ b$^2$. \

However, we obtain the contradiction that p $|$ b and p $|$ a. \

Hence, $\sqrt{p} \in \bi$.
\epf
}
\bsc{Definition 3.3.7}{
Let S $\sbs \br$.

If $\exists$ m $\in \br$ st s $\leq$ m $\fa$ s $\in S$, \

then m is an upper bound of S and we say that S is \textbf{bounded above}. \

Similarly, we can define \textbf{bounded below}. \

If S is bounded above and below, then S is said to be \textbf{bounded}.

--n--[--S--]--m

If an upper bound m of S is a member of S, then m is called the maximum (or largest element) of S, and we say that m = \textbf{max S}.

--n--[--S--m--

Similarly, we may decline \textbf{minimum of S (min S)}.
}

\bsc{Theorem 1}{
If a set S $\sbs \br$ possesses a max element, then it is unique. A similar result holds for a minimum element.
\bgpf
\supp{$\exists m_1, m_2 \in \br$ st $m_1 =$ max S, $m_2 = $ max S}

Thus, $m_1$, $m_2 \in S$ and, $\fa s \in$ S

\step{s \leq m_1}{1}
\step{s \leq m_2}{2}

Let m $=$ m$_2$ in \bpth{1} and m $=$ 

\epf
}

\end{document}